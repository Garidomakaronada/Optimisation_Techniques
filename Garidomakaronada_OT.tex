\documentclass[a4paper,11pt]{article}

%======================================================================
%   LLM INSTRUCTIONS & GROUNDING RULE
%======================================================================
% This template is to be used by the AI/LLM to generate exam solutions.
%
% *** GROUNDING RULE ***
% When generating a solution, if theory slides are provided, you MUST
% include a grounding truth based on the theory slides.
% Ensure that the solution is consistent with the provided materials.
%
% *** EXERCISE MAP INSTRUCTION ***
% You MUST also generate a "Map of Exercises" categorizing questions by type.
% This should list the exercise type followed by the exams where it appears.
% Example Format:
%   CYK: February 2025, September 2024
%   Turing Machines: June 2024, February 2022
%   ...
%
%======================================================================
%   COMPILATION INSTRUCTIONS FOR LLM
%======================================================================
% This template supports TWO compilation modes:
%
% ╔═══════════════════════════════════════════════════════════════════╗
% ║  MODE 1: READ MODE (Color - for screen viewing)                   ║
% ╠═══════════════════════════════════════════════════════════════════╣
% ║  Command: make read                                               ║
% ║  - Colorful boxes with green/blue accents                         ║
% ║  - Colored hyperlinks                                             ║
% ║  - Best for PDF viewers and tablets                               ║
% ╚═══════════════════════════════════════════════════════════════════╝
%
% ╔═══════════════════════════════════════════════════════════════════╗
% ║  MODE 2: PRINT MODE (B&W - for printing)                          ║
% ╠═══════════════════════════════════════════════════════════════════╣
% ║  Command: make print                                              ║
% ║  - Pure white backgrounds (saves ink)                             ║
% ║  - Black/gray frames and text                                     ║
% ║  - Black hyperlinks                                               ║
% ║  - Best for physical printing                                     ║
% ╚═══════════════════════════════════════════════════════════════════╝
%
% ADDITIONAL COMMANDS:
%   make all            - Build all versions
%   make clean          - Remove build artifacts
%   make read FILE=xyz  - Build a specific .tex file
%
%======================================================================
%   STRUCTURE GUIDELINES FOR LLM
%======================================================================
% Each exam section MUST follow this pattern:
%
%   \newpage
%   \phantomsection
%   \hypertarget{examN}{}
%   \pdfbookmark[1]{Month Year}{examN}
%   \examyear{Month Year}           % <-- REQUIRED: Sets page header
%
%   \begin{center}
%   \textbf{\Large Month Year -- Θέματα \& Λύσεις}
%   \end{center}
%
%   \begin{question}[ΘΕΜΑ X (Μονάδες: Y)]
%   ...question text...
%   \end{question}
%
%   \begin{answer}
%   ...solution...
%   \end{answer}
%
%======================================================================

%======================================================================
%   MODE FLAGS (passed via command line)
%======================================================================
% \printmodeflag  -> Print mode (B&W)
% Neither         -> Read mode (default)
%======================================================================
\newif\ifprintmode

\ifdefined\printmodeflag
    \printmodetrue
\else
    \printmodefalse
\fi

%======================================================================
%   EXAM YEAR HEADER
%======================================================================
\newcommand{\examyear}[1]{\def\currentexamyear{#1}\markboth{#1}{#1}}
\def\currentexamyear{}

%======================================================================
%	PREAMBLE & PACKAGES
%======================================================================
\usepackage{fontspec}
\usepackage{polyglossia}
\setmainlanguage{greek}
\setotherlanguage{english}

%======================================================================
%   FONT SELECTION (Read/Print Mode)
%======================================================================
% ═══════════════════════════════════════════════════════════════
% READ/PRINT MODE: Use Atkinson Hyperlegible
% ═══════════════════════════════════════════════════════════════
\setmainfont[
    Path = fonts/atkinson-hyperlegible/fonts/ttf/,
    Extension = .ttf,
    UprightFont = *-Regular,
    BoldFont = *-Bold,
    ItalicFont = *-Italic,
    BoldItalicFont = *-BoldItalic
]{AtkinsonHyperlegible}
\setsansfont[
    Path = fonts/atkinson-hyperlegible/fonts/ttf/,
    Extension = .ttf,
    UprightFont = *-Regular,
    BoldFont = *-Bold,
    ItalicFont = *-Italic,
    BoldItalicFont = *-BoldItalic
]{AtkinsonHyperlegible}
\setmonofont{DejaVu Sans Mono}
\newfontfamily\greekfont[Scale=MatchLowercase]{DejaVu Serif}
\newfontfamily\greekfontsf[Scale=MatchLowercase]{DejaVu Sans}

\usepackage{amsmath,amsfonts,amssymb}
\usepackage{listings}
\usepackage{booktabs}
\usepackage{array}
\usepackage{xcolor}
\usepackage[most]{tcolorbox}

%======================================================================
%   GEOMETRY (Normal)
%======================================================================
% Normal modes
\usepackage[top=1.5cm, bottom=1.25cm, left=1cm, right=1cm, headheight=14pt]{geometry}

\usepackage{hyperref}
\usepackage{pdfpages}
\usepackage{fancyhdr}
\usepackage{graphicx}
\usepackage{tikz}
\usetikzlibrary{shapes,arrows,positioning,calc,backgrounds,patterns}
\usepackage{tcolorbox}
\tcbuselibrary{breakable,skins}
\usepackage{enumitem}

%======================================================================
%   COLOR DEFINITIONS (Read/Print Mode)
%======================================================================
\ifprintmode
    % ═══════════════════════════════════════════════════════════════
    % PRINT MODE: Pure White Backgrounds / High Contrast Grayscale
    % ═══════════════════════════════════════════════════════════════
    \definecolor{answerbox}{gray}{1.0}       % White
    \definecolor{questionbg}{gray}{1.0}      % White
    \definecolor{partbg}{gray}{1.0}          % White
    \definecolor{answerframe}{gray}{0.1}     % Almost Black
    \definecolor{questionframe}{gray}{0.0}   % Black
    \definecolor{linkcolor}{gray}{0.0}       % Black
    \definecolor{headercolor}{gray}{0.1}     
    \definecolor{codelistingbg}{gray}{1.0}   % White

    % Distinct Grayscale Palette (Extremely High Contrast Gradient)
    % Designed to differentiate colors by shade intensity
    \definecolor{red}{gray}{0.15}      % Nearly Black (Critical)
    \definecolor{blue}{gray}{0.30}     % Dark Gray (Structural)
    \definecolor{green}{gray}{0.65}    % Mid-Light Gray (Success)
    \definecolor{cyan}{gray}{0.85}     % Very Light Gray (Auxiliary)
    \definecolor{orange}{gray}{0.50}   % Mid Gray (Warning)
    \definecolor{purple}{gray}{0.05}   % Blackest (Highlight)
    \definecolor{teal}{gray}{0.40}     % Mid-Dark
    \definecolor{brown}{gray}{0.25}    % Dark
    \definecolor{olive}{gray}{0.45}    % Mid
    \definecolor{magenta}{gray}{0.20}  % Very Dark
    \definecolor{yellow}{gray}{0.90}   % Faintest

    % PATTERN & FILL DEFINITIONS FOR PRINT MODE (High Legibility)
    \tikzset{
        fillcyan/.style={pattern=north east lines, pattern color=black!25},
        fillorange/.style={pattern=dots, pattern color=black!35},
        fillpurple/.style={pattern=crosshatch, pattern color=black!20},
        fillgreen/.style={pattern=grid, pattern color=black!20},
        fillred/.style={pattern=vertical lines, pattern color=black!30},
        fillgray/.style={fill=gray!10},
        fillblue/.style={pattern=fivepointed stars, pattern color=black!25},
        fillwhite/.style={fill=white}
    }

    \tikzset{
        every state/.append style={fill=gray!10, draw=black, thick},
        accepting/.append style={fill=gray!30, draw=black, very thick}
    }
\else
    % ═══════════════════════════════════════════════════════════════
    % READ MODE: Colorful design for screen viewing
    % ═══════════════════════════════════════════════════════════════
    \definecolor{answerbox}{RGB}{230,245,230}
    \definecolor{questionbg}{RGB}{240,248,255}
    \definecolor{partbg}{RGB}{255,250,240}
    \definecolor{answerframe}{RGB}{60,130,60}
    \definecolor{questionframe}{RGB}{50,80,140}
    \definecolor{linkcolor}{RGB}{0,0,180}
    \definecolor{headercolor}{RGB}{80,80,80}
    \definecolor{codelistingbg}{RGB}{245,245,245}

    % SOLID FILL DEFINITIONS FOR READ MODE
    \tikzset{
        fillcyan/.style={fill=cyan!40},
        fillorange/.style={fill=orange!50},
        fillpurple/.style={fill=purple!40},
        fillgreen/.style={fill=green!50},
        fillred/.style={fill=red!20},
        fillgray/.style={fill=gray!15},
        fillblue/.style={fill=blue!5},
        fillwhite/.style={fill=white}
    }
\fi

\lstset{
    basicstyle=\ttfamily\small,
    breaklines=true,
    frame=single,
    backgroundcolor=\color{codelistingbg},
    numbers=none,
    showstringspaces=false
}

%======================================================================
%   TCOLORBOX DEFINITIONS (Read/Print Mode)
%======================================================================
\newtcolorbox{answer}{
    colback=answerbox,
    colframe=answerframe,
    title=Λύση,
    fonttitle=\bfseries,
    boxrule=0.8pt,
    breakable,
    arc=2pt,
    left=6pt,
    right=6pt,
    top=4pt,
    bottom=4pt
}

\newtcolorbox{question}[1][]{
    colback=questionbg,
    colframe=questionframe,
    fonttitle=\bfseries,
    title=#1,
    boxrule=0.5pt,
    breakable
}

% Hyperlink styling
\hypersetup{
    colorlinks=true,
    linkcolor=linkcolor,
    bookmarks=true,
    bookmarksopen=true,
    pdfauthor={Όνομα Συγγραφέα},
    pdftitle={Τεχνικές Βελτιστοποίησης},
}

% Header/Footer Setup
\pagestyle{fancy}
\fancyhf{}
\fancyhead[C]{\textcolor{headercolor}{\small\bfseries\leftmark}}
\fancyfoot[C]{\thepage}
\renewcommand{\headrulewidth}{0.4pt}

\begin{document}


\thispagestyle{empty}
\begin{center}
    \vspace*{3cm}
    {\Huge \textbf{Τεχνικές Βελτιστοποίησης}}\\[1em]
    {\Large Συλλογή Παλαιών Θεμάτων \& Λύσεων Γαριδομακαροναδα}\\[3em]
\end{center}

\vfill
\begin{center}
{\small Επιμέλεια: VM+DK}\\[1em]
{\small Credits στα παιδιά που βγάλανε τις φωτογραφίες, στα παιδιά που βοήθησαν με τις λύσεις και στον Αστακομακαροναδα που ξεκίνησε την ιδέα.}\\[1em]
{\footnotesize \textbf{Disclaimer:} Οι παρούσες λύσεις και εκφωνήσεις ενδέχεται να περιέχουν λάθη. Κάντε comment στο github μου για τα λάθη.}
\end{center}

\newpage

\pdfbookmark[1]{Περιεχόμενα}{toc}
\section*{Περιεχόμενα}

\begin{itemize}
    \item[\hyperlink{iounios2014}{\textbf{1.}}] \hyperlink{iounios2014}{\textbf{Ιούνιος 2014}}
    \item[\hyperlink{ioulios2018}{\textbf{2.}}] \hyperlink{ioulios2018}{\textbf{Ιούλιος 2018}}
    \item[\hyperlink{febrouarios2020}{\textbf{3.}}] \hyperlink{febrouarios2020}{\textbf{Φεβρουάριος 2020}}
    \item[\hyperlink{fevrouarios2021}{\textbf{4.}}] \hyperlink{fevrouarios2021}{\textbf{Φεβρουάριος 2021}}
    \item[\hyperlink{septembrios2021}{\textbf{5.}}] \hyperlink{septembrios2021}{\textbf{Σεπτέμβριος 2021}}
    \item[\hyperlink{fevrouarios2022}{\textbf{6.}}] \hyperlink{fevrouarios2022}{\textbf{Φεβρουάριος 2022}}
    \item[\hyperlink{iounios2022}{\textbf{7.}}] \hyperlink{iounios2022}{\textbf{Ιούνιος 2022}}
    \item[\hyperlink{septembrios2022}{\textbf{8.}}] \hyperlink{septembrios2022}{\textbf{Σεπτέμβριος 2022}}
    \item[\hyperlink{fevrouarios2023}{\textbf{9.}}] \hyperlink{fevrouarios2023}{\textbf{Φεβρουάριος 2023}}
    \item[\hyperlink{iounios2023}{\textbf{10.}}] \hyperlink{iounios2023}{\textbf{Ιούνιος 2023}}
    \item[\hyperlink{septembrios2023}{\textbf{11.}}] \hyperlink{septembrios2023}{\textbf{Σεπτέμβριος 2023}}
    \item[\hyperlink{iounios2024}{\textbf{12.}}] \hyperlink{iounios2024}{\textbf{Ιούνιος 2024}}
    \item[\hyperlink{septembrios2024}{\textbf{13.}}] \hyperlink{septembrios2024}{\textbf{Σεπτέμβριος 2024}}
    \item[\hyperlink{fevrouarios2025}{\textbf{14.}}] \hyperlink{fevrouarios2025}{\textbf{Φεβρουάριος 2025}}
    \item[\hyperlink{ioulios2025}{\textbf{15.}}] \hyperlink{ioulios2025}{\textbf{Ιούλιος 2025}}
    \item[\hyperlink{septembrios2025}{\textbf{16.}}] \hyperlink{septembrios2025}{\textbf{Σεπτέμβριος 2025}}
\end{itemize}
\newpage

% EXERCISE MAP PLACEHOLDER




\part*{Λυμένες Ασκήσεις Βιβλίου \& Θεωρία}
\phantomsection
\addcontentsline{toc}{section}{Επιπλέον Λυμένες Ασκήσεις}

%	SECTION 1: Ασκήσεις 3.6.3
%----------------------------------------------------------------------
\phantomsection
\hypertarget{exercises1}{}
\pdfbookmark[1]{Ασκήσεις 3.6.3}{exercises1}
\examyear{Ασκήσεις 3.6.3}

\begin{center}
\textbf{\Large Ασκήσεις 3.6.3 -- Σελίδα 71}
\end{center}

\begin{question}[Άσκηση 3.6.3 (α)]
Δείξτε ότι:
\[ \frac{x}{4} + \frac{3y}{4} \leq \sqrt{\ln \left( \frac{e^{x^2}}{4} + \frac{3}{4} e^{y^2} \right)}, \quad \forall x, y > 0 \]
\end{question}

\begin{answer}
\textbf{Λύση:}

Θέλουμε να δείξουμε ότι: 
\[ \frac{x}{4} + \frac{3y}{4} \leq \sqrt{\ln \left( \frac{e^{x^2}}{4} + \frac{3}{4} e^{y^2} \right)} \implies \left( \frac{x + 3y}{4} \right)^2 \leq \ln \left( \frac{e^{x^2}}{4} + \frac{3}{4} e^{y^2} \right) \quad (1) \]

Θεωρούμε τη συνάρτηση $\phi(t) = t^2$, η οποία είναι \textbf{κυρτή}.
Από την ανισότητα Jensen:
\[ f \left( \sum_{i=1}^k \lambda_i x_i \right) \leq \sum_{i=1}^k \lambda_i f(x_i) \]

Για $\lambda_1 = \frac{1}{4}$, $\lambda_2 = \frac{3}{4}$ ($\lambda_1 + \lambda_2 = 1$):
\[ \left( \frac{1}{4}x + \frac{3}{4}y \right)^2 \leq \frac{1}{4}x^2 + \frac{3}{4}y^2 \quad (2) \]

Επίσης, θεωρούμε τη συνάρτηση $\psi(t) = -\ln t$, η οποία είναι \textbf{κυρτή} (ή $\phi(t) = \ln t$ κοίλη).
Εφαρμόζοντας πάλι την ανισότητα Jensen για την $\ln t$:
\[ \ln \left( \sum \lambda_i z_i \right) \geq \sum \lambda_i \ln z_i \]

Για $z_1 = e^{x^2}$ και $z_2 = e^{y^2}$:
\[ \ln \left( \frac{1}{4} e^{x^2} + \frac{3}{4} e^{y^2} \right) \geq \frac{1}{4} \ln(e^{x^2}) + \frac{3}{4} \ln(e^{y^2}) = \frac{1}{4} x^2 + \frac{3}{4} y^2 \quad (3) \]

Από τις ανισότητες (2) και (3) προκύπτει:
\[ \ln \left( \frac{1}{4} e^{x^2} + \frac{3}{4} e^{y^2} \right) \geq \frac{1}{4} x^2 + \frac{3}{4} y^2 \geq \left( \frac{x + 3y}{4} \right)^2 \]
Άρα:
\[ \frac{x}{4} + \frac{3y}{4} \leq \sqrt{\ln \left( \frac{e^{x^2}}{4} + \frac{3}{4} e^{y^2} \right)} \]
\end{answer}

\begin{question}[Άσκηση 3.6.3 (β)]
Δείξτε ότι:
\[ \left( \frac{x}{2} + \frac{y}{3} + \frac{z}{12} + \frac{w}{12} \right)^4 \leq \frac{1}{2} x^4 + \frac{1}{3} y^4 + \frac{1}{12} z^4 + \frac{1}{12} w^4 \]
με την ισότητα να ισχύει αν και μόνο αν $x=y=z=w$.
\end{question}

\begin{answer}
\textbf{Θεωρία:} Θεώρημα 3.1.3 (Jensen). Έστω $f(x)$ κυρτή συνάρτηση ορισμένη σε κυρτό υποσύνολο του $\mathbb{R}^n$. Αν $\lambda_i \geq 0$ για $i=1, \dots, k$ και $\sum_{i=1}^k \lambda_i = 1$, τότε:
\[ f \left( \sum_{i=1}^k \lambda_i x_i \right) \leq \sum_{i=1}^k \lambda_i f(x_i) \quad (1) \]

Αν η $f$ είναι \textbf{γνήσια κυρτή} και $\lambda_i > 0, i=1,\dots,k$, τότε η ισότητα στην (1) ισχύει αν και μόνο αν τα $x_i, i=1,\dots,k$ είναι \textbf{ίσα}.
\end{answer}

\newpage
%----------------------------------------------------------------------
%	SECTION 2: Άσκηση 3.5.8
%----------------------------------------------------------------------
\phantomsection
\hypertarget{exercises2}{}
\pdfbookmark[1]{Άσκηση 3.5.8}{exercises2}
\examyear{Άσκηση 3.5.8}

\begin{center}
\textbf{\Large Άσκηση 3.5.8 -- Σελίδα 55}
\end{center}

\begin{question}[Ανισότητα Αριθμητικού-Γεωμετρικού Μέσου]
Έστω $x_i, i=1,2,\dots,n$ θετικοί πραγματικοί αριθμοί και $\delta_i > 0, i=1,2,\dots,n$ με $\sum_{i=1}^n \delta_i = 1$. Τότε:
\[ \prod_{i=1}^n x_i^{\delta_i} \leq \sum_{i=1}^n \delta_i x_i \]
με την ισότητα να ισχύει αν και μόνο αν $x_1 = x_2 = \dots = x_n$.
\end{question}

\begin{answer}
\textbf{Απόδειξη της Ανισότητας με τη χρήση του Θεωρήματος 3.1.3:}

Θεωρώ την $f(z) = -\ln z, z > 0$.
Η $f(z)$ είναι γνήσια κυρτή. Έτσι, μπορώ να εφαρμόσω το Θεώρημα 3.1.3 για κάποια $x_i$ και $\delta_i$ τα οποία $\sum_{i=1}^n \delta_i = 1$.

Ισχύει ότι:
\[ -\ln \left( \sum_{i=1}^n \delta_i x_i \right) \leq \sum_{i=1}^n \delta_i (-\ln x_i) = -\sum_{i=1}^n \delta_i \ln(x_i) \implies \]
\[ \ln \left( \sum_{i=1}^n \delta_i x_i \right) \geq \sum_{i=1}^n \delta_i \ln(x_i) \implies \]
χρησιμοποιώ την ιδιότητα λογαρίθμου $\ln(a) + \ln(b) = \ln(ab)$:
\[ \ln \left( \sum_{i=1}^n \delta_i x_i \right) \geq \sum_{i=1}^n \ln(x_i^{\delta_i}) = \ln \left( \prod_{i=1}^n x_i^{\delta_i} \right) \implies \]
Επειδή η $\ln$ είναι γνησίως αύξουσα (${\ln \uparrow}$):
\[ \sum_{i=1}^n \delta_i x_i \geq \prod_{i=1}^n x_i^{\delta_i} \implies \boxed{\prod_{i=1}^n x_i^{\delta_i} \leq \sum_{i=1}^n \delta_i x_i} \]
με την ισότητα να ισχύει αν και μόνο αν $x_1 = x_2 = \dots = x_n$.
\end{answer}

\newpage
%----------------------------------------------------------------------
%	SECTION 3: Άσκηση 3.6.9
%----------------------------------------------------------------------
\phantomsection
\hypertarget{exercises3}{}
\pdfbookmark[1]{Άσκηση 3.6.9}{exercises3}
\examyear{Άσκηση 3.6.9}

\begin{center}
\textbf{\Large Άσκηση 3.6.9 -- Σελίδα 72 (Άλυτες Ασκήσεις)}
\end{center}

\begin{question}[Χρήση της Ανισότητας Hölder]
Κάνοντας χρήση της ανισότητας του Hölder, δείξτε ότι αν $a_i, i=1,2,\dots,m$ είναι σταθερά διανύσματα στο $\mathbb{R}^n$ και αν $c_i > 0, i=1,2,\dots,m$, τότε η:
\[ f(x) = \ln \left( \sum_{i=1}^m c_i e^{a_i^T x} \right) \]
είναι κυρτή στο $\mathbb{R}^n$.
\end{question}

\begin{answer}
Ουσιαστικά η εκφώνηση μας δίνει μία συνάρτηση $f(x)$ και μας λέει να αποδείξουμε ότι είναι κυρτή στο πεδίο ορισμού της. (Η απόδειξη της κυρτότητας της $f(x)$ σύμφωνα με την άσκηση πρέπει να γίνει με την χρήση της ανισότητας Hölder).

\begin{tcolorbox}[colback=partbg, colframe=blue!30!black, title=Ανισότητα Hölder (σελ. 61)]
Αν $p, q > 1$ και $\frac{1}{p} + \frac{1}{q} = 1$, τότε $\forall x, y \in \mathbb{R}^n$:
\[ \sum_{i=1}^n |x_i y_i| \leq \left( \sum_{i=1}^n |x_i|^p \right)^{\frac{1}{p}} \left( \sum_{i=1}^n |y_i|^q \right)^{\frac{1}{q}} \]
\end{tcolorbox}

\textbf{Λύση:}

Για να είναι κυρτή η $f(x)$ πρέπει να ισχύει το Θ. 3.1.3 (Jensen).
Θα γράψουμε την ανισότητα αυτή του Θ. 3.1.3 και αν αυτή ικανοποιείται, τότε η $f(x)$ θα είναι κυρτή.
(Αυτό θα το κάνουμε για 2 σημεία με 2 συντελεστές).

1) Επιλέγουμε τους συντελεστές $\lambda_i$ με $i=1,2$ ως εξής:
\[ \lambda_1 = \frac{1}{p} \quad \text{και} \quad \lambda_2 = \frac{1}{q} \]
Ισχύει ότι: $\lambda_1 + \lambda_2 = \frac{1}{p} + \frac{1}{q} = 1$.

Από το Θεώρημα 3.1.3 θα πρέπει να ελέγξω κατά πόσο το παρακάτω είναι αληθές:
\[ \ln \left( \sum_{i=1}^m c_i e^{a_i^T (\frac{1}{p}x + \frac{1}{q}y)} \right) \leq \frac{1}{p} \ln \left( \sum_{i=1}^m c_i e^{a_i^T x} \right) + \frac{1}{q} \ln \left( \sum_{i=1}^m c_i e^{a_i^T y} \right) \quad (1) \]
Β' ΜΕΛΟΣ = \[ \frac{1}{p} \ln \left( \sum_{i=1}^m c_i e^{a_i^T x} \right) + \frac{1}{q} \ln \left( \sum_{i=1}^m c_i e^{a_i^T y} \right) = \]
\[ = \ln \left( \sum_{i=1}^m c_i e^{a_i^T x} \right)^{\frac{1}{p}} + \ln \left( \sum_{i=1}^m c_i e^{a_i^T y} \right)^{\frac{1}{q}} = \]
χρησ. ιδιότητα λογαρίθμων: $\ln a + \ln b = \ln(ab)$
\[ = \ln \left[ \left( \sum_{i=1}^m c_i e^{a_i^T x} \right)^{\frac{1}{p}} \left( \sum_{i=1}^m c_i e^{a_i^T y} \right)^{\frac{1}{q}} \right] \quad (2) \]

Αυτό έχει αρχίσει να μοιάζει με την ανισότητα του Hölder που μου λέει η εκφώνηση να χρησιμοποιήσω.
Δουλεύω λίγο ακόμη το Β' μέλος, για να το κάνω ακριβώς να μοιάζει με το Hölder...
Παίρνω λοιπόν λίγο την ποσότητα του $\ln$ του Β' μέλους:
\[ \left( \sum_{i=1}^m c_i e^{a_i^T x} \right)^{\frac{1}{p}} \left( \sum_{i=1}^m c_i e^{a_i^T y} \right)^{\frac{1}{q}} \quad (3) \]

Άρα, έχουμε ότι:
\[ \left( \sum_{i=1}^m c_i e^{a_i^T x} \right)^{\frac{1}{p}} \left( \sum_{i=1}^m c_i e^{a_i^T y} \right)^{\frac{1}{q}} = \]
\[ = \left( \sum_{i=1}^m \left| c_i^{\frac{1}{p}} e^{\frac{1}{p} a_i^T x} \right|^p \right)^{\frac{1}{p}} \left( \sum_{i=1}^m \left| c_i^{\frac{1}{q}} e^{\frac{1}{q} a_i^T y} \right|^q \right)^{\frac{1}{q}} \]
Αυτή η ποσότητα είναι ακριβώς στην ίδια μορφή με την ανισότητα Hölder.
Οπότε, εφαρμόζοντας την ανισότητα Hölder:
\[ \left( \sum_{i=1}^m \left| c_i^{\frac{1}{p}} e^{\frac{1}{p} a_i^T x} \right|^p \right)^{\frac{1}{p}} \left( \sum_{i=1}^m \left| c_i^{\frac{1}{q}} e^{\frac{1}{q} a_i^T y} \right|^q \right)^{\frac{1}{q}} \geq \]
\[ \geq \sum_{i=1}^m \left| c_i^{\frac{1}{p}} e^{\frac{1}{p} a_i^T x} \cdot c_i^{\frac{1}{q}} e^{\frac{1}{q} a_i^T y} \right| \]
βγάζω την απόλυτη τιμή γιατί όλες οι ποσότητες είναι θετικές:
\[ = \sum_{i=1}^m c_i^{(\frac{1}{p} + \frac{1}{q})} e^{\frac{1}{p} a_i^T x + \frac{1}{q} a_i^T y} \xrightarrow{\ln \uparrow} \dots \]
\[ \implies \ln \left( \sum_{i=1}^m c_i e^{a_i^T (\frac{1}{p} x + \frac{1}{q} y)} \right) \leq \ln \left[ \left( \sum_{i=1}^m c_i e^{a_i^T x} \right)^{\frac{1}{p}} \left( \sum_{i=1}^m c_i e^{a_i^T y} \right)^{\frac{1}{q}} \right] \]
\[ \implies \ln \left( \sum_{i=1}^m c_i e^{a_i^T (\frac{1}{p} x + \frac{1}{q} y)} \right) \leq \frac{1}{p} \ln \left( \sum_{i=1}^m c_i e^{a_i^T x} \right) + \frac{1}{q} \ln \left( \sum_{i=1}^m c_i e^{a_i^T y} \right) \]

Η αρχική ανισότητα (1) είναι αληθής, άρα η $f$ είναι κυρτή.
\end{answer}

\newpage
%----------------------------------------------------------------------
%	SECTION 4: Θεώρημα 3.1.4, 3.1.6 & Συμπέρασμα 3.1.1
%----------------------------------------------------------------------
\phantomsection
\hypertarget{theorems1}{}
\pdfbookmark[1]{Θεωρήματα \& Συμπεράσματα}{theorems1}
\examyear{Θεωρήματα}

\begin{center}
\textbf{\Large Βασικά Θεωρήματα \& Συμπεράσματα}
\end{center}

\begin{answer}
Ξεκινάμε με τα θεωρήματα που θα χρησιμοποιήσουμε σήμερα, για να λύσουμε τις ασκήσεις...

\begin{tcolorbox}[colback=partbg, colframe=blue!30!black, title=Θεώρημα 3.1.4]
Κάθε τοπικό ελάχιστο μιας κυρτής συνάρτησης $f$ ορισμένης σε κάποιο κυρτό υποσύνολο του $S \subseteq \mathbb{R}^n$ είναι και ολικό ελάχιστο.
Αν η $f$ είναι γνήσια κυρτή, τότε το ολικό ελάχιστο είναι και μοναδικό.
\end{tcolorbox}

\begin{tcolorbox}[colback=partbg, colframe=blue!30!black, title=Συμπέρασμα 3.1.1]
Αν η $f$ είναι κυρτή με συνεχείς μερικές παραγώγους $1^{\eta \varsigma}$ τάξης ορισμένη στο $S \subseteq \mathbb{R}^n$, τότε κάθε κρίσιμο σημείο της $f$, αποτελεί και ολικό ελάχιστο.
\end{tcolorbox}

\begin{tcolorbox}[colback=partbg, colframe=blue!30!black, title=Θεώρημα 3.1.6]
Αν η $f$ έχει συνεχείς παραγώγους $2^{\eta \varsigma}$ τάξης σε ανοιχτό κυρτό υποσύνολο $S \subseteq \mathbb{R}^n$ και αν ο $\nabla^2 f(x)$ είναι θετικά ημιορισμένος, τότε η $f$ είναι κυρτή, ενώ αν ο $\nabla^2 f(x)$ είναι θετικά ορισμένος η $f$ είναι γνήσια κυρτή.
\end{tcolorbox}

\begin{tcolorbox}[colback=partbg, colframe=blue!30!black, title=Θεώρημα 3.1.7]
\begin{enumerate}[label=\roman*)]
    \item Αν έχω κάποιες συναρτήσεις $f_i(x), i=1,\dots,k$ που είναι κυρτές σε κυρτό υποσύνολο $S \subseteq \mathbb{R}^n$, τότε η $f(x) = \sum_{i=1}^k f_i(x)$ είναι κυρτή συνάρτηση.
    Αν τουλάχιστον μία από τις $f_i(x)$ είναι γνήσια κυρτή, τότε και η $f(x)$ είναι γνήσια κυρτή.
    \item Για κάποιο $a > 0$ η $a f(x)$ είναι (γνήσια) κυρτή, αν η $f(x)$ είναι (γνήσια) κυρτή.
    \item Αν η $f$ είναι (γνήσια) κυρτή και η $g$ είναι κυρτή και (γνησίως) αύξουσα, τότε η $g(f(x))$ είναι (γνήσια) κυρτή.
\end{enumerate}
\end{tcolorbox}
\end{answer}

\newpage
%----------------------------------------------------------------------
%	SECTION 5: Άσκηση 3.6.1
%----------------------------------------------------------------------
\phantomsection
\hypertarget{exercises5}{}
\pdfbookmark[1]{Άσκηση 3.6.1}{exercises5}
\examyear{Άσκηση 3.6.1}

\begin{center}
\textbf{\Large Άσκηση 3.6.1 -- Σελίδα 70}
\end{center}

\begin{question}[Κυρτότητα Συναρτήσεων]
Δείξτε αν οι παρακάτω συναρτήσεις είναι κυρτές ή αυστηρά (γνήσια) κυρτές:
\[ \delta) f(x_1, x_2) = 4 e^{3x_1 - x_2} + 5 e^{x_1^2 + x_2^2}, \quad x = [x_1 \ x_2]^T \in \mathbb{R}^2 \]
\end{question}

\begin{answer}
\textbf{Λύση:}

Ορίζω την $f_1(x_1, x_2) = 4 e^{3x_1 - x_2}$ και την $g_1(y) = 4e^y$.
Επίσης ορίζω την $h_1(x_1, x_2) = 3x_1 - x_2$.

Γνωρίζουμε ότι κάθε γραμμικός συνδυασμός της μορφής $f(x) = a^T x + b, x \in \mathbb{R}^n, b \in \mathbb{R}, a \in \mathbb{R}^n$ είναι κυρτή συνάρτηση (δες παραδ. 3.3.1 σελ. 25).
Επομένως, η $h_1(x_1, x_2) = 3x_1 - x_2$ είναι \textbf{κυρτή} συνάρτηση.

Επίσης, η $g_1(y)$ είναι \textbf{γνησίως αύξουσα} και \textbf{γνήσια κυρτή}, διότι:
\[ g_1'(y) = 4e^y > 0 \quad \text{και} \quad g_1''(y) = 4e^y > 0 \]

Παρατηρώ ότι $f_1(x_1, x_2) = g_1(h_1)$.
Επειδή η $h_1$ είναι κυρτή συνάρτηση και η $g_1$ είναι γνήσια κυρτή και γνησίως αύξουσα, η $f_1(x_1, x_2) = g_1(h_1)$ είναι \textbf{κυρτή} (Εδώ χρησιμοποιούμε ουσιαστικά το Θ. 3.1.7 (iii)).

Ορίζω τώρα τις:
$f_2(x_1, x_2) = 5 e^{x_1^2 + x_2^2}$
$g_2(y) = 5 e^y$
$h_2(x_1, x_2) = x_1^2 + x_2^2$

Έχουμε ότι:
\begin{itemize}
    \item H $h_2(x_1, x_2)$ είναι \textbf{γνήσια κυρτή} συνάρτηση ως άθροισμα δύο γνήσιων κυρτών συναρτήσεων ($q_1(x_1) = x_1^2, q_2(x_2) = x_2^2$).
    (Σημείωση: Οι $q_1(x_1), q_2(x_2)$ είναι γνήσια κυρτές συναρτήσεις γιατί είναι της μορφής $(a^T x)^2$).
    \item Η $g_2(y)$ είναι \textbf{γνήσια κυρτή} και \textbf{γνησίως αύξουσα}.
\end{itemize}
Έτσι, λόγω του Θ. 3.1.7 (iii) η $f_2(x_1, x_2) = g_2(h_2)$ είναι \textbf{γνήσια κυρτή} συνάρτηση.

Παρατηρούμε ότι η $f(x) = f_1(x) + f_2(x)$. Η $f_1(x)$ (όπως δείξαμε) είναι κυρτή, ενώ η $f_2(x)$ είναι γνήσια κυρτή. Έτσι, από το Θ. 3.1.7 (i) η $f(x)$ είναι \textbf{γνήσια κυρτή}.
\end{answer}
\newpage
%----------------------------------------------------------------------
%	SECTION 6: Άσκηση 3.6.5
%----------------------------------------------------------------------
\phantomsection
\hypertarget{exercises6}{}
\pdfbookmark[1]{Άσκηση 3.6.5}{exercises6}
\examyear{Άσκηση 3.6.5}

\begin{center}
\textbf{\Large Άσκηση 3.6.5}
\end{center}

\begin{question}[Χρήση Α-Γ Ανισότητας]
Χρησιμοποιείστε την (Α-Γ) ανισότητα για να λύσετε τα παρακάτω προβλήματα:

a) $\min (x^2 + y + z)$
υπό τους περιορισμούς:
$xyz = 1$
$x, y, z > 0$
\end{question}

\begin{answer}
\textbf{Λύση:}

Αρχικά, πάμε να θυμήσουμε την (Α-Γ) $\to$ Ανισότητα Αριθμητικού Γεωμετρικού Μέσου:

\begin{tcolorbox}[colback=partbg, colframe=blue!30!black, title=Ανισότητα Α-Γ]
Έστω $x_i, i=1,\dots,n, x_i > 0$ και $\delta_i, i=1,\dots,n, \delta_i > 0$ και $\sum_{i=1}^n \delta_i = 1$, τότε:
\[ \prod_{i=1}^n x_i^{\delta_i} \leq \sum_{i=1}^n \delta_i x_i \]
\end{tcolorbox}

\begin{itemize}
    \item Έστω ότι υπάρχουν κάποια $\lambda_1, \lambda_2, \lambda_3 \geq 0$ τέτοια ώστε $\sum_{i=1}^3 \lambda_i = 1$.
    \item Από (Α-Γ):
    \[ \lambda_1(x^2) + \lambda_2(y) + \lambda_3(z) \geq (x^2)^{\lambda_1} \cdot y^{\lambda_2} \cdot z^{\lambda_3} \]
    \[ \implies \boxed{(x^2)^{\lambda_1} \cdot y^{\lambda_2} \cdot z^{\lambda_3} \leq \lambda_1(x^2) + \lambda_2(y) + \lambda_3(z)} \quad (1) \]
    \item Θα πρέπει: $2\lambda_1 = \lambda_2 = \lambda_3$ (2) και $\lambda_1 + \lambda_2 + \lambda_3 = 1$ (3).
    \item Από τις (2) και (3) έχουμε ότι:
    \[ 5\lambda_1 = 1 \implies \lambda_1 = \frac{1}{5} \quad \text{και} \quad \lambda_2 = \lambda_3 = \frac{2}{5} \]
\end{itemize}

Αντικαθιστώντας στην (1) έχουμε ότι:
\[ x^2 + y + z = 5 \left( \frac{1}{5} x^2 + \frac{2}{5} \left( \frac{y}{2} \right) + \frac{2}{5} \left( \frac{z}{2} \right) \right) \geq \]
\[ \geq 5 \left( x^2 \right)^{\frac{1}{5}} \cdot \left( \frac{y}{2} \right)^{\frac{2}{5}} \cdot \left( \frac{z}{2} \right)^{\frac{2}{5}} \implies \]
\[ \implies x^2 + y + z \geq 5 \left( \frac{x \cdot y \cdot z}{2 \cdot 2} \right)^{\frac{2}{5}} \implies x^2 + y + z \geq 5 \frac{(xyz)^\frac{2}{5}}{4^\frac{2}{5}} \]

Από τους περιορισμούς ($xyz=1$):
\[ \implies x^2 + y + z \geq \frac{5}{4^\frac{2}{5}} \implies \boxed{x^2 + y + z \geq \frac{5}{\sqrt[5]{4^2}}} \quad (4) \]

Για ποιες τιμές του $x, y, z$ παίρνουμε την ισότητα;
Από την θεωρία της Α-Γ η ισότητα ισχύει όταν τα $x_i$ είναι ίσα μεταξύ τους.
Έτσι, η ισότητα στην (4) ισχύει για: $x^2 = \frac{y}{2} = \frac{z}{2}$.
Όμως: $xyz=1$ και $x,y,z > 0$.
Έτσι, λύνοντας το σύστημα:
\[
\begin{cases}
xyz=1 \\
x,y,z > 0 \\
x^2 = \frac{y}{2} = \frac{z}{2}
\end{cases}
\]

βρίσκουμε το σημείο $(x,y,z)$ για το οποίο η $x^2+y+z$ ελαχιστοποιείται.
Το (τοπικό) ελάχιστο της $x^2+y+z$ είναι το: $\frac{5}{4^{2/5}}$.

Και επειδή η $x^2+y+z$ είναι γνήσια κυρτή, το $\frac{5}{4^{2/5}}$ είναι και το ολικό (και μάλιστα μοναδικό) ελάχιστο.
\end{answer}
\newpage
%----------------------------------------------------------------------
%	SECTION 7: Άσκηση 3.6.10
%----------------------------------------------------------------------
\phantomsection
\hypertarget{exercises7}{}
\pdfbookmark[1]{Άσκηση 3.6.10}{exercises7}
\examyear{Άσκηση 3.6.10}

\begin{center}
\textbf{\Large Άσκηση 3.6.10 -- Σελίδα 72}
\end{center}

\begin{question}[Μέγιστος Κύλινδρος σε Σφαίρα]
Κάνοντας χρήση της (Α-Γ) ανισότητας λύστε τα παρακάτω προβλήματα:

a) Βρείτε τον μεγαλύτερο κύλινδρο που μπορεί να εγγραφεί σε σφαίρα δοσμένης ακτίνας.
\end{question}

\begin{answer}
\textbf{Λύση:}

\begin{itemize}
    \item Έστω $r_s$ ακτίνα σφαίρας.
    \item Ξέρουμε ότι ο όγκος του κυλίνδρου είναι: $\boxed{V_c = \pi \cdot r_c^2 \cdot h}$, όπου $r_c$ η ακτίνα του κυλίνδρου και $h$ το ύψος του κυλίνδρου. (Αυτή είναι η συνάρτηση προς μεγιστοποίηση).
\end{itemize}

\textbf{Σχήμα:}
Παρακάτω βλέπουμε μια τομή του προβλήματος της σφαίρας:

\begin{center}
\begin{tikzpicture}[scale=1.5]
    % Sphere (Circle)
    \draw[thick] (0,0) circle (1.5cm);
    % Cylinder (Rectangle) - inscribed
    % Let h/2 = 0.8, then rc = sqrt(1.5^2 - 0.8^2) = sqrt(2.25 - 0.64) = sqrt(1.61) approx 1.27
    \draw[thick, fill=gray!10] (-1.27, -0.8) rectangle (1.27, 0.8);
    
    % Dimensions
    \draw[dashed] (0,0) -- (1.27, 0) node[midway, below] {$r_c$};
    \draw[dashed] (0,0) -- (0, 0.8) node[midway, left] {$\frac{h}{2}$};
    \draw[->, thick] (0,0) -- (1.27, 0.8) node[midway, above left] {$r_s$};
    
    % Center point
    \filldraw (0,0) circle (1pt);
\end{tikzpicture}
\end{center}

Από πυθαγόρειο θεώρημα έχουμε ότι:
\[ \boxed{r_c^2 + \frac{h^2}{4} = r_s^2} \]

Παρατηρώ ότι:
\[ 3 \left( \frac{2}{3} \left( \frac{r_c^2}{2} \right) + \frac{1}{3} \left( \frac{h^2}{4} \right) \right) = r_c^2 + \frac{h^2}{4} \xrightarrow{(A-\Gamma)} \]
\[ \geq 3 \left( \left( \frac{r_c^2}{2} \right)^{\frac{2}{3}} \left( \frac{h^2}{4} \right)^{\frac{1}{3}} \right) \implies r_s^2 \geq \frac{3}{2^{\frac{2}{3}} 4^{\frac{1}{3}}} (r_c^2)^{\frac{2}{3}} (h^2)^{\frac{1}{3}} \]
\[ \implies r_s^2 \geq \frac{3}{2^{\frac{2}{3}} 4^{\frac{1}{3}} \pi^{\frac{2}{3}}} (\pi r_c^2 h)^{\frac{2}{3}} \]
\[ \implies r_s^2 \geq \frac{3}{2^{\frac{2}{3}} 4^{\frac{1}{3}} \pi^{\frac{2}{3}}} (V_c)^{\frac{2}{3}} \]
Λύνοντας ως προς $V_c$:
\[ \implies \boxed{V_c \leq \frac{4\pi r_s^3}{3 \sqrt{3}}} \quad (\text{ή } \frac{4\pi r_s^3}{3^{\frac{3}{2}}}) \]

Για την ισότητα έχουμε τον μέγιστο όγκο του κυλίνδρου, που μπορεί να εγγραφεί στην σφαίρα δοσμένης ακτίνας.
\begin{itemize}
    \item Η ισότητα ισχύει όταν: $\frac{r_c^2}{2} = \frac{h^2}{4} \quad (1)$
    \item Όμως $r_c^2 + \frac{h^2}{4} = r_s^2 \quad (2)$
\end{itemize}

Έτσι, λύνοντας το σύστημα (1), (2) μπορούμε να βρούμε τα $h, r_c$ συναρτήσει της $r_s$.
\end{answer}
\newpage
%----------------------------------------------------------------------
%	SECTION 8: Άσκηση 3.6.8
%----------------------------------------------------------------------
\phantomsection
\hypertarget{exercises8}{}
\pdfbookmark[1]{Άσκηση 3.6.8}{exercises8}
\examyear{Άσκηση 3.6.8}

\begin{center}
\textbf{\Large Άσκηση 3.6.8 -- Σελίδα 72}
\end{center}

\begin{question}[Ελαχιστοποίηση Συνάρτησης]
Βρείτε τις τιμές του $x > 0$ που ελαχιστοποιούν τη συνάρτηση:
\[ f(x) = c_1 x^3 + \frac{c_2}{x}, \quad \text{με } c_1, c_2 \text{ θετικές σταθερές.} \]
\end{question}

\begin{answer}
\textbf{Λύση:}

\[ f'(x) = 3 c_1 x^2 - \frac{c_2}{x^2} \]

Επειδή $x > 0$ βρίσκουμε πότε η $f'(x) = 0$:
\[ f'(x) = 0 \implies 3 c_1 x^2 - \frac{c_2}{x^2} = 0 \xrightarrow{x \neq 0} 3 c_1 x^4 - c_2 = 0 \implies \]
\[ \implies 3 c_1 x^4 = c_2 \implies x^4 = \frac{c_2}{3 c_1} \implies \boxed{x^* = \left( \frac{c_2}{3 c_1} \right)^{\frac{1}{4}}} \]
(το κρίσιμο σημείο της $f$).

Ελέγχουμε την δεύτερη παράγωγο:
\[ f''(x) = 6 c_1 x + \frac{2 c_2}{x^3} > 0, \quad \forall x > 0 \]
(αφού $c_1, c_2 > 0$ και $x > 0$).

$\nabla^2 f > 0 \implies H \ f$ είναι γνήσια κυρτή $\forall x > 0$.
Από το Θεώρημα 3.1.4 (ή Συμπέρασμα 3.1.1), επειδή η $f$ είναι γνήσια κυρτή, το κρίσιμο σημείο $x^*$ είναι μοναδικό ολικό ελάχιστο.
\end{answer}
\newpage
%----------------------------------------------------------------------
%	SECTION 9: Άσκηση 3.6.14
%----------------------------------------------------------------------
\phantomsection
\hypertarget{exercises9}{}
\pdfbookmark[1]{Άσκηση 3.6.14}{exercises9}
\examyear{Άσκηση 3.6.14}

\begin{center}
\textbf{\Large Άσκηση 3.6.14 -- Σελίδα 73}
\end{center}

\begin{question}[Αυστηρή Κυρτότητα]
Έστω $f(x)$ μια αυστηρά κυρτή συνάρτηση στο $\mathbb{R}^n$. Αν $x, y$ είναι διακριτά σημεία στο $\mathbb{R}^n$ τέτοια ώστε $f(x)=f(y)=0$, δείξτε ότι υπάρχει $z \in \mathbb{R}^n$ τέτοιο ώστε $f(z) < 0$.
\end{question}

\begin{answer}
\textbf{Λύση:}

Θα εφαρμόσω τον ορισμό της κυρτότητας.
Επειδή η $f$ είναι γνήσια (αυστηρά) κυρτή, έχουμε από τον ορισμό της κυρτότητας ότι για $x \neq y$ και $\lambda \in (0,1)$:
\[ f(\lambda x + (1-\lambda)y) < \lambda f(x) + (1-\lambda)f(y) \quad (1) \]

Όμως $f(x) = f(y) = 0$. Οπότε η (1) γίνεται:
\[ f(\lambda x + (1-\lambda)y) < \lambda \cdot 0 + (1-\lambda) \cdot 0 = 0 \]
\[ \implies \boxed{f(\lambda x + (1-\lambda)y) < 0} \]

Άρα, αν επιλέξουμε $z = \lambda x + (1-\lambda)y$ για κάποιο $\lambda \in (0,1)$, τότε $f(z) < 0$.
(Το $z$ είναι ένα σημείο στο ευθύγραμμο τμήμα που ενώνει τα $x$ και $y$).
\end{answer}
\newpage
%----------------------------------------------------------------------
%	SECTION 10: Άσκηση 3.6.21
%----------------------------------------------------------------------
\phantomsection
\hypertarget{exercises10}{}
\pdfbookmark[1]{Άσκηση 3.6.21}{exercises10}
\examyear{Άσκηση 3.6.21}

\begin{center}
\textbf{\Large Άσκηση 3.6.21 -- Σελίδα 74}
\end{center}

\begin{question}[Συνθήκες KKT]
Εφαρμόστε το Θεώρημα των Karush-Kuhn-Tucker (KKT) για να προσδιορίσετε όλες τις λύσεις του προβλήματος:

Βρείτε τα $x_1, x_2$ που ελαχιστοποιούν την:
\[ f(x_1, x_2) = e^{-(x_1 + x_2)} \]
και ικανοποιούν:
\[ e^{x_1} + e^{x_2} \leq 20 \quad \text{και} \quad x_1 \geq 0 \]

Είναι το πρόβλημα κυρτό;
\end{question}

\begin{answer}
\textbf{Λύση:}

Ουσιαστικά προσπαθούμε να βρούμε τα $x_1, x_2$ που ελαχιστοποιούν την αντικειμενική συνάρτηση $f(x)$ και ταυτόχρονα ικανοποιούν τους περιορισμούς:
\[ e^{x_1} + e^{x_2} \leq 20, \quad x_1 \geq 0 \]

1) Ας ονοματίσουμε λίγο πρώτα κάποια πράγματα:

Για τους περιορισμούς:
Θεωρώ $g_1(x_1, x_2) = e^{x_1} + e^{x_2} - 20$ (για να είναι $\leq 0$)
και $g_2(x_1, x_2) = -x_1$ (για να είναι $\leq 0$).

Για την αντικειμενική συνάρτηση έχουμε:
\[ f(x_1, x_2) = e^{-(x_1 + x_2)} \]

2) Είναι το πρόβλημα κυρτό;

Θυμόμαστε ότι ένα πρόβλημα ονομάζεται κυρτό, όταν η αντικειμενική συνάρτηση και οι περιορισμοί είναι κυρτές συναρτήσεις.

Για την $f(x_1, x_2)$:
\[ \nabla f(x) = \begin{bmatrix} \frac{\partial f}{\partial x_1} \\[6pt] \frac{\partial f}{\partial x_2} \end{bmatrix} = \begin{bmatrix} -e^{-(x_1 + x_2)} \\[6pt] -e^{-(x_1 + x_2)} \end{bmatrix} \]
\[ \nabla^2 f(x) = \begin{bmatrix} e^{-(x_1 + x_2)} & e^{-(x_1 + x_2)} \\ e^{-(x_1 + x_2)} & e^{-(x_1 + x_2)} \end{bmatrix} \]
Ο $\nabla^2 f(x)$ είναι θετικά ημιορισμένος (Ορίζουσα = 0, Ίχνος > 0). Έτσι, η $f(x)$ είναι κυρτή.

Για την $g_1(x)$:
\begin{itemize}
    \item $\nabla g_1(x) = \begin{bmatrix} e^{x_1} \\ e^{x_2} \end{bmatrix}$
    \item $\nabla^2 g_1(x) = \begin{bmatrix} e^{x_1} & 0 \\ 0 & e^{x_2} \end{bmatrix}$
\end{itemize}
Ο $\nabla^2 g_1(x)$ είναι θετικά ορισμένος (διαγώνιοι > 0). Άρα, η $g_1(x)$ είναι γνήσια κυρτή συνάρτηση.

Η $g_2(x) = -x_1$ είναι γραμμική, άρα και κυρτή.
Θα εφαρμόσουμε τώρα το Θεώρημα των KKT conditions...

\textbf{KKT conditions:}

Επειδή έχουμε δύο περιορισμούς θα έχουμε και δύο συντελεστές $\lambda_1, \lambda_2 \geq 0$.
Δηλαδή θα έχουμε ότι:
\begin{itemize}
    \item $\lambda_1, \lambda_2 \geq 0$
    \item $\lambda_1 g_1(x) = 0 \implies \lambda_1 (e^{x_1} + e^{x_2} - 20) = 0 \quad (1)$
    \item $\lambda_2 g_2(x) = 0 \implies \lambda_2 (-x_1) = 0 \quad (2)$
    \item $\nabla f(x) + \lambda_1 \nabla g_1(x) + \lambda_2 \nabla g_2(x) = 0$
\end{itemize}

Γράφοντας ξανά αυτές τις σχέσεις (συνοψίζοντας):
\[ \begin{bmatrix} -e^{-(x_1+x_2)} \\ -e^{-(x_1+x_2)} \end{bmatrix} + \lambda_1 \begin{bmatrix} e^{x_1} \\ e^{x_2} \end{bmatrix} + \lambda_2 \begin{bmatrix} -1 \\ 0 \end{bmatrix} = \begin{bmatrix} 0 \\ 0 \end{bmatrix} \]

Έχουμε δημιουργήσει το εξής σύστημα εξισώσεων:
\begin{enumerate}
    \item $\lambda_1, \lambda_2 \geq 0$
    \item $\lambda_1 (e^{x_1} + e^{x_2} - 20) = 0$
    \item $-\lambda_2 x_1 = 0$
    \item $-e^{-(x_1+x_2)} + \lambda_1 e^{x_1} - \lambda_2 = 0 \quad (3)$
    \item $-e^{-(x_1+x_2)} + \lambda_1 e^{x_2} = 0 \quad (4)$
\end{enumerate}
Αν υπάρχουν λοιπόν τέτοια $\lambda_i$ και $x_i$, ώστε να ικανοποιούν το παραπάνω σύστημα, τότε αυτά τα $\lambda_i$ και $x_i$ αποτελούν την λύση του προβλήματος...

Οπότε, θα εξετάσουμε το σύστημα ανά περιπτώσεις:

\textbf{case i) $\lambda_1 = \lambda_2 = 0$:}
Αν $\lambda_1 = \lambda_2 = 0 \xrightarrow{(3)} -e^{-(x_1+x_2)} = 0$ (άτοπο, γιατί $e^z > 0$).

\textbf{case ii) $\lambda_1 = 0$ και $\lambda_2 \neq 0$:}
Αν $\lambda_1 = 0$ και $\lambda_2 \neq 0 \xrightarrow{(4)} -e^{-(x_1+x_2)} = 0$ (άτοπο).

\textbf{case iii) $\lambda_1 \neq 0$ και $\lambda_2 = 0$:}

Για $\lambda_2 = 0$ από την (3) και (4) προκύπτει ότι:
\[ -e^{-(x_1+x_2)} + \lambda_1 e^{x_1} = 0 \]
\[ -e^{-(x_1+x_2)} + \lambda_1 e^{x_2} = 0 \]
$\implies e^{x_1} = e^{x_2} \implies \boxed{x_1 = x_2}$

Για $x_1 = x_2$ η (1) γίνεται:
\[ \lambda_1 (2e^{x_1} - 20) = 0 \xrightarrow{\lambda_1 \neq 0} 2e^{x_1} - 20 = 0 \implies e^{x_1} = 10 \implies \boxed{x_1 = \ln 10} \]
Όμως $x_1 = x_2$, άρα $x_2 = \ln 10$.

Για την τιμή του $\lambda_1$ έχουμε:
\[ -e^{-(x_1+x_2)} + \lambda_1 e^{x_2} = 0 \implies -e^{-2\ln 10} + \lambda_1 e^{\ln 10} = 0 \implies -e^{\ln(10^{-2})} + \lambda_1 \cdot 10 = 0 \]
\[ \implies -0.01 + 10\lambda_1 = 0 \implies \lambda_1 = 0.001 > 0 \quad (\text{δεκτή}) \]

\textbf{case iv) $\lambda_1 \neq 0$ και $\lambda_2 \neq 0$:}
Αν $\lambda_1 \neq 0, \lambda_2 \neq 0$, τότε από την (1) έχουμε ότι $e^{x_1} + e^{x_2} - 20 = 0$ και από την (2) έχουμε ότι $x_1 = 0$.
Από την (1) προκύπτει: $e^0 + e^{x_2} = 20 \implies 1 + e^{x_2} = 20 \implies \boxed{x_2 = \ln 19}$

Αντικαθιστώντας τα $x_1, x_2$ στην (4) έχουμε:
\[ -e^{-(0 + \ln 19)} + \lambda_1 e^{\ln 19} = 0 \implies -\frac{1}{19} + 19\lambda_1 = 0 \implies \lambda_1 = \frac{1}{19^2} > 0 \quad (\text{δεκτή}) \]
Αντικαθιστώντας τα $x_1, x_2, \lambda_1$ στην (3) έχουμε:
\[ -e^{-(0 + \ln 19)} + \lambda_1 e^0 - \lambda_2 = 0 \implies -\frac{1}{19} + \frac{1}{19^2} - \lambda_2 = 0 \implies \lambda_2 = \frac{1 - 19}{19^2} = -\frac{18}{19^2} < 0 \quad (\text{απορρίπτεται}) \]

Συνεπώς η μόνη λύση είναι η $x_1^* = x_2^* = \ln 10$.
\end{answer}

\newpage
%----------------------------------------------------------------------
%	SECTION 11: Άσκηση 3.6.17
%----------------------------------------------------------------------
\phantomsection
\hypertarget{exercises11}{}
\pdfbookmark[1]{Άσκηση 3.6.17}{exercises11}
\examyear{Άσκηση 3.6.17}

\begin{center}
\textbf{\Large Άσκηση 3.6.17 -- Σελίδα 73}
\end{center}

\begin{question}[Ελαχιστοποίηση $\sum \frac{c_j}{x_j}$]
Για το πρόβλημα:
Βρείτε το $x \in \mathbb{R}^n$ που ελαχιστοποιεί την:
\[ f(x) = \sum_{j=1}^n \frac{c_j}{x_j} \]
και ικανοποιεί:
\[ \sum_{j=1}^n a_j x_j - b = 0 \]
$x_j \geq 0, j=1,\dots,n$ με $a_j > 0, c_j > 0, b > 0$.

α) Να γραφούν οι συνθήκες Karush-Kuhn-Tucker.
β) Να βρεθεί το σημείο $x^*$ που τις ικανοποιεί.
\end{question}

\begin{answer}
\textbf{Λύση:}

Θεωρώ: $h(x) = \sum_{j=1}^n a_j x_j - b$
και $g_j(x) = -x_j, j=1,\dots,n$.

\begin{itemize}
    \item $\nabla f(x) = \begin{bmatrix} -\frac{c_1}{x_1^2} \\ \vdots \\ -\frac{c_n}{x_n^2} \end{bmatrix}$
    \item $\nabla h(x) = \begin{bmatrix} a_1 \\ \vdots \\ a_n \end{bmatrix}, \nabla g_1(x) = \begin{bmatrix} -1 \\ 0 \\ \vdots \\ 0 \end{bmatrix}, \dots, \nabla g_n(x) = \begin{bmatrix} 0 \\ \vdots \\ 0 \\ -1 \end{bmatrix}$
\end{itemize}

\textbf{KKT conditions:}
\begin{enumerate}
    \item $\lambda_j \geq 0$ για $j=1,\dots,n$
    \item $\lambda_j \cdot g_j(x) = 0 \implies \lambda_j (-x_j) = 0$
    \item $\nabla f(x) + \mu \nabla h(x) + \sum_{j=1}^n \lambda_j \nabla g_j(x) = 0$
\end{enumerate}
Κάνοντας αντικαταστάσεις στις παραπάνω σχέσεις έχουμε ότι:
\begin{itemize}
    \item $\lambda_j \geq 0, j=1,\dots,n \quad (1)$
    \item $-\lambda_j x_j = 0, j=1,\dots,n \quad (2)$
    \item $\begin{bmatrix} -\frac{c_1}{x_1^2} \\ \vdots \\ -\frac{c_n}{x_n^2} \end{bmatrix} + \mu \begin{bmatrix} a_1 \\ \vdots \\ a_n \end{bmatrix} + \begin{bmatrix} -\lambda_1 \\ \vdots \\ -\lambda_n \end{bmatrix} = \begin{bmatrix} 0 \\ \vdots \\ 0 \end{bmatrix} \quad (3)$
\end{itemize}

Από την (2) και το γεγονός ότι η λύση μας είναι κάποια $x_j \neq 0, j=1,\dots,n$, καταλήγουμε στο συμπέρασμα ότι:
\[ \boxed{\lambda_j = 0, \quad j=1,\dots,n} \quad (4) \]

Αντικαθιστώντας το (4) στην σχέση (3) θα καταλήξουμε στα εξής:
\[ -\frac{c_j}{x_j^2} + \mu a_j = 0 \implies x_j^2 = \frac{c_j}{\mu a_j} \implies x_j = \sqrt{\frac{c_j}{\mu a_j}} \]
Δηλαδή:
\[ x_1 = \sqrt{\frac{c_1}{\mu a_1}}, \dots, x_n = \sqrt{\frac{c_n}{\mu a_n}} \]

\textbf{Πώς βρίσκουμε το $\mu$;}
Παίρνοντας τον ισοτικό περιορισμό:
\[ \sum_{j=1}^n a_j x_j - b = 0 \implies \sum_{j=1}^n a_j \sqrt{\frac{c_j}{\mu a_j}} = b \]
\[ \implies \frac{1}{\sqrt{\mu}} \sum_{j=1}^n \sqrt{a_j c_j} = b \implies \sqrt{\mu} = \frac{\sum_{j=1}^n \sqrt{a_j c_j}}{b} \]
\[ \implies \boxed{\mu = \left( \frac{\sum_{j=1}^n \sqrt{a_j c_j}}{b} \right)^2} \]

Αντικαθιστώντας το $\mu$ στο $x_j$, βρίσκουμε το σημείο $x^*$ που ικανοποιεί τις συνθήκες KKT:
\[ x_j^* = \frac{\sqrt{c_j / a_j}}{\sqrt{\mu}} \implies \boxed{x_j^* = \frac{b \sqrt{c_j / a_j}}{\sum_{k=1}^n \sqrt{a_k c_k}}} \]
\end{answer}
\newpage
%----------------------------------------------------------------------
%	SECTION 12: Πρόβλημα Εφοπλιστικής Εταιρείας
%----------------------------------------------------------------------
\phantomsection
\hypertarget{shipping_prob}{}
\pdfbookmark[1]{Πρόβλημα Εφοπλιστικής Εταιρείας}{shipping_prob}
\examyear{Θέμα Βελτιστοποίησης}

\begin{center}
\textbf{\Large Πρόβλημα Εφοπλιστικής Εταιρείας}
\end{center}

\begin{question}[Ελαχιστοποίηση Κόστους Μεταφοράς]
Μια εφοπλιστική εταιρεία διαθέτει πλοία δύο τύπων $\Pi_1, \Pi_2$ και θέλει να μεταφέρει $254$ τόνους εμπορεύματος.
\begin{itemize}
    \item Το $\Pi_1$ μπορεί να μεταφέρει $22$ τόνους ανά ταξίδι.
    \item Το $\Pi_2$ μπορεί να μεταφέρει $12$ τόνους ανά ταξίδι.
\end{itemize}

Για κάθε ταξίδι:
\begin{itemize}
    \item Το $\Pi_1$ κοστίζει $12.000$ και καίει $4.000$ λίτρα καύσιμα.
    \item Το $\Pi_2$ κοστίζει $10.000$ και καίει $900$ λίτρα καύσιμα.
\end{itemize}

Η εταιρεία διαθέτει $30.800$ λίτρα καύσιμα.
Ποιο είναι το ελάχιστο κόστος μεταφοράς ώστε τα πλοία να ταξιδεύουν με γεμάτα αμπάρια (δηλαδή με γεμάτες αποθήκες);
\end{question}

\begin{answer}
\textbf{Λύση:}

Πρέπει να βρούμε προφανώς την αντικειμενική συνάρτηση και τους περιορισμούς του προβλήματος...

\textbf{Κόστος μεταφοράς:}
\[ f(x,y) = 12.000x + 10.000y \]
(Συνολικό Κόστος $\to$ αυτή είναι η αντικειμενική συνάρτηση).

Τώρα θα βρούμε τους περιορισμούς...
Για να ταξιδεύουν τα πλοία με γεμάτες αποθήκες θα πρέπει:
\[ 22x + 12y = 254 \quad (1^o \text{ περιορισμός}) \]

Για τα καύσιμα έχουμε ότι:
\[ 4.000x + 900y \leq 30.800 \]
\begin{itemize}
    \item Θεωρώ λοιπόν: $h(x,y) = 22x + 12y - 254$
    \item $g_1(x,y) = 4.000x + 900y - 30.800$
\end{itemize}

Προφανώς τα $x,y$ δεν μπορούν να πάρουν αρνητικές τιμές! Έτσι, θα έχουμε δύο ακόμη περιορισμούς:
\begin{itemize}
    \item $g_2(x,y) = -x$
    \item $g_3(x,y) = -y$
\end{itemize}

\textbf{Κλίσεις (Gradients):}
\begin{itemize}
    \item $\nabla f(x,y) = \begin{bmatrix} 12.000 \\ 10.000 \end{bmatrix}$
    \item $\nabla h(x,y) = \begin{bmatrix} 22 \\ 12 \end{bmatrix}$
    \item $\nabla g_1(x,y) = \begin{bmatrix} 4.000 \\ 900 \end{bmatrix}$
    \item $\nabla g_2(x,y) = \begin{bmatrix} -1 \\ 0 \end{bmatrix}, \nabla g_3(x,y) = \begin{bmatrix} 0 \\ -1 \end{bmatrix}$
\end{itemize}

\textbf{KKT Conditions:}
\begin{enumerate}
    \item $\lambda_1, \lambda_2, \lambda_3 \geq 0$
    \item $\lambda_1 (4.000x + 900y - 30.800) = 0 \quad (1)$
    \item $-\lambda_2 x = 0 \quad (2)$
    \item $-\lambda_3 y = 0 \quad (3)$
    \item $12.000 + 22\mu + 4.000\lambda_1 - \lambda_2 = 0 \quad (4)$
    \item $10.000 + 12\mu + 900\lambda_1 - \lambda_3 = 0 \quad (5)$
\end{enumerate}
\textbf{Εξέταση περιπτώσεων:}

\textbf{case i: $x=0$:}
Για $x=0$ το $y$ από την σχέση $22x+12y=254$ βγαίνει ίσο με: $y = 21,16$ (Απορρίπτεται, γιατί τα ταξίδια πρέπει να είναι ακέραιος αριθμός και επίσης για να είναι γεμάτα τα αμπάρια).

\textbf{case ii: $y=0$:}
Για $y=0$ το $x$ από την σχέση $22x+12y=254$ βγαίνει ίσο με: $x = 11,54$ (Απορρίπτεται).

\textbf{case iii: $x, y > 0$:}
Για να ικανοποιούνται οι (2), (3) όταν $x, y > 0$ έχουμε ότι:
\[ \lambda_2 = 0 \quad \text{και} \quad \lambda_3 = 0 \]

Εφόσον έχω ότι $\lambda_1 \neq 0$, από την (1) προκύπτει ότι:
\[ 4.000x + 900y - 30.800 = 0 \quad (1) \]
Όμως, ισχύει επίσης ότι: $22x + 12y - 254 = 0 \quad (6)$.

Από τις (1) και (6) βρίσκω ότι: $\boxed{x=5 \text{ και } y=12}$.

Αντικαθιστώντας τώρα τα $\lambda_2, \lambda_3$ στις (4), (5) έχουμε ότι:
\begin{itemize}
    \item $12.000 + 22\mu + 4.000\lambda_1 = 0 \quad (7)$
    \item $10.000 + 12\mu + 900\lambda_1 = 0 \quad (8)$
\end{itemize}

Λύνοντας το σύστημα των εξισώσεων (7), (8) βρίσκουμε ότι:
\[ \boxed{\mu = -1035,5 \quad \text{και} \quad \lambda_1 = 2,7} \]
\end{answer}
\newpage
%----------------------------------------------------------------------
%	SECTION 13: Άσκηση 3.6.22
%----------------------------------------------------------------------
\phantomsection
\hypertarget{exercises13}{}
\pdfbookmark[1]{Άσκηση 3.6.22}{exercises13}
\examyear{Άσκηση 3.6.22}

\begin{center}
\textbf{\Large Άσκηση 3.6.22 -- Σελίδα 75}
\end{center}

\begin{question}[Επαναλήψτε την Άσκηση 3.6.21]
Επαναλάβετε την Άσκηση 3.6.21 για το πρόβλημα:
Βρείτε τα $x_1, x_2$ που ελαχιστοποιούν την:
\[ f(x_1, x_2) = x_1^2 + x_2^2 - 4x_1 - 4x_2 \]
και ικανοποιούν:
\[ x_1^2 - x_2 \leq 0 \]
\[ x_1 + x_2 \leq 2 \]

Είναι το πρόβλημα κυρτό;
\end{question}

\begin{answer}
\textbf{Λύση:}

1) Ορίζω ως: $g_1(x_1, x_2) = x_1^2 - x_2 \leq 0$ και $g_2(x_1, x_2) = x_1 + x_2 - 2 \leq 0$.

2) Υπολογίζω τις κλίσεις των $f(x_1, x_2), g_1(x_1, x_2)$ και $g_2(x_1, x_2)$.
\[ \nabla f(x_1, x_2) = \begin{bmatrix} 2x_1 - 4 \\ 2x_2 - 4 \end{bmatrix}, \quad \nabla g_1(x_1, x_2) = \begin{bmatrix} 2x_1 \\ -1 \end{bmatrix}, \quad \nabla g_2(x_1, x_2) = \begin{bmatrix} 1 \\ 1 \end{bmatrix} \]

3) Υπολογίζω τους Εσσιανούς πίνακες:
\[ \nabla^2 f = \begin{bmatrix} 2 & 0 \\ 0 & 2 \end{bmatrix} \implies D_1=2>0, D_2=4>0 \implies \text{Θετικά ορισμένος} \implies f \text{ γνήσια κυρτή.} \]
\[ \nabla^2 g_1 = \begin{bmatrix} 2 & 0 \\ 0 & 0 \end{bmatrix} \implies D_1=2>0, D_2=0 \implies \text{Θετικά ημιορισμένος} \implies g_1 \text{ κυρτή.} \]
\[ \nabla^2 g_2 = \begin{bmatrix} 0 & 0 \\ 0 & 0 \end{bmatrix} \implies \text{Θετικά ημιορισμένος} \implies g_2 \text{ κυρτή.} \]
Άρα, το πρόβλημα είναι κυρτό.

4) \textbf{KKT conditions:}
\begin{itemize}
    \item $\lambda_1, \lambda_2 \geq 0$
    \item $\lambda_1 (x_1^2 - x_2) = 0 \quad (1)$
    \item $\lambda_2 (x_1 + x_2 - 2) = 0 \quad (2)$
    \item $\begin{bmatrix} 2x_1 - 4 \\ 2x_2 - 4 \end{bmatrix} + \lambda_1 \begin{bmatrix} 2x_1 \\ -1 \end{bmatrix} + \lambda_2 \begin{bmatrix} 1 \\ 1 \end{bmatrix} = \begin{bmatrix} 0 \\ 0 \end{bmatrix}$
\end{itemize}
$\implies 2x_1 - 4 + 2\lambda_1 x_1 + \lambda_2 = 0 \quad (3)$
$\implies 2x_2 - 4 - \lambda_1 + \lambda_2 = 0 \quad (4)$

\textbf{case i) $\lambda_1 = \lambda_2 = 0$:}
Από (3) $\to 2x_1 = 4 \to x_1 = 2$. Από (4) $\to 2x_2 = 4 \to x_2 = 2$.
Ελέγχω τους περιορισμούς: $g_1(2,2) = 2^2 - 2 = 2 \not\leq 0$ (Απορρίπτεται).

\textbf{case ii) $\lambda_1 = 0, \lambda_2 \neq 0$:}
Από (2) $\to x_1 + x_2 = 2 \to x_1 = 2 - x_2$.
Από (3), (4) με $\lambda_1=0 \to 2x_1 - 4 + \lambda_2 = 0$ και $2x_2 - 4 + \lambda_2 = 0 \implies x_1 = x_2$.
Άρα $2x_1 = 2 \to x_1 = 1, x_2 = 1$.
$\lambda_2 = 4 - 2(1) = 2 > 0$ (Δεκτή).
$g_1(1,1) = 1^2 - 1 = 0 \leq 0$ (Ικανοποιείται).
Τελικά $\boxed{x_1^*=1, x_2^*=1}$.

\textbf{case iii) $\lambda_1 \neq 0, \lambda_2 = 0$:}
Από (1) $\to x_2 = x_1^2$.
Από (3) $\to 2x_1 - 4 + 2\lambda_1 x_1 = 0 \to \lambda_1 = \frac{4 - 2x_1}{2x_1} = \frac{2 - x_1}{x_1}$.
Από (4) $\to 2x_2 - 4 - \lambda_1 = 0 \to 2x_1^2 - 4 - \frac{2 - x_1}{x_1} = 0 \to 2x_1^3 - 4x_1 - 2 + x_1 = 0 \to 2x_1^3 - 3x_1 - 2 = 0$.
Λύνοντας προσεγγιστικά $x_1 \approx 1.47, x_2 \approx 2.17$.
$g_2(1.47, 2.17) = 1.47 + 2.17 = 3.64 \not\leq 2$ (Απορρίπτεται).

\textbf{case iv) $\lambda_1 \neq 0, \lambda_2 \neq 0$:}
$x_2 = x_1^2$ και $x_1 + x_2 = 2 \to x_1^2 + x_1 - 2 = 0 \to (x_1+2)(x_1-1)=0$.
Για $x_1=1, x_2=1 \to$ βρίσκουμε $\lambda_1=0$, άτοπο.
Για $x_1=-2, x_2=4 \to \lambda_1 < 0$, άτοπο.
\end{answer}

\newpage
%----------------------------------------------------------------------
%	SECTION 14: Άσκηση 3.6.24
%----------------------------------------------------------------------
\phantomsection
\hypertarget{exercises14}{}
\pdfbookmark[1]{Άσκηση 3.6.24}{exercises14}
\examyear{Άσκηση 3.6.24}

\begin{center}
\textbf{\Large Άσκηση 3.6.24 -- Σελίδα 75}
\end{center}

\begin{question}[Κατανομή Φορτίου Γεννητριών]
Δύο ηλεκτρικές γεννήτριες είναι συνδεδεμένες παράλληλα για να ικανοποιούν ένα φορτίο ισχύος $60 \text{ W}$. Το κόστος παραγωγής κάθε γεννήτριας είναι συνάρτηση της ισχύος εξόδου $P_i, i=1,2$ ως εξής:
\begin{itemize}
    \item $1^\eta$ γεννήτρια: $c_1 = 1 - P_1 + P_1^2 \text{ Ευρώ/W}$
    \item $2^\eta$ γεννήτρια: $c_2 = 1 + 0,6 P_2 + P_2^2 \text{ Ευρώ/W}$
\end{itemize}
α) Να καταστρώσετε το πρόβλημα ελαχιστοποίησης κόστους.
β) Να λύσετε το πρόβλημα αναλυτικά με συνθήκες KKT.
\end{question}

\begin{answer}
\textbf{Λύση:}

Ορίζουμε την αντικειμενική συνάρτηση κόστους:
\[ f(P_1, P_2) = (1 - P_1 + P_1^2) + (1 + 0,6 P_2 + P_2^2) = P_1^2 + P_2^2 - P_1 + 0,6 P_2 + 2 \]

\textbf{Περιορισμοί:}
\begin{itemize}
    \item $h(P_1, P_2) = P_1 + P_2 - 60 = 0$
    \item $g_1(P_1, P_2) = -P_1 \leq 0$
    \item $g_2(P_1, P_2) = -P_2 \leq 0$
\end{itemize}

\textbf{Κυρτότητα:}
$\nabla^2 f = \begin{bmatrix} 2 & 0 \\ 0 & 2 \end{bmatrix} \to$ Θετικά ορισμένος (γνήσια κυρτή).
Οι περιορισμοί είναι γραμμικοί $\to$ κυρτοί. Άρα το πρόβλημα είναι κυρτό.

\textbf{KKT Conditions:}
\begin{enumerate}
    \item $\lambda_1, \lambda_2 \geq 0$
    \item $-\lambda_1 P_1 = 0 \quad (1)$
    \item $-\lambda_2 P_2 = 0 \quad (2)$
    \item $2P_1 - 1 + \mu - \lambda_1 = 0 \quad (3)$
    \item $2P_2 + 0,6 + \mu - \lambda_2 = 0 \quad (4)$
\end{enumerate}

\textbf{case 4) $P_1 \neq 0, P_2 \neq 0$:}
$\lambda_1 = \lambda_2 = 0$.
Από (3), (4) $\to 2P_1 - 1 + \mu = 0$ και $2P_2 + 0,6 + \mu = 0 \implies 2P_1 - 1 = 2P_2 + 0,6 \to P_1 - P_2 = 0,8$.
Επίσης $P_1 + P_2 = 60$.
Λύνοντας: $2P_1 = 60,8 \to \boxed{P_1^* = 30,4, P_2^* = 29,6}$.
$\mu = 1 - 2(30,4) = -59,8$.
Η λύση είναι δεκτή.
\end{answer}

\newpage
%----------------------------------------------------------------------
%	SECTION 15: Κεφάλαιο 5 - Θεωρητική Εισαγωγή
%----------------------------------------------------------------------
\phantomsection
\hypertarget{theory5}{}
\pdfbookmark[1]{Κεφάλαιο 5}{theory5}
\examyear{Θεωρία}

\begin{center}
\textbf{\Large Κεφάλαιο 5 -- Θεωρητική Εισαγωγή}
\end{center}

\begin{tcolorbox}[colback=partbg, colframe=blue!30!black, title=Θεώρημα 5.1.1 (Strictly quasi-convex)]
Έστω $f(x)$ μια αυστηρά σχεδόν-κυρτή συνάρτηση στο διάστημα $[a, b]$. Έστω δύο σημεία $x_1, x_2 \in [a, b]$ με $x_1 < x_2$.
\begin{enumerate}
    \item Αν $f(x_1) < f(x_2)$, τότε $f(x) \geq f(x_1), \forall x \in (x_2, b]$. (Αποκλείουμε το $(x_2, b]$).
    \item Αν $f(x_1) \geq f(x_2)$, τότε $f(x) \geq f(x_2), \forall x \in [a, x_1)$. (Αποκλείουμε το $[a, x_1)$).
\end{enumerate}
\end{tcolorbox}

\begin{tcolorbox}[colback=white, colframe=black, title=Ορισμός αυστηρά σχεδόν-κυρτής]
Μια συνάρτηση $f$ λέγεται αυστηρά σχεδόν-κυρτή όταν:
\[ f(\lambda x_1 + (1-\lambda) x_2) < \max\{f(x_1), f(x_2)\}, \quad \forall x_1, x_2 \in S, \lambda \in (0,1) \]
Διαισθητικά, μια τέτοια συνάρτηση είναι μονότονα φθίνουσα έως ένα σημείο και μετά μονότονα αύξουσα (unimodal).
\end{tcolorbox}

\begin{tcolorbox}[colback=white, colframe=black, title=Ερμηνεία με Compact Sets]
Μια συνάρτηση είναι αυστηρά σχεδόν κυρτή αν για οποιαδήποτε οριζόντια γραμμή, το σύνολο των σημείων κάτω από τη γραμμή ($S_\alpha = \{x : f(x) \leq \alpha\}$) είναι κυρτό (στο $\mathbb{R}$ αυτό σημαίνει διάστημα, δηλαδή compact set αν η συνάρτηση "ανεβαίνει" και στις δύο πλευρές).
\end{tcolorbox}

\newpage
%----------------------------------------------------------------------
%	SECTION 16: Αλγόριθμοι Αναζήτησης Ελαχίστου
%----------------------------------------------------------------------
\phantomsection
\hypertarget{algorithms}{}
\pdfbookmark[1]{Αλγόριθμοι}{algorithms}
\examyear{Θεωρία}

\begin{center}
\textbf{\Large Αλγόριθμοι Μονοδιάστατης Αναζήτησης}
\end{center}

\begin{tcolorbox}[colback=white, colframe=black, title=1) Μέθοδος της Διχοτόμου (Dichotomous Search)]
Στη μέθοδο αυτή επιλέγουμε μια σταθερά ανοχής $l > 0$ και μια πολύ μικρή τιμή $\varepsilon > 0$. Στόχος είναι να βρούμε ένα τελικό διάστημα $[a_n, b_n]$ εύρους $< l$.

\textbf{Βήματα:}
Σε κάθε επανάληψη $k$, επιλέγουμε δύο σημεία $x_1, x_2$ γύρω από το μέσο του διαστήματος $[a_k, b_k]$:
\[ x_{1,k} = \frac{a_k + b_k}{2} - \varepsilon, \quad x_{2,k} = \frac{a_k + b_k}{2} + \varepsilon \]
\begin{itemize}
    \item Αν $f(x_1) < f(x_2)$, τότε το ελάχιστο βρίσκεται στο $[a_k, x_2]$. (Αποκλείουμε το $(x_2, b_k]$).
    \item Αν $f(x_1) > f(x_2)$, τότε το ελάχιστο βρίσκεται στο $[x_1, b_k]$. (Αποκλείουμε το $[a_k, x_1)$).
\end{itemize}
Η διαδικασία επαναλαμβάνεται μέχρι $b_n - a_n < l$.
\end{tcolorbox}

\begin{tcolorbox}[colback=white, colframe=black, title=2) Μέθοδος του Χρυσού Τομέα (Golden Section Search)]
Η μέθοδος αυτή είναι πιο αποδοτική γιατί σε κάθε επανάληψη απαιτεί μόνο έναν νέο υπολογισμό της συνάρτησης (επαναχρησιμοποιεί το ένα από τα δύο σημεία της προηγούμενης επανάληψης).

Επιλέγουμε τα σημεία $x_1, x_2$ σύμφωνα με τον χρυσό λόγο $\gamma \approx 0,618$:
\[ x_{1,k} = a_k + (1-\gamma)(b_k - a_k) \]
\[ x_{2,k} = a_k + \gamma(b_k - a_k) \]
όπου $\gamma = \frac{\sqrt{5}-1}{2} \approx 0,618$.

\textbf{Ιδιότητα reuse:}
Αν αποκλείσουμε το δεξί κομμάτι, το νέο $x_2$ της επόμενης επανάληψης θα συμπίπτει με το παλιό $x_1$. Έτσι γλυτώνουμε υπολογισμούς.
\end{tcolorbox}

\begin{tcolorbox}[colback=white, colframe=black, title=3) Μέθοδος της Διχοτόμου με χρήση Παραγώγων]
Η μέθοδος αυτή προϋποθέτει ότι γνωρίζουμε την παράγωγο της $f$.
Πηγαίνουμε στο μέσο $x_k = \frac{a_k + b_k}{2}$ και υπολογίζουμε την κλίση $f'(x_k)$.
\begin{enumerate}
    \item Αν $f'(x_k) = 0$, τότε το $x_k$ είναι το ελάχιστο.
    \item Αν $f'(x_k) > 0$, τότε το ελάχιστο βρίσκεται στο $[a_k, x_k]$. (Αποκλείουμε το $(x_k, b_k]$).
    \item Αν $f'(x_k) < 0$, τότε το ελάχιστο βρίσκεται στο $[x_k, b_k]$. (Αποκλείουμε το $[a_k, x_k)$).
\end{enumerate}
Η μέθοδος αυτή είναι υπολογιστικά "βαριά" αν η παράγωγος είναι δύσκολη στον υπολογισμό, αλλά συγκλίνει γρήγορα.
\end{tcolorbox}

\newpage
%----------------------------------------------------------------------
%	SECTION 17: Μέθοδος Παρεμβολής Τετραγωνικών Συναρτήσεων
%----------------------------------------------------------------------
\phantomsection
\hypertarget{quadratic_interp}{}
\pdfbookmark[1]{Τετραγωνική Παρεμβολή}{quadratic_interp}
\examyear{Θεωρία}

\begin{center}
\textbf{\Large Μέθοδος Παρεμβολής Τετραγωνικών Συναρτήσεων}
\end{center}

\begin{tcolorbox}[colback=white, colframe=black, title=Αλγόριθμος: Τετραγωνική Παρεμβολή]
Έστω $f(x)$ μια αυστηρά σχεδόν-κυρτή συνάρτηση στο $[0, \infty)$.
Υποθέτουμε ότι διαθέτουμε τρία σημεία $x_1, x_2, x_3$ τέτοια ώστε:
\[ 0 \leq x_1 < x_2 < x_3 \quad \text{και} \quad f(x_1) \geq f(x_2), \quad f(x_2) \leq f(x_3) \]
(Τα σημεία αυτά σχηματίζουν ένα "σχήμα V" που περικλείει το ελάχιστο).

\textbf{Εκτέλεση:}
1. Παρεμβάλλουμε μια τετραγωνική καμπύλη (παραβολή) που περνάει από τα $(x_1, f(x_1)), (x_2, f(x_2)), (x_3, f(x_3))$.
2. Βρίσκουμε το ελάχιστο $\hat{x}$ της παραβολής.
3. Επιλέγουμε μια νέα τριάδα σημείων:
\begin{itemize}
    \item Αν $\hat{x} > x_2$:
    \begin{itemize}
        \item Αν $f(\hat{x}) \geq f(x_2)$, η νέα τριάδα είναι $\{x_1, x_2, \hat{x}\}$.
        \item Αν $f(\hat{x}) < f(x_2)$, η νέα τριάδα είναι $\{x_2, \hat{x}, x_3\}$.
    \end{itemize}
    \item Αν $\hat{x} < x_2$:
    \begin{itemize}
        \item Αν $f(\hat{x}) \geq f(x_2)$, η νέα τριάδα είναι $\{\hat{x}, x_2, x_3\}$.
        \item Αν $f(\hat{x}) < f(x_2)$, η νέα τριάδα είναι $\{x_1, \hat{x}, x_2\}$.
    \end{itemize}
\end{itemize}
4. Επαναλαμβάνουμε μέχρι το διάστημα $[x_3 - x_1] < l$.
\end{tcolorbox}

\newpage
%----------------------------------------------------------------------
%	SECTION 18: Ανισότητα Minkowski
%----------------------------------------------------------------------
\phantomsection
\hypertarget{minkowski}{}
\pdfbookmark[1]{Ανισότητα Minkowski}{minkowski}
\examyear{Θεωρία}

\begin{center}
\textbf{\Large Ανισότητα Minkowski}
\end{center}

\begin{tcolorbox}[colback=partbg, colframe=blue!30!black, title=Θεώρημα Minkowski]
Έστω $x, y \in \mathbb{R}^n$ και $p \geq 1$. Τότε ισχύει:
\[ \left( \sum_{i=1}^n |x_i + y_i|^p \right)^{1/p} \leq \left( \sum_{i=1}^n |x_i|^p \right)^{1/p} + \left( \sum_{i=1}^n |y_i|^p \right)^{1/p} \]
\end{tcolorbox}

\begin{tcolorbox}[colback=white, colframe=black, title=Απόδειξη]
1. Για $p=1$, η ανισότητα ανάγεται στην τριγωνική ανισότητα $|x_i + y_i| \leq |x_i| + |y_i|$.
2. Για $p > 1$, θεωρούμε τη συνάρτηση $\phi(t) = t^p$ για $t > 0$.
Η δεύτερη παράγωγος είναι $\phi''(t) = p(p-1)t^{p-2} > 0$, άρα η $\phi$ είναι γνήσια κυρτή.
Ορίζουμε:
\[ \|x\|_p = \left( \sum |x_i|^p \right)^{1/p}, \quad \|y\|_p = \left( \sum |y_i|^p \right)^{1/p} \]
Επιλέγουμε $\lambda_1 = \frac{\|x\|_p}{\|x\|_p + \|y\|_p}$ και $\lambda_2 = \frac{\|y\|_p}{\|x\|_p + \|y\|_p}$, ώστε $\lambda_1 + \lambda_2 = 1$.
Εφαρμόζουμε τον ορισμό της κυρτότητας για τα σημεία $a_i = \frac{|x_i|}{\|x\|_p}$ και $b_i = \frac{|y_i|}{\|y\|_p}$:
\[ \phi(\lambda_1 a_i + \lambda_2 b_i) \leq \lambda_1 \phi(a_i) + \lambda_2 \phi(b_i) \]
\[ \left( \frac{\|x\|_p}{\|x\|_p + \|y\|_p} \frac{|x_i|}{\|x\|_p} + \frac{\|y\|_p}{\|x\|_p + \|y\|_p} \frac{|y_i|}{\|y\|_p} \right)^p \leq \lambda_1 \left( \frac{|x_i|}{\|x\|_p} \right)^p + \lambda_2 \left( \frac{|y_i|}{\|y\|_p} \right)^p \]
\[ \left( \frac{|x_i| + |y_i|}{\|x\|_p + \|y\|_p} \right)^p \leq \frac{\|x\|_p}{\|x\|_p + \|y\|_p} \frac{|x_i|^p}{\|x\|_p^p} + \frac{\|y\|_p}{\|x\|_p + \|y\|_p} \frac{|y_i|^p}{\|y\|_p^p} \]
Αθροίζοντας ως προς $i$:
\[ \frac{\sum (|x_i| + |y_i|)^p}{(\|x\|_p + \|y\|_p)^p} \leq \frac{1}{\|x\|_p + \|y\|_p} \left( \frac{\sum |x_i|^p}{\|x\|_p^{p-1}} + \frac{\sum |y_i|^p}{\|y\|_p^{p-1}} \right) \]
Επειδή $\sum |x_i|^p = \|x\|_p^p$:
\[ \frac{\sum (|x_i| + |y_i|)^p}{(\|x\|_p + \|y\|_p)^p} \leq \frac{1}{\|x\|_p + \|y\|_p} (\|x\|_p + \|y\|_p) = 1 \]
Άρα:
\[ \sum |x_i + y_i|^p \leq \sum (|x_i| + |y_i|)^p \leq (\|x\|_p + \|y\|_p)^p \]
Υψώνοντας στην $1/p$:
\[ \|x+y\|_p \leq \|x\|_p + \|y\|_p \]
\end{tcolorbox}

\begin{tcolorbox}[colback=white, colframe=black, title=4) Μέθοδος Fibonacci]
Η μέθοδος αυτή είναι παρόμοια με τη μέθοδο του χρυσού τομέα, αλλά χρησιμοποιεί την ακολουθία Fibonacci $F_n$.
\[ F_0 = 1, \quad F_1 = 1, \quad F_{n+1} = F_n + F_{n-1} \]
Τα σημεία επιλέγονται ως:
\[ x_{1,k} = a_k + \frac{F_{n-k-1}}{F_{n-k+1}}(b_k - a_k), \quad x_{2,k} = a_k + \frac{F_{n-k}}{F_{n-k+1}}(b_k - a_k) \]
Στην τελευταία επανάληψη ($k=n$), τα $x_1, x_2$ συμπίπτουν στο μέσο. Για να γίνει η σύγκριση, επιλέγουμε ένα σημείο $x_2'$ σε απόσταση $\varepsilon$ από το $x_1$.
Ο αριθμός των επαναλήψεων $n$ προκαθορίζεται από την ακρίβεια $l$:
\[ \frac{1}{F_n} (b_1 - a_1) < l \]
\end{tcolorbox}

\newpage
%----------------------------------------------------------------------
%	SECTION 19: Ασκήσεις Κεφαλαίου 5
%----------------------------------------------------------------------
\phantomsection
\hypertarget{ch5_exercises}{}
\pdfbookmark[1]{Ασκήσεις Κεφ. 5}{ch5_exercises}

\begin{question}
\textbf{Άσκηση 5.5.6 (σελ. 172)} \\
Έστω $\{x_k\}$ η ακολουθία που παράγεται από τη μέθοδο της μέγιστης καθόδου με σφάλμα:
\[ x_{k+1} = x_k - \gamma_k (\nabla f(x_k) + e_k), \quad |e_k| \leq \delta, \quad \forall k \]
Δείξτε ότι αν $|\nabla f(x_k)| > \delta$, τότε υπάρχει $\gamma_1 > 0$ τέτοιο ώστε $f(x_{k+1}) < f(x_k)$ για $\gamma_k \in (0, \gamma_1]$.
\end{question}

\begin{answer}
\textbf{Λύση:}

Θεωρούμε τη συνάρτηση $\phi_k(\gamma) = f(x_k - \gamma(\nabla f(x_k) + e_k))$.
Η παράγωγος της $\phi_k$ ως προς $\gamma$ είναι:
\[ \phi_k'(\gamma) = -\nabla f(x_k - \gamma(\nabla f(x_k) + e_k))^T \cdot (\nabla f(x_k) + e_k) \]
Για $\gamma = 0$:
\[ \phi_k'(0) = -\nabla f(x_k)^T \cdot (\nabla f(x_k) + e_k) = -|\nabla f(x_k)|^2 - \nabla f(x_k)^T e_k \]
Χρησιμοποιώντας την ανισότητα Cauchy-Schwarz και το γεγονός ότι $|e_k| \leq \delta$:
\[ \phi_k'(0) \leq -|\nabla f(x_k)|^2 + |\nabla f(x_k)| \cdot |e_k| \leq -|\nabla f(x_k)| \cdot (|\nabla f(x_k)| - \delta) \]
Επειδή από την εκφώνηση $|\nabla f(x_k)| > \delta$, έχουμε:
\[ \phi_k'(0) < 0 \]
Εφόσον η παράγωγος στο $0$ είναι αρνητική, η συνάρτηση $\phi_k(\gamma)$ είναι φθίνουσα για μικρά $\gamma$. Άρα υπάρχει $\gamma_1 > 0$ τέτοιο ώστε για κάθε $\gamma \in (0, \gamma_1]$ να ισχύει:
\[ \phi_k(\gamma) < \phi_k(0) \implies f(x_{k+1}) < f(x_k) \]
\end{answer}

\begin{question}
\textbf{Άσκηση 5.5.8 (σελ. 175)} \\
Θεωρήστε το πρόβλημα ελαχιστοποίησης της $f(x) = x^4 - 1, x \in \mathbb{R}$. Λύστε το κάνοντας χρήση της μεθόδου Newton. Βρείτε τη συνθήκη σύγκλισης και τον ρυθμό σύγκλισης. Επιλέξτε το βήμα $\gamma_k$ για βέλτιστη ταχύτητα.
\end{question}

\begin{answer}
\textbf{Λύση:}

Για την $f(x) = x^4 - 1$:
\[ f'(x) = 4x^3, \quad f''(x) = 12x^2 \]
Η μέθοδος Newton είναι:
\[ x_{k+1} = x_k - \gamma_k [f''(x_k)]^{-1} f'(x_k) \]
\[ x_{k+1} = x_k - \gamma_k \frac{4x_k^3}{12x_k^2} = x_k - \gamma_k \frac{1}{3} x_k = x_k \left(1 - \frac{\gamma_k}{3}\right) \]
Για σύγκλιση πρέπει $|x_{k+1} / x_k| < 1$, άρα:
\[ |1 - \gamma_k/3| < 1 \implies -1 < 1 - \gamma_k/3 < 1 \implies 0 < \gamma_k < 6 \]
Ο ρυθμός σύγκλισης είναι γραμμικός με λόγο $(1 - \gamma_k/3)$.
Για βέλτιστη επιλογή $\gamma_k$, θέλουμε $x_{k+1} = 0$ σε ένα βήμα:
\[ 1 - \gamma_k/3 = 0 \implies \gamma_k = 3 \]
Με $\gamma_k = 3$ η μέθοδος συγκλίνει ακαριαία (σε 1 βήμα).
\end{answer}

\begin{question}
\textbf{Άσκηση 5.6.17 (σελ. 192)} \\
Έστω $f_1(x) = -x_1^2 - x_2^2$ και $f_2(x) = x_1^2 - x_2^3$. Δείξτε ότι ο αλγόριθμος Levenberg-Marquardt με $\gamma_k = 1$ δεν συγκλίνει στο μέγιστο $(0,0)$ της $f_1$ αν $x_0 \neq 0$. Δείξτε επίσης ότι δεν συγκλίνει στο σαμαροειδές σημείο $(0,0)$ της $f_2$ αν $x_2(0) \neq 0$.
\end{question}

\begin{answer}
\textbf{Λύση:}

\textbf{Για την $f_1(x) = -x_1^2 - x_2^2$:}
\[ \nabla f_1 = [-2x_1, -2x_2]^T, \quad \nabla^2 f_1 = \text{diag}(-2, -2) \]
Ο Levenberg-Marquardt προσθέτει $\mu_k I$ ώστε ο πίνακας $(\nabla^2 f + \mu_k I)$ να είναι θετικά ορισμένος.
Επιλέγουμε $\mu_k > 2$, έστω $\mu_k = 3$. Τότε ο πίνακας γίνεται $I$.
Η διεύθυνση είναι $d_k = -[I]^{-1} \nabla f_1 = 2x_k$.
Με $\gamma_k = 1$: $x_{k+1} = x_k + 2x_k = 3x_k$.
Η ακολουθία $x_k = 3^k x_0$ αποκλίνει για $x_0 \neq 0$.

\textbf{Για την $f_2(x) = x_1^2 - x_2^3$:}
\[ \nabla f_2 = [2x_1, -3x_2^2]^T, \quad \nabla^2 f_2 = \text{diag}(2, -6x_2) \]
Ο Εσσιανός στη διεύθυνση $x_2$ μπορεί να είναι αρνητικός. Προσθέτοντας $\mu_k$ ώστε ο πίνακας να είναι θετικά ορισμένος, η κίνηση στον άξονα $x_2$ θα είναι της μορφής $d_{2,k} = \frac{3x_2^2}{\mu_k - 6x_2}$.
Αν $x_{2,0} \neq 0$, η ακολουθία στον άξονα $x_2$ δεν θα μηδενιστεί απαραίτητα (ανάλογα με το $\mu_k$). Σημειώνεται ότι για $x_2 \to 0$, η παράγωγος $-3x_2^2$ μηδενίζεται πιο γρήγορα από τον παρονομαστή, αλλά ο αλγόριθμος Levenberg-Marquardt έχει σχεδιαστεί για ελαχιστοποίηση, ενώ εδώ το $x_2$ "θέλει" να αυξηθεί για να μειώσει την $f_2 = -x_2^3$.
\end{answer}
\newpage
%----------------------------------------------------------------------
%	SECTION 20: Τελευταία 15 Θέματα (Pages 101-115)
%----------------------------------------------------------------------
\phantomsection
\hypertarget{last15}{}
\pdfbookmark[1]{Τελευταία 15 Θέματα}{last15}
\section*{Τελευταία 15 Θέματα (Pages 101-115)}

\begin{question}[Άσκηση 5.5.3 (Σελίδα 170)]
\textbf{Πρόβλημα:} Θεωρείστε τη συνάρτηση $f(x) = \frac{1}{3} |x|^3, x \in \mathbb{R}^n$. Θέλουμε να την ελαχιστοποιήσουμε χρησιμοποιώντας τη μέθοδο της μέγιστης καθόδου με σταθερό βήμα $\gamma_k = \gamma, \forall k$. Βρείτε τη σχέση που πρέπει να ικανοποιούν οι αρχικές τιμές εκκίνησης του αλγορίθμου ώστε η μέθοδος να συγκλίνει.
\end{question}

\begin{answer}
\textbf{Λύση:}
\begin{itemize}
    \item Μέθοδος Μέγιστης Καθόδου: $x_{k+1} = x_k + \gamma_k d_k$, όπου $d_k = -\nabla f(x_k)$.
    \item Υπολογισμός Κλίσης: $\nabla f(x) = \frac{1}{3} \cdot 3 |x|^2 \nabla(|x|) = |x|^2 \frac{x}{|x|} = x|x|$.
    \item Αναδρομή: $x_{k+1} = x_k - \gamma x_k |x_k| = x_k (1 - \gamma |x_k|)$.
    \item Μέτρο: $|x_{k+1}| = |x_k| |1 - \gamma |x_k||$.
\end{itemize}

\textbf{Συνθήκη Σύγκλισης:} $|1 - \gamma |x_k|| < 1$
\begin{itemize}
    \item $-1 < 1 - \gamma |x_k| < 1 \implies -2 < -\gamma |x_k| < 0 \implies 2 > \gamma |x_k| > 0$.
    \item Άρα: $|x_k| < \frac{2}{\gamma}$. (Το $\gamma$ πρέπει να επιλεγεί έτσι ώστε να ισχύει αυτή η σχέση).
\end{itemize}

\textbf{Επιλογή $\gamma_k$ (Optimal Step):}
\begin{itemize}
    \item Ελαχιστοποίηση της $f(x_k - \gamma \nabla f(x_k))$ ως προς $\gamma$.
    \item $\frac{df(x_k - \gamma \nabla f(x_k))}{d\gamma} = \nabla f(x_k - \gamma \nabla f(x_k))^T (-\nabla f(x_k)) = 0$.
    \item Καταλήγει στο: $1 - \gamma_k |x_k| = 0 \implies \gamma_k = \frac{1}{|x_k|}$.
    \item Για αυτή την επιλογή: $x_{k+1} = x_k - \frac{1}{|x_k|} x_k |x_k| = 0$. (Σύγκλιση σε 1 βήμα).
\end{itemize}
\end{answer}

\begin{question}[Άσκηση 5.5.9 (Σελίδα 176)]
\textbf{Πρόβλημα:} Έστω ότι $|x - x^*| = \epsilon$ όπου $x^*$ είναι τοπικό ελάχιστο. Βρείτε φράγματα στις ποσότητες $|f(x) - f(x^*)|$ και $|\nabla f(x) - \nabla f(x^*)|$ συναρτήσει του $\nabla^2 f(x)$.
\end{question}

\begin{answer}
\textbf{Λύση:}
\begin{itemize}
    \item Taylor 2ης τάξης γύρω από το $x^*$: $f(x) = f(x^*) + \nabla f(x^*)^T (x - x^*) + \frac{1}{2} (x - x^*)^T \nabla^2 f(x^*) (x - x^*)$.
    \item Επειδή $x^*$ ελάχιστο: $\nabla f(x^*) = 0$.
    \item $|f(x) - f(x^*)| \leq \frac{1}{2} |x - x^*|^2 |\nabla^2 f(x^*)| = \frac{1}{2} \epsilon^2 |\nabla^2 f(x^*)|$.
    \item Taylor 1ης τάξης για $\nabla f$: $\nabla f(x) = \nabla f(x^*) + \nabla^2 f(x^*) (x-x^*) \implies |\nabla f(x) - \nabla f(x^*)| \leq |\nabla^2 f(x^*)| \epsilon$.
\end{itemize}
\end{answer}

\begin{question}[Άσκηση 5.5.12 (Σελίδα 181)]
\textbf{Πρόβλημα:} Θεωρείστε τη μέθοδο μέγιστης καθόδου $x_{k+1} = x_k - \gamma_k \nabla f(x_k)$. Υποθέτοντας ότι η $f$ είναι κυρτή, δείξτε ότι $\forall y \in \mathbb{R}^n$: $|x_{k+1} - y|^2 \leq |x_k - y|^2 - 2\gamma_k (f(x_k) - f(y)) + (\gamma_k |\nabla f(x_k)|)^2$.

\textbf{Ερώτημα β):} Υποθέστε $\sum_{k=0}^{\infty} \gamma_k = \infty$ και $\gamma_k |\nabla f(x_k)|^2 \to 0$. Δείξτε ότι $\liminf_{k \to \infty} f(x_k) = \inf_{x \in \mathbb{R}^n} f(x)$.
\end{question}

\begin{answer}
\textbf{Λύση:}

\textbf{Απόδειξη (α):}
\begin{itemize}
    \item $|x_{k+1} - y|^2 = |x_k - \gamma_k \nabla f(x_k) - y|^2 = |x_k - y|^2 + (\gamma_k |\nabla f(x_k)|)^2 - 2\gamma_k \nabla f(x_k)^T (x_k - y)$.
    \item Από κυρτότητα (Θ. 3.1.5): $f(x_k) + \nabla f(x_k)^T (y - x_k) \leq f(y) \implies \nabla f(x_k)^T (x_k - y) \geq f(x_k) - f(y)$.
    \item Αντικατάσταση στην 1η σχέση δίνει το ζητούμενο.
\end{itemize}

\textbf{Λύση (β) - Εις άτοπον απαγωγή:}
\begin{itemize}
    \item Έστω $\liminf f(x_k) \neq \inf f(x)$.
    \item Αν $\liminf f(x_k) > \inf f(x)$, τότε $\exists y, \delta > 0$ τέτοια ώστε $f(y) < f(x_k) - \delta, \forall k \geq \bar{k}$.
    \item Από τη σχέση στο (α): $|x_{k+1} - y|^2 \leq |x_k - y|^2 + \gamma_k^2 |\nabla f(x_k)|^2 - 2\gamma_k \delta$.
    \item Άθροιση από $\bar{k}$ έως $\infty$: $0 \leq |x_{\bar{k}} - y|^2 - \delta \sum \gamma_k + \sum (\gamma_k^2 |\nabla f(x_k)|^2 \text{ terms})$.
    \item Επειδή $\sum \gamma_k = \infty$, καταλήγουμε σε $0 \leq -\infty$ (Άτοπο).
    \item Άρα η υπόθεση ήταν λανθασμένη.
\end{itemize}
\end{answer}

\begin{question}[Μέσος Όρος ως Πρόβλημα Βελτιστοποίησης]
\textbf{Πρόβλημα:} Ο μέσος όρος $x^* = \frac{\sum y_i}{m}$ ελαχιστοποιεί τη συνάρτηση $f(x) = \sum_{i=1}^m \frac{1}{2} (y_i - x)^2$. Εφαρμόστε τη μέθοδο μέγιστης κλίσης.
\end{question}

\begin{answer}
\textbf{Λύση:}
\begin{itemize}
    \item $\nabla f(x) = -\sum_{i=1}^m (y_i - x) = m x - \sum y_i$.
    \item $x_{k+1} = x_k - \gamma_k \nabla f(x_k) = x_k - \gamma_k (m x_k - \sum y_i) = (1 - \gamma_k m) x_k + \gamma_k \sum y_i$.
    \item Επιλέγοντας $\gamma_k = \frac{1}{m}$: $x_{k+1} = \frac{1}{m} \sum y_i = x^*$. (Σύγκλιση σε 1 βήμα).
\end{itemize}
\end{answer}

\begin{question}[Μέθοδος Σχεδόν Newton (Quasi-Newton Update)]
\textbf{Πρόβλημα:} $f(x) = \frac{1}{2} x^T Q x - b^T x$. Ο πίνακας $\Delta_k$ ενημερώνεται ως: $\Delta_{k+1} = \Delta_k + \frac{y_k y_k^T}{q_k^T y_k}$, όπου $y_k = \rho_k - \Delta_k q_k$.
\begin{enumerate}
    \item Δείξτε $\Delta_{k+1} q_i = \rho_i, \forall i \leq k$.
    \item Δείξτε $\Delta_n = Q^{-1}$.
\end{enumerate}
\end{question}

\begin{answer}
\textbf{Λύση:}

\textbf{Απόδειξη (Επαγωγή):}
\begin{itemize}
    \item Για $i=k$: $\Delta_{k+1} q_k = \Delta_k q_k + y_k = \Delta_k q_k + \rho_k - \Delta_k q_k = \rho_k$. (Ισχύει).
    \item Για $i < k$: $\Delta_{k+1} q_i = \Delta_k q_i + \frac{y_k (y_k^T q_i)}{q_k^T y_k} = \dots$ (χρήση ιδιοτήτων $Q$-συζυγών κατευθύνσεων).
    \item Τελικά $\Delta_n q_i = \rho_i \implies \Delta_n Q \rho_i = \rho_i \implies \Delta_n = Q^{-1}$.
\end{itemize}
\end{answer}

\begin{question}[Άσκηση 6.7.2 (Σελίδα 230) - Προβολή Κλίσης]
\textbf{Πρόβλημα:} $f(x) = \frac{1}{2} x^2$ στο $X = [-1, 2]$ με σταθερό βήμα $s$.
\end{question}

\begin{answer}
\textbf{Λύση:}
\begin{itemize}
    \item $\bar{x}_k = Pr_X \{ x_k - s \nabla f(x_k) \} = Pr_X \{ x_k (1-s) \}$.
    \item Προβολή:
    \begin{itemize}
        \item $-1$, αν $x_k(1-s) \leq -1$
        \item $2$, αν $x_k(1-s) \geq 2$
        \item $x_k(1-s)$, αν $-1 \leq x_k(1-s) \leq 2$
    \end{itemize}
    \item $x_{k+1} = x_k + \gamma_k (\bar{x}_k - x_k)$. Για $\gamma_k = 1$: $x_{k+1} = \dots$ (δίνει τις 3 περιπτώσεις).
\end{itemize}
\end{answer}

\begin{question}[Άσκηση 6.8.14 (Σελίδα 236)]
\textbf{Πρόβλημα:} $f(x) = \frac{1}{2} x_1^2 x_2$ με $x_1 \in [-1, 2]$ και $x_2 \in \mathbb{R}$.
\end{question}

\begin{answer}
\textbf{Λύση:}
\begin{itemize}
    \item $\nabla f(x) = [x_1 x_2, \frac{1}{2} x_1^2]^T$.
    \item $\bar{x}_k = Pr_X \{ [x_{1,k} - s x_{1,k} x_{2,k}, x_{2,k} - \frac{s}{2} x_{1,k}^2]^T \}$.
    \item Επειδή $x_2$ ελεύθερο: $\bar{x}_{2,k} = x_{2,k} - \frac{s}{2} x_{1,k}^2$.
    \item Για το $x_1$ εφαρμόζεται η προβολή στο $[-1, 2]$.
\end{itemize}
\end{answer}

\part*{Θέματα Εξετάσεων}
\phantomsection
\addcontentsline{toc}{section}{Θέματα Εξετάσεων}

% ============== Ιούνιος 2014 (Πρώην Ιούλιος 2014) ==============
\newpage
\phantomsection
\hypertarget{iounios2014}{}
\pdfbookmark[1]{Ιούνιος 2014}{iounios2014}
\examyear{Ιούνιος 2014}

\begin{center}
\textbf{\Large Ιούνιος 2014 -- Θέματα \& Λύσεις}
\end{center}

\begin{question}[Θέμα 1]
Να βρεθεί το ορθογώνιο παραλληλεπίπεδο με τη μέγιστη δυνατή χωρητικότητα (όγκο) αν το άθροισμα των τριών ακμών του ισούται με $a$.
\end{question}

\begin{answer}
\textbf{Λύση:}

\textbf{Διατύπωση:}
Μεγιστοποίηση $V = xyz$ υπό τον περιορισμό $x+y+z = a$ και $x,y,z > 0$.
Ισοδύναμα: $\min f(x,y,z) = -xyz$.

\textbf{KKT:}
Lagrangian: $L = -xyz + \lambda(x+y+z-a)$.
Εξισώσεις:
1. $-yz + \lambda = 0 \implies \lambda = yz$
2. $-xz + \lambda = 0 \implies \lambda = xz$
3. $-xy + \lambda = 0 \implies \lambda = xy$
4. $x+y+z = a$

Από (1),(2): $yz = xz \implies y=x$ (για $z \neq 0$).
Από (2),(3): $xz = xy \implies z=y$ (για $x \neq 0$).
Άρα $x=y=z$.
Αντικαθιστώντας στον περιορισμό:
\[ 3x = a \implies x = \frac{a}{3} \]
Άρα το βέλτιστο σχήμα είναι \textbf{κύβος} με πλευρά $a/3$.
\end{answer}

% ============== Ιούλιος 2018 (Πρώην Ιούνιος 2018) ==============
\newpage
\phantomsection
\hypertarget{ioulios2018}{}
\pdfbookmark[1]{Ιούλιος 2018}{ioulios2018}
\examyear{Ιούλιος 2018}

\begin{center}
\textbf{\Large Ιούλιος 2018 -- Θέματα \& Λύσεις}
\end{center}

\begin{question}[Θέμα 1]
Δίνονται οι παρατηρήσεις $y_1, y_2, \dots, y_m$.
\textbf{α)} Να δειχθεί ότι για την ελαχιστοποίηση της συνάρτησης $f(x) = \frac{1}{2} \sum_{i=1}^m (y_i - x)^2$ με τη μέθοδο της μέγιστης καθόδου, υπάρχει βήμα $\gamma_k$ τέτοιο ώστε ο αλγόριθμος να συγκλίνει στη βέλτιστη λύση $\bar{x}$ σε μία επανάληψη.
\textbf{β)} Να δειχθεί ότι ο αναδρομικός τύπος του αριθμητικού μέσου είναι $x_{k+1} = x_k + \frac{1}{k+1}(y_{k+1} - x_k)$.
\end{question}

\begin{answer}
\textbf{Λύση:}

\textbf{α)} Η βέλτιστη λύση είναι ο αριθμητικός μέσος $\bar{x} = \frac{1}{m} \sum y_i$.
Η παράγωγος είναι:
\[ \nabla f(x) = \sum_{i=1}^m -(y_i - x) = \sum y_i - m x \]
Ο αλγόριθμος μέγιστης καθόδου:
\[ x_{k+1} = x_k - \gamma_k \nabla f(x_k) = x_k + \gamma_k (\sum y_i - m x_k) \]
\[ x_{k+1} = x_k (1 - \gamma_k m) + \gamma_k \sum y_i \]
Αν επιλέξουμε $\gamma_k = \frac{1}{m}$:
\[ x_{k+1} = x_k (1 - 1) + \frac{1}{m} \sum y_i = \bar{x} \]
Άρα συγκλίνει σε 1 βήμα.

\textbf{β)} Για τον αναδρομικό τύπο του μέσου όρου $x_k$ (των $k$ παρατηρήσεων):
\[ x_k = \frac{1}{k} \sum_{i=1}^k y_i \implies \sum_{i=1}^k y_i = k x_k \]
Για $k+1$:
\[ x_{k+1} = \frac{1}{k+1} \sum_{i=1}^{k+1} y_i = \frac{1}{k+1} (\sum_{i=1}^k y_i + y_{k+1}) \]
\[ x_{k+1} = \frac{1}{k+1} (k x_k + y_{k+1}) = \frac{k x_k + x_k - x_k + y_{k+1}}{k+1} = \frac{(k+1)x_k + y_{k+1} - x_k}{k+1} \]
\[ x_{k+1} = x_k + \frac{1}{k+1}(y_{k+1} - x_k) \]
\end{answer}

\begin{question}[Θέμα 2]
Πρόβλημα Network Utility Maximization (NUM) σε δίκτυο με 3 κόμβους.
Πηγή $\to$ Κόμβος 1 ($x_{r1}$) $\to$ Κόμβος 2 ($x_{r2}$).
Χωρητικότητες $C_1, C_2$.
Συνάρτηση χρησιμότητας $U_i(x_i) = \ln x_i$ (Proportional Fairness).
Να διατυπωθεί το πρόβλημα και να δειχθεί ότι η συνθήκη βελτιστότητας οδηγεί σε δίκαιη κατανομή.
\end{question}

\begin{answer}
\textbf{Λύση:}

\textbf{Διατύπωση:}
\[ \max \sum_{i} \ln x_i \iff \min \sum_{i} -\ln x_i \]
Περιορισμοί:
\[ \sum_{i \in L_j} x_i \leq C_j \quad (\forall \text{ link } j) \]

Λόγω κυρτότητας των συναρτήσεων $-\ln x_i$, το πρόβλημα έχει μοναδικό ολικό ελάχιστο.
Η συνθήκη 1ης τάξης για κυρτή συνάρτηση:
\[ \nabla f(x^*)^T (x - x^*) \geq 0 \]
Για $f(x) = -\sum \ln x_i$, $\nabla f_i = -1/x_i$.
Αυτό οδηγεί στο κριτήριο:
\[ \sum_i \frac{x_i - x_i^*}{x_i^*} \leq 0 \]
που αποτελεί τον ορισμό της αναλογικής δικαιοσύνης (Proportional Fairness).
Αν προσπαθήσουμε να αυξήσουμε τον ρυθμό μιας ροής κατά ποσοστό $p$, το άθροισμα των ποσοστιαίων μειώσεων των άλλων ροών θα είναι μεγαλύτερο.
\end{answer}

% ============== Φεβρουάριος 2020 ==============
\newpage
\phantomsection
\hypertarget{febrouarios2020}{}
\pdfbookmark[1]{Φεβρουάριος 2020}{febrouarios2020}
\examyear{Φεβρουάριος 2020}

\begin{center}
\textbf{\Large Φεβρουάριος 2020 -- Θέματα \& Λύσεις}
\end{center}

\begin{question}[Θέμα 2]
Δύο διυλιστήρια $\Delta_1, \Delta_2$ τροφοδοτούν δύο κέντρα κατανάλωσης $K_1, K_2$.
Υπάρχουν οι εξής συνδέσεις με τα αντίστοιχα κόστη:
$\Delta_1 \to K_1$ (ροή $x_1$, μοναδιαίο κόστος $l_1$)
$\Delta_1 \to K_2$ (ροή $x_2$, μοναδιαίο κόστος $l_2$)
$\Delta_2 \to K_2$ (ροή $x_4$, μοναδιαίο κόστος $l_4$)
$K_1 \to K_2$ (ροή $x_3$, μοναδιαίο κόστος $l_3$)

Δίνονται οι μέγιστες παραγωγές των διυλιστηρίων $D_1, D_2$ και οι απαιτήσεις των κέντρων $K_1, K_2$.
Να διατυπωθεί το πρόβλημα ελαχιστοποίησης του τετραγωνικού κόστους:
\[ f(x) = l_1 x_1^2 + l_2 x_2^2 + l_3 x_3^2 + l_4 x_4^2 \]
\end{question}

\begin{answer}
\textbf{Λύση:}

\textbf{Α) Διατύπωση Προβλήματος}
Ελαχιστοποίηση κόστους:
\[ \min f(x) = l_1 x_1^2 + l_2 x_2^2 + l_3 x_3^2 + l_4 x_4^2 \]

Υπό τους περιορισμούς:
1. \underline{Ισοζύγιο Ροών (Νόμοι Kirchoff):}
   Βάσει της διόρθωσης ότι το $K_1$ έχει μόνο εισροές, η ροή $x_3$ έχει φορά $K_2 \to K_1$.
   Για το $K_1$: Εισροές $x_1, x_3$, Ζήτηση $K_1$.
   \[ x_1 + x_3 = K_1 \implies x_1 + x_3 - K_1 = 0 \quad (h_1) \]
   Για το $K_2$: Εισροές $x_2, x_4$, Εκροή $x_3$, Ζήτηση $K_2$.
   \[ x_2 + x_4 = K_2 + x_3 \implies x_2 - x_3 + x_4 - K_2 = 0 \quad (h_2) \]

2. \underline{Περιορισμοί Παραγωγής:}
   Μέγιστη παραγωγή $\Delta_1$:
   \[ x_1 + x_2 \leq D_1 \implies x_1 + x_2 - D_1 \leq 0 \quad (g_1) \]
   Μέγιστη παραγωγή $\Delta_2$:
   \[ x_4 \leq D_2 \implies x_4 - D_2 \leq 0 \quad (g_2) \]

3. \underline{Περιορισμοί Χωρητικότητας Αγωγών:}
   Έστω $C_i$ η μέγιστη χωρητικότητα του αγωγού $i$.
   \[ x_i \leq C_i \implies x_i - C_i \leq 0, \quad i=1,2,3,4 \quad (k_i) \]

4. \underline{Μη Αρνητικότητα:}
   \[ x_i \geq 0 \iff -x_i \leq 0, \quad i=1,2,3,4 \]

\textbf{Β) Συνθήκες KKT}
Σχηματίζουμε τη συνάρτηση Lagrangian.
Σημείωση: Τα $\nu_i$ (νι) είναι οι πολλαπλασιαστές Lagrange που αντιστοιχούν στους περιορισμούς μη αρνητικότητας ($-x_i \leq 0$).
Τα $\rho_i$ είναι οι πολλαπλασιαστές για τις χωρητικότητες αγωγών.

\[
L(x, \lambda, \mu, \rho, \nu) = \sum_{i=1}^4 l_i x_i^2 + \mu_1 h_1 + \mu_2 h_2 + \lambda_1 g_1 + \lambda_2 g_2 + \sum_{i=1}^4 \rho_i (x_i - C_i) - \sum_{i=1}^4 \nu_i x_i
\]

Οι συνθήκες στασιμότητας ($\frac{\partial L}{\partial x_i} = 0$):
1. $\frac{\partial L}{\partial x_1}: 2l_1 x_1 + \mu_1 + \lambda_1 + \rho_1 - \nu_1 = 0$
2. $\frac{\partial L}{\partial x_2}: 2l_2 x_2 + \mu_2 + \lambda_1 + \rho_2 - \nu_2 = 0$
3. $\frac{\partial L}{\partial x_3}: 2l_3 x_3 + \mu_1 - \mu_2 + \rho_3 - \nu_3 = 0$
4. $\frac{\partial L}{\partial x_4}: 2l_4 x_4 + \mu_2 + \lambda_2 + \rho_4 - \nu_4 = 0$

Επιπλέον ισχύουν οι συνθήκες συμπληρωματικότητας:
\begin{itemize}
    \item $\lambda_1 (x_1+x_2-D_1)=0, \quad \lambda_2 (x_4-D_2)=0$
    \item $\rho_i (x_i - C_i) = 0, \quad \forall i$
    \item $\nu_i x_i = 0, \quad \forall i$
    \item $\lambda, \rho, \nu \geq 0$
\end{itemize}
\end{answer}

% ============== Φεβρουάριος 2021 ==============
\newpage
\phantomsection
\hypertarget{fevrouarios2021}{}
\pdfbookmark[1]{Φεβρουάριος 2021}{fevrouarios2021}
\examyear{Φεβρουάριος 2021}

\begin{center}
\textbf{\Large Φεβρουάριος 2021 -- Θέματα \& Λύσεις}
\end{center}

\begin{answer}
\textbf{Λύση:}

\textbf{Α) Δυνατότητα Χρήσης Μεθόδων Προβολής}
Για να χρησιμοποιηθούν μέθοδοι προβολής gradient πρέπει:
1. Η συνάρτηση $f$ να είναι κυρτή.
   $H_f = \begin{bmatrix} 2 & 2 \\ 2 & 4 \end{bmatrix}$. Οι ιδιοτιμές είναι $\lambda = 3 \pm \sqrt{5} > 0$.
   Άρα $H_f$ θετικά ορισμένος $\implies f$ αυστηρά κυρτή.
2. Το σύνολο των περιορισμών $\Omega$ να είναι κυρτό.
   Οι περιορισμοί ορίζουν ένα ορθογώνιο $[-1,1] \times [-2, 0.5]$, το οποίο είναι κυρτό σύνολο.
Άρα \textbf{μπορούν} να χρησιμοποιηθούν μέθοδοι προβολής.

\textbf{Β) Μέθοδος Newton με Προβολή}
Ο αλγόριθμος είναι:
\[ x_{k+1} = P_\Omega (x_k - \gamma_k H^{-1} \nabla f(x_k)) \]
Υπολογίζουμε $H^{-1} \nabla f$:
$\nabla f = \begin{bmatrix} 2x+2y \\ 2x+4y \end{bmatrix}$.
$H^{-1} = \frac{1}{4} \begin{bmatrix} 4 & -2 \\ -2 & 2 \end{bmatrix} = \begin{bmatrix} 1 & -0.5 \\ -0.5 & 0.5 \end{bmatrix}$.
$H^{-1} \nabla f = \begin{bmatrix} 1 & -0.5 \\ -0.5 & 0.5 \end{bmatrix} \begin{bmatrix} 2x+2y \\ 2x+4y \end{bmatrix} = \begin{bmatrix} 2x+2y -x-2y \\ -x-y +x+2y \end{bmatrix} = \begin{bmatrix} x \\ y \end{bmatrix}$.
Άρα το βήμα Newton μας πάει στο:
$x_{new} = x_k - 1 \cdot \begin{bmatrix} x_k \\ y_k \end{bmatrix} = \begin{bmatrix} 0 \\ 0 \end{bmatrix}$.
Το σημείο $(0,0)$ ανήκει στο σύνολο περιορισμών (εφικτό).
Άρα η μέθοδος συγκλίνει στο $(0,0)$ σε \textbf{μία επανάληψη}.
Η προβολή είναι ταυτοτική αφού το σημείο είναι εντός.
\end{answer}

% ============== Σεπτέμβριος 2021 ==============
\newpage
\phantomsection
\hypertarget{septembrios2021}{}
\pdfbookmark[1]{Σεπτέμβριος 2021}{septembrios2021}
\examyear{Σεπτέμβριος 2021}

\begin{center}
\textbf{\Large Σεπτέμβριος 2021 -- Θέματα \& Λύσεις}
\end{center}

\begin{question}[Θέμα 1 (6 μονάδες)]
Να λυθεί αναλυτικά το πρόβλημα:
\[ \min \left( -\sum_{i=1}^{3} \ln x_i \right) \]
υπό τους περιορισμούς:
\begin{align*}
x_1 + x_2 \leq 2 \\
x_1 + x_3 \leq 1 \\
x_i \geq 0, i=1,2,3
\end{align*}
\end{question}

\begin{answer}
\textbf{Λύση:}

To πρόβλημα είναι ισοδύναμο με τη μεγιστοποίηση του γινομένου $x_1 x_2 x_3$.
Οι περιορισμοί $x_i \geq 0$ ικανοποιούνται αυστηρά ($x_i > 0$) λόγω του λογαρίθμου.
Εφαρμόζουμε συνθήκες KKT με Lagrangian:
\[ L = -\ln x_1 - \ln x_2 - \ln x_3 + \lambda_1(x_1+x_2-2) + \lambda_2(x_1+x_3-1) \]
Συνθήκες στασιμότητας:
1) $\frac{\partial L}{\partial x_1} = -\frac{1}{x_1} + \lambda_1 + \lambda_2 = 0 \implies x_1 = \frac{1}{\lambda_1+\lambda_2}$
2) $\frac{\partial L}{\partial x_2} = -\frac{1}{x_2} + \lambda_1 = 0 \implies x_2 = \frac{1}{\lambda_1}$
3) $\frac{\partial L}{\partial x_3} = -\frac{1}{x_3} + \lambda_2 = 0 \implies x_3 = \frac{1}{\lambda_2}$

Υποθέτουμε ότι οι περιορισμοί είναι ενεργοί ($\lambda_1, \lambda_2 > 0$):
$x_1 + x_2 = 2 \implies \frac{1}{\lambda_1+\lambda_2} + \frac{1}{\lambda_1} = 2$
$x_1 + x_3 = 1 \implies \frac{1}{\lambda_1+\lambda_2} + \frac{1}{\lambda_2} = 1$

Λύνοντας το σύστημα (από μεταγραφή χειρόγραφου):
$3x_1^2 - 6x_1 + 2 = 0 \implies x_1 = \frac{6 \pm \sqrt{36-24}}{6} = 1 \pm \frac{\sqrt{3}}{3}$.
Επειδή $x_1 + x_3 = 1 \implies x_1 < 1$ (αφού $x_3 > 0$).
Άρα $\boxed{x_1 = 1 - \frac{\sqrt{3}}{3}}$.

Τότε $x_3 = 1 - x_1 = \frac{\sqrt{3}}{3}$.
Και $x_2 = 2 - x_1 = 1 + \frac{\sqrt{3}}{3}$.

Επαλήθευση:
$x_1 \approx 0.42$, $x_2 \approx 1.57$, $x_3 \approx 0.57$.
$\lambda_2 = 1/x_3 = \sqrt{3} > 0$.
$\lambda_1 = 1/x_2 > 0$.
Οι συνθήκες πληρούνται.
\end{answer}

\begin{question}[Θέμα 2 (4 μονάδες)]
Για την ελαχιστοποίηση της συνάρτησης
\[ f(x,y) = x^2 + y^2 - 2x - 4y + 5 \]
ως προς $x, y$, εκκινώντας από οποιοδήποτε σημείο, έχετε στη διάθεσή σας τους αλγορίθμους: α) μέγιστης κλίσης, β) Newton, γ) Levenberg-Marquardt. Ποιον από τους τρεις θα επιλέξετε αν σας ενδιαφέρει η εύρεση του ελαχίστου να πραγματοποιηθεί μ' όσο το δυνατόν λιγότερα βήματα; Να αιτιολογήσετε την απάντησή σας.
\end{question}

\begin{answer}
\textbf{Λύση:}

Η συνάρτηση $f(x,y)$ είναι τετραγωνική:
$f(x,y) = (x-1)^2 + (y-2)^2$.
Ο πίνακας Hessian είναι $H = \begin{bmatrix} 2 & 0 \\ 0 & 2 \end{bmatrix}$.
Είναι σταθερός και θετικά ορισμένος (ιδιοτιμές 2, 2).

\textbf{Επιλογή: Μέθοδος Newton.}
\textbf{Αιτιολόγηση:} Η μέθοδος Newton, όταν εφαρμόζεται σε τετραγωνική συνάρτηση με θετικά ορισμένο Hessian, συγκλίνει στο ακριβές ελάχιστο σε \textbf{μία μόνο επανάληψη}, ανεξαρτήτως του σημείου εκκίνησης.
Η Levenberg-Marquardt θα συνέκλινε επίσης γρήγορα (ισοδυναμεί με Newton αν επιλεγεί παράμετρος $\mu=0$ καθώς $H$ θετικά ορισμένος), αλλά η Newton είναι η πιο άμεση απάντηση για "λιγότερα βήματα" (1 βήμα). Η μέγιστη κλίση θα χρειαστεί περισσότερα βήματα (εκτός αν ξεκινήσουμε από ειδικό σημείο).
\end{answer}


% ============== Φεβρουάριος 2022 ==============
\newpage
\phantomsection
\hypertarget{fevrouarios2022}{}
\pdfbookmark[1]{Φεβρουάριος 2022}{fevrouarios2022}
\examyear{Φεβρουάριος 2022}

\begin{center}
\textbf{\Large Φεβρουάριος 2022 -- Θέματα \& Λύσεις}
\end{center}

\begin{question}[Θέμα 1 (6 μονάδες)]
Έστω ένα σύστημα μίας εισόδου $x$ και μίας εξόδου $y$ για το οποίο γνωρίζουμε ότι η μαθηματική σχέση που συνδέει την είσοδο με την έξοδο είναι:
\[ y = a_0 + a_1 x + a_2 x^2 \]
Για την είσοδο και την έξοδο έχουμε συλλέξει τρία ζεύγη μετρήσεων όπως στον πίνακα που ακολουθεί:

\begin{center}
\begin{tabular}{|c|c|c|c|}
\hline
i & 1 & 2 & 3 \\ \hline
x(i) & 1 & 2 & 3 \\ \hline
y(i) & 4 & 9 & 16 \\ \hline
\end{tabular}
\end{center}

\textbf{Α) (2 μονάδες)} Να προσδιοριστούν αναλυτικά οι σταθερές αλλά άγνωστες παράμετροι $a_0, a_1, a_2$ έτσι ώστε να ελαχιστοποιείται το άθροισμα των τετραγώνων των σφαλμάτων μεταξύ της εκτιμώμενης και της πραγματικής εξόδου. \\
\textbf{Β) (2 μονάδες)} Να υλοποιήσετε τη μέθοδο της μέγιστης κλίσης με σταθερό βήμα για την αναδρομική εκτίμηση των $a_0, a_1, a_2$. \\
\textbf{Γ) (2 μονάδες)} Πως πρέπει να επιλεγούν οι αρχικές εκτιμήσεις $\hat{a}_0(1), \hat{a}_1(1), \hat{a}_2(1)$ ώστε ο αλγόριθμος του ερωτήματος (Β) να συγκλίνει στις βέλτιστες τιμές σε μία επανάληψη;
\end{question}

\begin{answer}
\textbf{Λύση:}

\textbf{Α) Αναλυτική Επίλυση}
Το μοντέλο είναι $y = a_0 + a_1 x + a_2 x^2$.
Θέλουμε να ελαχιστοποιήσουμε το άθροισμα σφαλμάτων:
\[ E(a) = \sum_{i=1}^3 (y_i - (a_0 + a_1 x_i + a_2 x_i^2))^2 \]
Αντικαθιστώντας τα δεδομένα $(1,4), (2,9), (3,16)$:
\[ E = (4 - (a_0 + a_1 + a_2))^2 + (9 - (a_0 + 2a_1 + 4a_2))^2 + (16 - (a_0 + 3a_1 + 9a_2))^2 \]
Παρατηρούμε ότι $4=2^2, 9=3^2, 16=4^2$, δηλαδή $y = (x+1)^2 = x^2 + 2x + 1$.
Άρα η ακριβής λύση είναι:
\[ \boxed{a_0 = 1, \quad a_1 = 2, \quad a_2 = 1} \]

\textbf{Β) Μέθοδος Μέγιστης Κλίσης}
Η παράγωγος $\nabla E$ ως προς $a = [a_0, a_1, a_2]^T$ είναι:
\[
\nabla E = -2 \sum (y_i - \hat{y}_i) \begin{bmatrix} 1 \\ x_i \\ x_i^2 \end{bmatrix}
\]
Ο αλγόριθμος ανανέωσης είναι:
\[
a_{k+1} = a_k - \gamma \nabla E(a_k)
\]

\textbf{Γ) Σύγκλιση σε μία επανάληψη}
Για να συγκλίνει ο αλγόριθμος μέγιστης κλίσης σε μία επανάληψη, πρέπει να ξεκινήσουμε από το βέλτιστο!
Δηλαδή, αν επιλέξουμε αρχικές τιμές:
\[
\hat{a}_0(1) = 1, \quad \hat{a}_1(1) = 2, \quad \hat{a}_2(1) = 1
\]
τότε $\nabla E(a_{start}) = 0$, και ο αλγόριθμος τερματίζει αμέσως.
\end{answer}

\begin{question}[Θέμα 2 (4 μονάδες)]
\textbf{Α) (1 μονάδα)} Να μελετηθεί ως προς την κυρτότητα η συνάρτηση:
\[ f(x) = \begin{cases} x^2, & 0 \leq x \leq 1 \\ 2x - 1, & 1 < x \end{cases} \]

\textbf{Β) (1 μονάδα)} Δίνονται οι περιορισμοί:
\begin{align*}
x_1 \geq 0 \\
x_2 \geq 0 \\
-x_1 + (x_2 - 1)^2 \leq 0
\end{align*}
Είναι το $(x_1^*, x_2^*) = (0, 1)$ εφικτό και κανονικό; Να αιτιολογήσετε την απάντησή σας.

\textbf{Γ) (1 μονάδα)} Δίνεται το πρόβλημα:
\[ \min(x_1^2 + x_2^2 - 4x_1 - 4x_2) \]
υπό τους περιορισμούς:
\begin{align*}
-x_1^2 + x_2 \leq 0 \\
x_1 + x_2 \geq 0
\end{align*}
\textbf{α)} Να γραφούν οι συνθήκες Karush-Kuhn-Tucker. \\
\textbf{β)} Είναι οι συνθήκες αυτές ικανές και αναγκαίες για το δοσμένο πρόβλημα; Να αιτιολογήσετε την απάντησή σας.

\textbf{Δ) (1 μονάδα)} Στους γενετικούς αλγορίθμους το μέγεθος του πληθυσμού είναι κρίσιμη λειτουργική παράμετρος. Τι θα συμβεί πιθανότητα στη λειτουργία του αλγορίθμου αν το μέγεθος του πληθυσμού αυξηθεί και τι αν μειωθεί;
\end{question}

\begin{answer}
\textbf{Λύση:}

\textbf{Α) (1 μονάδα)}
Για $0 \le x \le 1$: $f(x) = x^2$, $f''(x) = 2 > 0$ (κυρτή).
Για $x > 1$: $f(x) = 2x-1$, $f''(x) = 0$ (κυρτή).
Ελέγχουμε την κυρτότητα στο σημείο αλλαγής $x=1$ και γενικά τον ορισμό:
$f(\lambda x_1 + (1-\lambda)x_2) \le \lambda f(x_1) + (1-\lambda)f(x_2)$.
Αποδεικνύεται ότι η συνάρτηση είναι **κυρτή** (convex) σε όλο το πεδίο ορισμού, καθώς η εφαπτομένη (παράγωγος) είναι αύξουσα ($2x$ για $x \le 1$ που φτάνει το 2, και 2 για $x > 1$). Η κλίση δεν μειώνεται.

\textbf{Β) (1 μονάδα)}
Σημείο $(0,1)$.
Περιορισμοί: $g_1 = -x_1 \le 0$, $g_2 = -x_2 \le 0$, $g_3 = -x_1 + (x_2-1)^2 \le 0$.
Έλεγχος εφικτότητας: $0 \ge 0$, $1 \ge 0$, $0 + (1-1)^2 = 0 \le 0$. \textbf{Εφικτό.}
Ενεργοί περιορισμοί: $g_1$ και $g_3$.
Κλίσεις ενεργών περιορισμών:
$\nabla g_1 = \begin{bmatrix} -1 \\ 0 \end{bmatrix}$
$\nabla g_3 = \begin{bmatrix} -1 \\ 2(x_2-1) \end{bmatrix}_{(0,1)} = \begin{bmatrix} -1 \\ 0 \end{bmatrix}$
Παρατηρούμε ότι $\nabla g_1 = \nabla g_3$. Τα διανύσματα είναι γραμμικά εξαρτημένα.
Άρα το σημείο \textbf{δεν είναι κανονικό}.

\textbf{Γ) (1 μονάδα)}
Πρόβλημα: $\min(x_1^2+x_2^2-4x_1-4x_2)$ με $g_1(x) = x_1^2 - x_2 \le 0$ και $g_2(x) = -x_1 - x_2 \le 0$. (Διορθώθηκε βάσει σημειώσεων για να βγάζει νόημα η κυρτότητα).
\textbf{α) Συνθήκες KKT:}
$L = f + \lambda_1 g_1 + \lambda_2 g_2$.
1. Στασιμότητα: $\nabla f + \lambda_1 \nabla g_1 + \lambda_2 \nabla g_2 = 0$.
   $\begin{bmatrix} 2x_1-4 \\ 2x_2-4 \end{bmatrix} + \lambda_1 \begin{bmatrix} 2x_1 \\ -1 \end{bmatrix} + \lambda_2 \begin{bmatrix} -1 \\ -1 \end{bmatrix} = \vec{0}$
2. Εφικτότητα: $g_1(x) \le 0, g_2(x) \le 0$.
3. Συμπληρωματικότητα: $\lambda_1 g_1(x) = 0, \lambda_2 g_2(x) = 0$.
4. Μη αρνητικότητα: $\lambda_1, \lambda_2 \ge 0$.

\textbf{β) Ικανές και Αναγκαίες;}
Είναι \textbf{ικανές και αναγκαίες} διότι:
1. Η $f$ είναι κυρτή (θετικά ορισμένη τετραγωνική).
2. Οι περιορισμοί $g_1, g_2$ ορίζουν κυρτό σύνολο (αν θεωρήσουμε την τυπική μορφή που δίνει κυρτότητα).

\textbf{Δ) (1 μονάδα)}
Αν το μέγεθος του πληθυσμού \textbf{αυξηθεί}: Ο αλγόριθμος εξερευνά μεγαλύτερο μέρος του χώρου αναζήτησης (καλύτερο exploration), μειώνοντας την πιθανότητα εγκλωβισμού σε τοπικά ελάχιστα, αλλά αυξάνεται ο υπολογιστικός χρόνος ανά γενιά (πιο αργή σύγκλιση σε χρόνο).
Αν το μέγεθος του πληθυσμού \textbf{μειωθεί}: Ο αλγόριθμος τρέχει πιο γρήγορα ανά γενιά, αλλά υπάρχει μεγαλύτερος κίνδυνος πρόωρης σύγκλισης σε τοπικό ελάχιστο (χειρότερο exploration).
\end{answer}


% ============== Ιούνιος 2022 ==============
\newpage
\phantomsection
\hypertarget{iounios2022}{}
\pdfbookmark[1]{Ιούνιος 2022}{iounios2022}
\examyear{Ιούνιος 2022}

\begin{center}
\textbf{\Large Ιούνιος 2022 -- Θέματα \& Λύσεις}
\end{center}

\begin{question}[Θέμα 1 (6 μονάδες)]
Δίνεται η συνάρτηση $f(x,y) = x^2 + xy + y^2 - x + y, \quad x,y \in \mathbb{R}$. \\
\textbf{Α) (3 μονάδες)} Να υπολογιστούν αναλυτικά τα βέλτιστα σημεία και να σχεδιαστούν ποιοτικά οι ισοσταθμικές καμπύλες. \\
\textbf{Β) (3 μονάδες)} Για τον αριθμητικό υπολογισμό των βέλτιστων σημείων του προηγούμενου ερωτήματος χρησιμοποιούμε: α) τη μέθοδο της μέγιστης κλίσης, β) τη μέθοδο Newton, γ) τη μέθοδο εσωτερικής βελτιστοποίησης. Για σημείο εκκίνησης το $(x,y) = (-3,1)$ σχεδιάστε ποιοτικά, αλλά με αιτιολόγηση, πάνω στις ισοσταθμικές καμπύλες του προηγούμενου ερωτήματος, την εξέλιξη των αλγορίθμων. Να δώσετε διαφορετικό διάγραμμα για κάθε αλγόριθμο.
\end{question}

\begin{answer}
\textbf{Λύση:}

\textbf{Α) Αναλυτικός Υπολογισμός}
Υπολογίζουμε την κλίση της $f$:
\[ \nabla f = \begin{bmatrix} 2x + y - 1 \\ x + 2y + 1 \end{bmatrix} \]
Θέτουμε $\nabla f = 0$:
\[
\begin{cases}
2x + y = 1 \implies y = 1 - 2x \\
x + 2y = -1
\end{cases}
\]
Αντικαθιστώντας το $y$ στη δεύτερη:
\[
x + 2(1 - 2x) = -1 \implies x + 2 - 4x = -1 \implies -3x = -3 \implies x = 1
\]
Άρα $y = 1 - 2(1) = -1$.
Το κρίσιμο σημείο είναι το $\boxed{(1, -1)}$.

Για να χαρακτηρίσουμε το σημείο, υπολογίζουμε τον πίνακα Hessian:
\[ H = \begin{bmatrix} 2 & 1 \\ 1 & 2 \end{bmatrix} \]
Η ορίζουσα είναι $\det(H) = 4 - 1 = 3 > 0$ και η πρώτη ελάσσονα $2 > 0$.
Άρα ο πίνακας είναι θετικά ορισμένος και το σημείο $(1, -1)$ είναι \textbf{ολικό ελάχιστο}.

\textbf{Ισοσταθμικές καμπύλες}:
Οι καμπύλες $f(x,y) = c$ είναι ελλείψεις με κέντρο το $(1, -1)$ και κύριους άξονες στραμμένους κατά 45 μοίρες (λόγω του όρου $xy$).

\textbf{Β) Εξέλιξη Αλγορίθμων από $(-3, 1)$}
Το σημείο εκκίνησης είναι το $A(-3, 1)$. Ο στόχος είναι το $O(1, -1)$.

\begin{enumerate}
    \item \textbf{Μέθοδος Μέγιστης Κλίσης}:
    Η κατεύθυνση είναι $d_k = -\nabla f(x_k)$. Οι διαδοχικές κατευθύνσεις είναι κάθετες μεταξύ τους (Zig-zag). Αφού οι ισοσταθμικές είναι ελλείψεις (όχι κύκλοι), η μέθοδος θα συγκλίνει αργά με πολλές επαναλήψεις, κάνοντας ζιγκ-ζαγκ προς το ελάχιστο.

    \item \textbf{Μέθοδος Newton}:
    Δεδομένου ότι η συνάρτηση είναι αυστηρά τετραγωνική (quadratic), η μέθοδος Newton συγκλίνει σε \textbf{μία επανάληψη}.
    Η τροχιά θα είναι ένα ευθύγραμμο τμήμα απευθείας από το $(-3, 1)$ στο $(1, -1)$.

    \item \textbf{Μέθοδος Συζυγών Κλίσεων (Conjugate Gradients)}:
    Για τετραγωνική συνάρτηση $n$ μεταβλητών, συγκλίνει σε το πολύ $n$ βήματα. Εδώ $n=2$.
    Άρα θα φτάσει στο ελάχιστο σε το πολύ 2 βήματα. Το πρώτο βήμα είναι ίδιο με τη μέγιστη κλίση, και το δεύτερο διορθώνει την κατεύθυνση για να πέσει κατευθείαν στο κέντρο.
\end{enumerate}
\end{answer}

\begin{question}[Θέμα 2 (4 μονάδες)]
Έστω ο κύκλος κέντρου $(0,0)$ και ακτίνας $\rho > 0$. Χρησιμοποιώντας τη μέθοδο ποινής να αποδείξετε πως απ' όλα τα εγγεγραμμένα στον κύκλο ορθογώνια παραλληλόγραμμα το τετράγωνο έχει το μεγαλύτερο εμβαδό. Ποιο είναι το μήκος της πλευράς αυτού του τετραγώνου;
\end{question}

\begin{answer}
\textbf{Λύση:}

\textbf{Λύση με Μέθοδο Ποινής}
Θέλουμε να μεγιστοποιήσουμε το εμβαδόν $E = ab$ (όπου $a,b$ οι πλευρές), υπό τον περιορισμό ότι το ορθογώνιο είναι εγγεγραμμένο στον κύκλο.
Η διαγώνιος του ορθογωνίου είναι ίση με τη διάμετρο του κύκλου, άρα: $a^2 + b^2 = (2\rho)^2 = 4\rho^2$.
Το πρόβλημα μετασχηματίζεται σε:
\[ \min (-ab) \quad \text{υπό τον περιορισμό} \quad h(a,b) = a^2 + b^2 - 4\rho^2 = 0 \]

Ορίζουμε τη συνάρτηση ποινής:
\[ P(a,b,r_k) = -ab + \frac{r_k}{2}(a^2 + b^2 - 4\rho^2)^2 \]
Υπολογίζουμε τις μερικές παραγώγους και τις εξισώνουμε με το μηδέν:
\[ \frac{\partial P}{\partial a} = -b + r_k(a^2 + b^2 - 4\rho^2) \cdot 2a = 0 \]
\[ \frac{\partial P}{\partial b} = -a + r_k(a^2 + b^2 - 4\rho^2) \cdot 2b = 0 \]
Από την πρώτη εξίσωση: $2a r_k (a^2 + b^2 - 4\rho^2) = b$.
Από τη δεύτερη εξίσωση: $2b r_k (a^2 + b^2 - 4\rho^2) = a$.

Διαιρώντας κατά μέλη (υποθέτοντας $a,b \neq 0$):
\[ \frac{a}{b} = \frac{b}{a} \implies a^2 = b^2 \implies a = b \]
Άρα το βέλτιστο ορθογώνιο είναι \textbf{τετράγωνο}.

Για να βρούμε την πλευρά, αντικαθιστούμε στον περιορισμό (καθώς $r_k \to \infty$, ο περιορισμός ικανοποιείται):
\[ a^2 + a^2 = 4\rho^2 \implies 2a^2 = 4\rho^2 \implies a^2 = 2\rho^2 \implies a = \sqrt{2}\rho \]
Επομένως, το μήκος της πλευράς είναι $\boxed{a = \rho\sqrt{2}}$.
\end{answer}


% ============== Σεπτέμβριος 2022 ==============
\newpage
\phantomsection
\hypertarget{septembrios2022}{}
\pdfbookmark[1]{Σεπτέμβριος 2022}{septembrios2022}
\examyear{Σεπτέμβριος 2022}

\begin{center}
\textbf{\Large Σεπτέμβριος 2022 -- Θέματα \& Λύσεις}
\end{center}

\begin{question}[Θέμα 1 (4 μονάδες)]
Μια αεροπορική εταιρία επιτρέπει στους επιβάτες της τη μεταφορά μιας μόνο αποσκευής σχήματος ορθογωνίου παραλληλεπιπέδου της οποίας οι διαστάσεις $x$, $y$, $z$ πρέπει να ικανοποιούν:
\[
x + y + z \leq a
\]
όπου $a > 0$ δεδομένη σταθερά. Να υπολογιστούν αναλυτικά οι βέλτιστες διαστάσεις μιας αποσκευής που πληροί την παραπάνω συνθήκη και έχει μέγιστο όγκο.
\end{question}

\begin{answer}
\textbf{Λύση:}

Θέλουμε να μεγιστοποιήσουμε τον όγκο $V(x,y,z) = xyz$ υπό τον περιορισμό $x + y + z \leq a$ και $x,y,z > 0$.

\begin{tcolorbox}[colback=partbg, colframe=blue!30!black, title=Ανισότητα Αριθμητικού-Γεωμετρικού Μέσου (AM-GM)]
Για θετικούς αριθμούς $x_1, x_2, \dots, x_n$ ισχύει:
\[ \frac{x_1 + x_2 + \dots + x_n}{n} \geq \sqrt[n]{x_1 x_2 \dots x_n} \]
Η ισότητα ισχύει αν και μόνο αν $x_1 = x_2 = \dots = x_n$.
\end{tcolorbox}

Εφαρμόζουμε την ανισότητα AM-GM για τους τρεις θετικούς αριθμούς $x, y, z$:
\[ \frac{x+y+z}{3} \geq \sqrt[3]{xyz} \]

Υψώνουμε στην τρίτη δύναμη:
\[ \left( \frac{x+y+z}{3} \right)^3 \geq xyz = V \]

Από τον περιορισμό έχουμε $x+y+z \leq a$, οπότε:
\[ V \leq \left( \frac{a}{3} \right)^3 = \frac{a^3}{27} \]

Για να μεγιστοποιηθεί ο όγκος, πρέπει να ισχύει η ισότητα στην ανισότητα AM-GM. Αυτό συμβαίνει όταν:
\[ x = y = z \]

Αντικαθιστώντας στον ενεργό περιορισμό $x+y+z = a$:
\[ 3x = a \implies x = \frac{a}{3} \]

Άρα, οι βέλτιστες διαστάσεις είναι:
\[ \boxed{x = y = z = \frac{a}{3}} \]
και ο μέγιστος όγκος είναι $V_{max} = \frac{a^3}{27}$.
\end{answer}

\begin{question}[Θέμα 2 (3 μονάδες)]
Για την ελαχιστοποίηση της συνάρτησης
\[
f(x, y) = x^2 + y^2 - 2x - 4y + 5
\]
ως προς $x$, $y$ εκκινώντας από οποιοδήποτε σημείο έχετε στη διάθεσή σας τους αλγορίθμους: α) μέγιστης κλίσης, β) Newton, γ) Levenberg-Marquardt. Ποιον από τους τρεις θα επιλέξετε αν σας ενδιαφέρει η εύρεση του ελαχίστου να πραγματοποιηθεί μ' όσο το δυνατό λιγότερες επαναλήψεις, εκτελώντας σε κάθε επανάληψη τις λιγότερες δυνατές πράξεις; Να αιτιολογήσετε την απάντησή σας.
\end{question}

\begin{answer}
\textbf{Λύση:}

Η δοσμένη συνάρτηση είναι:
\[ f(x, y) = x^2 + y^2 - 2x - 4y + 5 \]
Παρατηρούμε ότι μπορεί να γραφτεί στη μορφή (συμπλήρωση τετραγώνου):
\[ f(x, y) = (x^2 - 2x + 1) + (y^2 - 4y + 4) = (x-1)^2 + (y-2)^2 \]
Πρόκειται για μια αυστηρά κυρτή τετραγωνική συνάρτηση.

Υπολογίζουμε την κλίση ($\nabla f$) και τον Εσσιανό πίνακα ($\nabla^2 f$):
\[ \nabla f(x,y) = \begin{bmatrix} \frac{\partial f}{\partial x} \\ \frac{\partial f}{\partial y} \end{bmatrix} = \begin{bmatrix} 2x - 2 \\ 2y - 4 \end{bmatrix} \]
\[ H(x,y) = \nabla^2 f(x,y) = \begin{bmatrix} \frac{\partial^2 f}{\partial x^2} & \frac{\partial^2 f}{\partial x \partial y} \\ \frac{\partial^2 f}{\partial y \partial x} & \frac{\partial^2 f}{\partial y^2} \end{bmatrix} = \begin{bmatrix} 2 & 0 \\ 0 & 2 \end{bmatrix} \]

Παρατηρούμε ότι ο πίνακας $H$ είναι σταθερός, διαγώνιος και θετικά ορισμένος (ιδιοτιμές $\lambda_1 = \lambda_2 = 2 > 0$).

\begin{tcolorbox}[colback=partbg, colframe=blue!30!black, title=Θεώρημα 5.2.4 (Σύγκλιση Newton)]
Αν η $f(x)$ είναι τετραγωνική συνάρτηση με θετικά ορισμένο Εσσιανό πίνακα, τότε η μέθοδος του Newton συγκλίνει στο ακριβές ελάχιστο $x^*$ σε \textbf{μία} μόνο επανάληψη, ξεκινώντας από οποιοδήποτε αρχικό σημείο $x_0$.
\end{tcolorbox}

Συνεπώς:
\begin{enumerate}
    \item \textbf{Μέθοδος Μέγιστης Κλίσης:} Έχει γραμμικό ρυθμό σύγκλισης και ενδέχεται να χρειαστεί πολλές επαναλήψεις, ειδικά αν οι ισοσταθμικές καμπύλες είναι επιμήκεις (αν και εδώ είναι κύκλοι, οπότε θα σύγλινε γρήγορα, αλλά όχι σε 1 βήμα γενικά).
    \item \textbf{Μέθοδος Newton:} Θα βρει το ελάχιστο ακριβώς σε \textbf{1 επανάληψη} ανεξαρτήτως αρχικού σημείου.
    \item \textbf{Μέθοδος Levenberg-Marquardt:} Είναι μια παραλλαγή της μεθόδου Newton ($H + \mu I$). Αν επιλεχθεί $\mu = 0$, ταυτίζεται με τη Newton. Αν $\mu > 0$, θα χρειαστεί περισσότερες επαναλήψεις.
\end{enumerate}

\textbf{Επιλογή:}
Θα επιλέξουμε τη \textbf{μέθοδο Newton}, διότι για τετραγωνική συνάρτηση εντοπίζει το ολικό ελάχιστο σε \textbf{μία μόλις επανάληψη}, ελαχιστοποιώντας έτσι το συνολικό υπολογιστικό κόστος.
\end{answer}

\begin{question}[Θέμα 3 (3 μονάδες)]
Θέλουμε να ελαχιστοποιήσουμε τη συνάρτηση
\[
f(x, y) = x^2 + 2y^2 + 2xy
\]
ως προς $x$, $y$ με περιορισμούς:
\begin{align*}
-1 \leq x \leq 1 \\
-2 \leq y \leq \frac{1}{2}
\end{align*}

Μπορούν να χρησιμοποιηθούν μέθοδοι προβολής για να λυθεί το πρόβλημα; Να αιτιολογηθεί η απάντησή σας.
\end{question}

\begin{answer}
\textbf{Λύση:}

Για να εφαρμοστεί η μέθοδος της Προβολής της Κλίσης (Gradient Projection) και να εγγυηθούμε τη σύγκλισή της στο ολικό ελάχιστο, πρέπει να πληρούνται δύο προϋποθέσεις:
\begin{enumerate}
    \item Η αντικειμενική συνάρτηση $f(x)$ να είναι κυρτή.
    \item Το σύνολο των περιορισμών $\Omega$ να είναι κυρτό σύνολο.
\end{enumerate}

\textbf{Βήμα 1: Έλεγχος κυρτότητας της $f(x,y)$}
Υπολογίζουμε τον Εσσιανό πίνακα της $f(x,y) = x^2 + 2y^2 + 2xy$:
\[ \nabla f = \begin{bmatrix} 2x + 2y \\ 4y + 2x \end{bmatrix}, \quad H = \nabla^2 f = \begin{bmatrix} 2 & 2 \\ 2 & 4 \end{bmatrix} \]
Βρίσκουμε τις κύριες ελάσσονες ορίζουσες ή τις ιδιοτιμές.
Ιδιοτιμές ($\det(H - \lambda I) = 0$):
\[ (2-\lambda)(4-\lambda) - 4 = 0 \implies \lambda^2 - 6\lambda + 8 - 4 = 0 \implies \lambda^2 - 6\lambda + 4 = 0 \]
\[ \Delta = 36 - 16 = 20 \implies \lambda_{1,2} = \frac{6 \pm \sqrt{20}}{2} = 3 \pm \sqrt{5} \]
Και οι δύο ιδιοτιμές είναι θετικές ($3+\sqrt{5} > 0$ και $3-\sqrt{5} > 0$).
Άρα, ο πίνακας $H$ είναι \textbf{θετικά ορισμένος}, οπότε η $f$ είναι \textbf{αυστηρά κυρτή} (και άρα κυρτή) στο $\mathbb{R}^2$.

\textbf{Βήμα 2: Έλεγχος κυρτότητας του $\Omega$}
Οι περιορισμοί είναι:
\[ \Omega = \{ (x,y) \in \mathbb{R}^2 \mid -1 \leq x \leq 1, \ -2 \leq y \leq \frac{1}{2} \} \]
Το σύνολο $\Omega$ είναι ένα ορθογώνιο και ως τομή ημιεπιπέδων (γραμμικοί περιορισμοί), είναι \textbf{κυρτό σύνολο}.

\textbf{Συμπέρασμα:}
Ναι, μπορούν να χρησιμοποιηθούν μέθοδοι προβολής (όπως η μέθοδος Προβολής της Κλίσης) διότι το πρόβλημα είναι κυρτό (ελαχιστοποίηση κυρτής συνάρτησης σε κυρτό σύνολο). 
Επιπλέον, επειδή οι περιορισμοί είναι απλά όρια (box constraints), η προβολή ενός σημείου $(x,y)$ στο $\Omega$ είναι υπολογιστικά πολύ εύκολη:
\[ P_\Omega(x,y) = \left( \min(1, \max(-1, x)), \ \min(1/2, \max(-2, y)) \right) \]
\end{answer}




% ============== Φεβρουάριος 2023 ==============
\newpage
\phantomsection
\hypertarget{fevrouarios2023}{}
\pdfbookmark[1]{Φεβρουάριος 2023}{fevrouarios2023}
\examyear{Φεβρουάριος 2023}

\begin{center}
\textbf{\Large Φεβρουάριος 2023 -- Θέματα \& Λύσεις}
\end{center}

\begin{question}[Θέμα 1 (5 μονάδες)]
\textbf{Α) (2.5 μονάδες)}

Έστω ο κύκλος κέντρου $(0,0)$ και ακτίνας $\rho > 0$. Με χρήση της ανισότητας Αριθμητικού Γεωμετρικού Μέσου να δείξετε ότι, από όλα τα εγγεγραμμένα στον κύκλο ορθογώνια παραλληλόγραμμα, το τετράγωνο έχει το μέγιστο εμβαδόν. Προσδιορίστε και το μέγιστο εμβαδόν και την πλευρά του τετραγώνου.

\textbf{Β) (2.5 μονάδες)} Να λυθεί το πρόβλημα του Θέματος 1Α) με την μέθοδο ποινής.
\end{question}

\begin{answer}
\textbf{Λύση:}

Κύκλος κέντρου $(0,0)$ ακτίνας $\rho$. Εγγεγραμμένο ορθογώνιο με πλευρές $a, b$.

\textbf{Λύση με AM-GM:}
Η διαγώνιος του ορθογωνίου ισούται με τη διάμετρο: $a^2 + b^2 = (2\rho)^2 = 4\rho^2$.
Εμβαδόν: $E = ab$.

Από AM-GM: $\frac{a^2 + b^2}{2} \geq \sqrt{a^2 b^2} = ab$
$\frac{4\rho^2}{2} \geq ab \implies E \leq 2\rho^2$.

Η ισότητα επιτυγχάνεται όταν $a^2 = b^2 \implies a = b$ (τετράγωνο).
Για τετράγωνο: $2a^2 = 4\rho^2 \implies a = \sqrt{2}\rho$.
$\boxed{E_{max} = 2\rho^2}$, πλευρά τετραγώνου $\boxed{a = \sqrt{2}\rho}$.

\vspace{1em}

\textbf{Μέθοδος ποινής:}

$\max E = ab$ υπό $a^2 + b^2 = 4\rho^2 \iff \min -ab$ υπό $h(a,b) = a^2 + b^2 - 4\rho^2 = 0$.

Συνάρτηση ποινής: $P(a,b,\mu) = -ab + \frac{\mu}{2}(a^2 + b^2 - 4\rho^2)^2$
$\nabla P = 0$:
$-b + 2\mu(a^2+b^2-4\rho^2) \cdot a = 0$
$-a + 2\mu(a^2+b^2-4\rho^2) \cdot b = 0$

Διαιρώντας: $\frac{b}{a} = \frac{a}{b} \implies a = b$.
Αντικαθιστώντας: $2a^2 = 4\rho^2 \implies a = \sqrt{2}\rho$.
Για $\mu \to \infty$: $(a,b) \to (\sqrt{2}\rho, \sqrt{2}\rho)$.
\end{answer}

\begin{question}[Θέμα 2 (2 μονάδες)]
Θέλουμε να ελαχιστοποιήσουμε τη συνάρτηση
\[
f(x, y) = x^2 + 2y^2 + 2xy
\]
ως προς $x$, $y$ με περιορισμούς:
\begin{align*}
-1 \leq x \leq 1 \\
-2 \leq y \leq 1/2
\end{align*}

Να δείξετε ότι μπορούν να χρησιμοποιηθούν μέθοδοι προβολής για να λυθεί το πρόβλημα.
\end{question}

\begin{answer}
\textbf{Λύση:}

Για να μπορούν να χρησιμοποιηθούν μέθοδοι προβολής, πρέπει το πρόβλημα βελτιστοποίησης να είναι κυρτό (Convex Programming). Αυτό απαιτεί δύο συνθήκες:
\begin{enumerate}
    \item Η αντικειμενική συνάρτηση $f(x,y)$ να είναι κυρτή.
    \item Το σύνολο των περιορισμών $S$ να είναι κυρτό.
\end{enumerate}

\textbf{1. Έλεγχος Κυρτότητας της $f(x,y)$:}
Η συνάρτηση είναι $f(x,y) = x^2 + 2y^2 + 2xy$.
Υπολογίζουμε την Εσσιανή μήτρα (Hessian Matrix):
\[ \nabla f = \begin{bmatrix} 2x + 2y \\ 4y + 2x \end{bmatrix} \]
\[ H_f = \begin{bmatrix} \frac{\partial^2 f}{\partial x^2} & \frac{\partial^2 f}{\partial x \partial y} \\ \frac{\partial^2 f}{\partial y \partial x} & \frac{\partial^2 f}{\partial y^2} \end{bmatrix} = \begin{bmatrix} 2 & 2 \\ 2 & 4 \end{bmatrix} \]
Οι κύριες ελάσσοες ορίζουσες είναι:
\begin{itemize}
    \item $\Delta_1 = 2 > 0$
    \item $\Delta_2 = \det(H) = 2 \cdot 4 - 2 \cdot 2 = 8 - 4 = 4 > 0$
\end{itemize}
Επειδή όλες οι κύριες ελάσσοες ορίζουσες είναι θετικές, ο πίνακας $H_f$ είναι θετικά ορισμένος παντού.
Άρα η συνάρτηση $f$ είναι \textbf{αυστηρά κυρτή}.

\textbf{2. Έλεγχος Κυρτότητας των Περιορισμών:}
Οι περιορισμοί $S = \{(x,y) \in \mathbb{R}^2 \mid -1 \leq x \leq 1, -2 \leq y \leq 1/2\}$ ορίζουν ένα ορθογώνιο.
Κάθε ορθογώνιο είναι κυρτό σύνολο (ως τομή ημιεπιπέδων που είναι κυρτά σύνολα).

\textbf{Συμπέρασμα:}
Εφόσον έχουμε ελαχιστοποίηση κυρτής συνάρτησης σε κυρτό σύνολο περιορισμών, μπορούν να χρησιμοποιηθούν επαναληπτικές μέθοδοι προβολής (όπως η Projected Gradient Descent), οι οποίες εγγυώνται σύγκλιση στο ολικό ελάχιστο.
\end{answer}

\begin{question}[Θέμα 3 (3 μονάδες)]
Χαρακτηρίστε τις παρακάτω προτάσεις ως Σωστές ή Λανθασμένες

\textbf{1) (1 μονάδα)} Η $f(x) = |x|$, $x \in \mathbb{R}$ είναι γνήσια κυρτή διότι
\[
f(x_1) + \nabla f^T(x_1)(x_2 - x_1) < f(x_2), \quad \forall x_1, x_2 \in \mathbb{R}.
\]

\textbf{2) (1 μονάδα)} Η συνάρτηση
\[
f(x) = \begin{cases}
x^2, & 0 \leq x \leq 1 \\
x + 1, & 1 < x
\end{cases}
\]
είναι γνήσια κυρτή.

\textbf{3) (1 μονάδα)} Δίνονται οι περιορισμοί
\begin{align*}
x_1 \geq 0 \\
x_2 \geq 0 \\
-x_1 + (x_2 - 1)^2 \geq 0
\end{align*}

Το $(x_1^*, x_2^*) = (0, 1)$ είναι εφικτό και κανονικό.
\end{question}

\begin{answer}
\textbf{Λύση:}

\textbf{1) (1 μονάδα)} $f(x) = |x|$ γνήσια κυρτή;
\textbf{ΛΑΘΟΣ.} Η $f(x) = |x|$ είναι κυρτή αλλά \textbf{όχι γνήσια κυρτή}.
Για $x_1, x_2 > 0$: $f(x_1) + f'(x_1)(x_2-x_1) = x_1 + (x_2-x_1) = x_2 = f(x_2)$ (ισότητα, όχι αυστηρή ανισότητα).
Επιπλέον, η $f$ δεν είναι διαφορίσιμη στο $x=0$.

\textbf{2) (1 μονάδα)} $f(x) = \begin{cases} x^2, & 0 \leq x \leq 1 \\ x+1, & x > 1 \end{cases}$ γνήσια κυρτή;
\textbf{ΛΑΘΟΣ.} Για $x > 1$, $f''(x) = 0$ (γραμμική), άρα δεν είναι γνήσια κυρτή εκεί.
Επίσης, στο $x=1$: $f(1) = 1$ (από αριστερά), $f(1) = 2$ (από δεξιά) - ασυνέχεια!
Άρα η συνάρτηση δεν είναι καν συνεχής.

\textbf{3) (1 μονάδα)} $(0,1)$ εφικτό και κανονικό;
Περιορισμοί: $x_1 \geq 0$, $x_2 \geq 0$, $-x_1 + (x_2-1)^2 \geq 0$.
Για $(0,1)$: $0 \geq 0$ $\checkmark$, $1 \geq 0$ $\checkmark$, $0 + 0 = 0 \geq 0$ $\checkmark$. \textbf{Εφικτό.}
Ενεργοί περιορισμοί: $g_1 = -x_1 = 0$, $g_3 = -x_1 + (x_2-1)^2 = 0$.
$\nabla g_1 = \begin{bmatrix} -1 \\ 0 \end{bmatrix}$, $\nabla g_3 = \begin{bmatrix} -1 \\ 2(x_2-1) \end{bmatrix}|_{(0,1)} = \begin{bmatrix} -1 \\ 0 \end{bmatrix}$.
Οι κλίσεις είναι γραμμικά εξαρτημένες! \textbf{Δεν είναι κανονικό} (LICQ αποτυγχάνει).
\textbf{ΛΑΘΟΣ.}
\end{answer}




% ============== Ιούνιος 2023 ==============
\newpage
\phantomsection
\hypertarget{iounios2023}{}
\pdfbookmark[1]{Ιούνιος 2023}{iounios2023}
\examyear{Ιούνιος 2023}

\begin{center}
\textbf{\Large Ιούνιος 2023 -- Θέματα \& Λύσεις}
\end{center}

\begin{question}[Θέμα 2 (7.5 μονάδες)]
Έστω το σύστημα μίας εισόδου $x$ και μίας εξόδου $y$ για το οποίο γνωρίζουμε ότι η μαθηματική σχέση που συνδέει την είσοδο με την έξοδο είναι:
\[
y = a_0 + a_1 x + a_2 x^2
\]
όπου $a_0, a_1, a_2$ είναι κάποιες σταθερές αλλά άγνωστες παράμετροι. Με σκοπό την εκτίμηση των άγνωστων παραμέτρων συλλέγουμε τρία ζεύγη μετρήσεων όπως στον πίνακα που ακολουθεί:

\begin{center}
\begin{tabular}{c|cccc}
$i$ & 1 & 2 & 3 \\
\midrule
$x(i)$ & 1 & 2 & 3 \\
$y(i)$ & 4 & 9 & 16 \\
\end{tabular}
\end{center}

\textbf{Α) (5 μονάδες)} Να προσδιοριστούν αναλυτικά (χωρίς την εφαρμογή κάποιας μεθόδου βελτιστοποίησης) οι εκτιμήσεις $\hat{a}_0, \hat{a}_1, \hat{a}_2$ των άγνωστων παραμέτρων, έτσι ώστε να ελαχιστοποιείται το άθροισμα των τετραγώνων των σφαλμάτων μεταξύ της πραγματικής εξόδου και της εκτίμησης αυτής $\hat{y}$. Να θεωρήσετε ότι:
\[
\hat{y} = \hat{a}_0 + \hat{a}_1 x + \hat{a}_2 x^2
\]

\textbf{Β) (2.5 μονάδες)} Να σχεδιάσετε τη μέθοδο της μέγιστης κλίσης με σταθερό βήμα για την αναδρομική εκτίμηση των άγνωστων παραμέτρων.
\end{question}

\begin{answer}
\textbf{Λύση:}

\textbf{Α) Αναλυτική Επίλυση (Ελάχιστα Τετράγωνα)}
Το μοντέλο είναι $y = a_0 + a_1 x + a_2 x^2$.
Θέλουμε να ελαχιστοποιήσουμε το άθροισμα σφαλμάτων:
\[ E(a) = \sum_{i=1}^3 (y_i - (a_0 + a_1 x_i + a_2 x_i^2))^2 \]
Αντικαθιστώντας τα δεδομένα $(1,4), (2,9), (3,16)$:
\[ E = (4 - (a_0 + a_1 + a_2))^2 + (9 - (a_0 + 2a_1 + 4a_2))^2 + (16 - (a_0 + 3a_1 + 9a_2))^2 \]
Για να ελαχιστοποιηθεί το $E$, μηδενίζουμε τις μερικές παραγώγους ή λύνουμε το σύστημα εξισώσεων:
\[ 4 = a_0 + a_1 + a_2 \]
\[ 9 = a_0 + 2a_1 + 4a_2 \]
\[ 16 = a_0 + 3a_1 + 9a_2 \]
(Παρατηρούμε ότι $4=2^2, 9=3^2, 16=4^2$, δηλαδή $y = (x+1)^2 = x^2 + 2x + 1$).
Άρα η ακριβής λύση είναι:
\[ \boxed{a_0 = 1, \quad a_1 = 2, \quad a_2 = 1} \]

\textbf{Β) Μέθοδος Μέγιστης Κλίσης}
Η παράγωγος $\nabla E$ ως προς $a = [a_0, a_1, a_2]^T$ είναι:
\[
\nabla E = \begin{bmatrix}
\frac{\partial E}{\partial a_0} \\
\frac{\partial E}{\partial a_1} \\
\frac{\partial E}{\partial a_2}
\end{bmatrix} = -2 \sum (y_i - \hat{y}_i) \begin{bmatrix} 1 \\ x_i \\ x_i^2 \end{bmatrix}
\]
Ο αλγόριθμος ανανέωσης είναι:
\[
a_{k+1} = a_k - \gamma \nabla E(a_k)
\]

\textbf{Γ) Σύγκλιση σε μία επανάληψη}
Για να συγκλίνει ο αλγόριθμος μέγιστης κλίσης σε μία επανάληψη, πρέπει να ξεκινήσουμε από το βέλτιστο!
Δηλαδή, αν επιλέξουμε αρχικές τιμές:
\[
\hat{a}_0(1) = 1, \quad \hat{a}_1(1) = 2, \quad \hat{a}_2(1) = 1
\]
τότε $\nabla E(a_{start}) = 0$, οπότε $a_{next} = a_{start} - 0 = a_{start}$.
Η λύση βρίσκεται ήδη εκεί.
\end{answer}




% ============== Σεπτέμβριος 2023 ==============
\newpage
\phantomsection
\hypertarget{septembrios2023}{}
\pdfbookmark[1]{Σεπτέμβριος 2023}{septembrios2023}
\examyear{Σεπτέμβριος 2023}

\begin{center}
\textbf{\Large Σεπτέμβριος 2023 -- Θέματα \& Λύσεις}
\end{center}

\begin{question}[Θέμα 1 (2 μονάδες)]
Έστω $f(x)$ μια κυρτή συνάρτηση στο $\mathbb{R}$. Αν $x_1, x_2, x_3$ τρεις πραγματικοί αριθμοί που ικανοποιούν $x_2 < x_3 < x_1$, να δειχθεί ότι:
\[
\frac{f(x_2) - f(x_1)}{x_2 - x_1} \leq \frac{f(x_3) - f(x_1)}{x_3 - x_1}
\]
\end{question}

\begin{answer}
\textbf{Λύση:}

Έστω $f(x)$ μια κυρτή συνάρτηση στο $\mathbb{R}$ και $x_1 < x_2 < x_3$.
Μπορούμε να εκφράσουμε το $x_2$ ως κυρτό συνδυασμό των $x_1$ και $x_3$:
\[
x_2 = \frac{x_3-x_2}{x_3-x_1} x_1 + \frac{x_2-x_1}{x_3-x_1} x_3
\]
Οι συντελεστές $\lambda_1 = \frac{x_3-x_2}{x_3-x_1}$ και $\lambda_2 = \frac{x_2-x_1}{x_3-x_1}$ είναι θετικοί και το άθροισμά τους είναι 1.

Λόγω της κυρτότητας της $f$, ισχύει:
\[
f(x_2) \leq \frac{x_3-x_2}{x_3-x_1} f(x_1) + \frac{x_2-x_1}{x_3-x_1} f(x_3) \quad (1)
\]
Επιπλέον, μπορούμε να γράψουμε το $f(x_2)$ ως:
\[
f(x_2) = 1 \cdot f(x_2) = \left( \frac{x_3-x_2}{x_3-x_1} + \frac{x_2-x_1}{x_3-x_1} \right) f(x_2) = \frac{x_3-x_2}{x_3-x_1} f(x_2) + \frac{x_2-x_1}{x_3-x_1} f(x_2) \quad (2)
\]
Από (1) και (2):
\[
\frac{x_3-x_2}{x_3-x_1} f(x_2) + \frac{x_2-x_1}{x_3-x_1} f(x_2) \leq \frac{x_3-x_2}{x_3-x_1} f(x_1) + \frac{x_2-x_1}{x_3-x_1} f(x_3)
\]
Αναδιατάσσοντας τους όρους:
\[
\frac{x_3-x_2}{x_3-x_1} (f(x_2) - f(x_1)) \leq \frac{x_2-x_1}{x_3-x_1} (f(x_3) - f(x_2))
\]
ή ισοδύναμα (πολλαπλασιάζοντας με $x_3-x_1 > 0$):
\[
(x_3-x_2) (f(x_2) - f(x_1)) \leq (x_2-x_1) (f(x_3) - f(x_2))
\]
Διαιρώντας με $(x_3-x_2)(x_2-x_1)$ (που είναι θετικό) καταλήγουμε στο ζητούμενο:
\[
\frac{f(x_2) - f(x_1)}{x_2 - x_1} \leq \frac{f(x_3) - f(x_2)}{x_3 - x_2}
\]
Η γεωμετρική ερμηνεία είναι ότι η κλίση της χορδής $(x_1, x_2)$ είναι μικρότερη ή ίση από την κλίση της χορδής $(x_2, x_3)$ για μια κυρτή συνάρτηση.
\end{answer}

\begin{question}[Θέμα 2 (4 μονάδες)]
Μια εταιρεία κατασκευάζει ηχεία σε δύο εργοστάσια που βρίσκονται σε διαφορετικές πόλεις, το $E_1$ και το $E_2$. Το $E_1$ παράγει 350 ηχεία ανά ημέρα, ενώ το $E_2$ 250 ηχεία ανά ημέρα. Τα ηχεία αποστέλλονται σε δύο αποθήκες την $A_1$ και την $A_2$. Οι αποθήκες θεωρούνται ότι είναι αρχικά γεμάτες. Επίσης, ηχεία αποστέλλονται από τις δύο αποθήκες στα κέντρα διανομής τριών πόλεων $\{K_1, K_2, K_3\}$. Μεταξύ των αποθηκών υπάρχει δυνατότητα αποστολής περιορισμένου αριθμού ηχείων.

Έστω ότι τα κέντρα διανομής $K_i$, $i = 1, 2, 3$ ζητούν 150, 200 και 220 ηχεία ανά ημέρα αντίστοιχα. Η μετοχή των αποθηκών αποστολή ηχείων περιορίζεται σε 35 ηχεία ανά ημέρα, με κόστος μεταφοράς 2 χρηματικές μονάδες ανά ηχείο. Το κόστος μεταφοράς (σε χρηματικές μονάδες) ανά ηχείο για τους υπόλοιπους προορισμούς είναι: $(E_1 \to A_1) = 36$, $(E_1 \to A_2) = 27$, $(E_2 \to A_1) = 33$, $(E_2 \to A_2) = 29$, $(A_1 \to K_1) = 21$, $(A_1 \to K_2) = 15$, $(A_1 \to K_3) = 13$, $(A_2 \to K_1) = 16$, $(A_2 \to K_2) = 16$, $(A_2 \to K_3) = 17$.

Να διατυπωθεί το πρόβλημα βελτιστοποίησης, η επίλυση του οποίου οδηγεί στην ελαχιστοποίηση του συνολικού κόστους διανομής ηχείων ανά ημέρα.
\end{question}

\begin{answer}
\textbf{Λύση:}

\textbf{Δικτυακή Αναπαράσταση:}
\begin{center}
\begin{tikzpicture}[
    node distance=2.5cm,
    layer/.style={execute at begin scope={\node[start chain]}},
    mycircle/.style={draw, circle, minimum size=0.9cm, font=\small\bfseries},
    tip/.style={->, >=stealth, shorten >=1pt},
    every node/.style={scale=0.9}
]

% Nodes
\node[mycircle, fillcyan] (E1) at (0, 2) {$E_1$};
\node[mycircle, fillcyan] (E2) at (0, -2) {$E_2$};

\node[mycircle, fillgreen] (A1) at (4, 2) {$A_1$};
\node[mycircle, fillgreen] (A2) at (4, -2) {$A_2$};

\node[mycircle, fillorange] (K1) at (8, 3) {$K_1$};
\node[mycircle, fillorange] (K2) at (8, 0) {$K_2$};
\node[mycircle, fillorange] (K3) at (8, -3) {$K_3$};

% Supplies/Demands Labels
\node[left=0.2cm of E1, align=right, font=\footnotesize] {Sup:\\350};
\node[left=0.2cm of E2, align=right, font=\footnotesize] {Sup:\\250};
\node[right=0.2cm of K1, font=\footnotesize] {Dem: 150};
\node[right=0.2cm of K2, font=\footnotesize] {Dem: 200};
\node[right=0.2cm of K3, font=\footnotesize] {Dem: 220};

% Edges E->A with costs
\draw[tip] (E1) -- node[above, font=\footnotesize] {36 ($x_1$)} (A1);
\draw[tip] (E1) -- node[above right, near start, font=\footnotesize] {27 ($x_2$)} (A2);
\draw[tip] (E2) -- node[below right, near start, font=\footnotesize] {33 ($x_3$)} (A1);
\draw[tip] (E2) -- node[below, font=\footnotesize] {29 ($x_4$)} (A2);

% Edges A<->A
\draw[tip, bend right=15] (A1) -- node[left, font=\footnotesize] {2 (Max 35) $x_{11}$} (A2);
\draw[tip, bend right=15] (A2) -- node[right, font=\footnotesize] {2 (Max 35) $x_{12}$} (A1);

% Edges A->K
% From A1
\draw[tip] (A1) -- node[above, font=\footnotesize] {21 ($x_5$)} (K1);
\draw[tip] (A1) -- node[above, near start, pos=0.7, font=\footnotesize] {15 ($x_6$)} (K2);
\draw[tip] (A1) -- node[above right, near start, pos=0.8, font=\footnotesize] {13 ($x_7$)} (K3);

% From A2
\draw[tip] (A2) -- node[below right, near start, pos=0.8, font=\footnotesize] {16 ($x_8$)} (K1);
\draw[tip] (A2) -- node[below, font=\footnotesize] {16 ($x_9$)} (K2);
\draw[tip] (A2) -- node[below, font=\footnotesize] {17 ($x_{10}$)} (K3);

\end{tikzpicture}
\end{center}

Ορίζουμε τις μεταβλητές απόφασης $x_{i}$ για τις ροές των ηχείων:
\begin{itemize}
    \item $x_1$: $E_1 \to A_1$, $x_2$: $E_1 \to A_2$ (Παραγωγή από $E_1$)
    \item $x_3$: $E_2 \to A_1$, $x_4$: $E_2 \to A_2$ (Παραγωγή από $E_2$)
    \item $x_5, x_6, x_7$: Ροές από $A_1$ προς $K_1, K_2, K_3$
    \item $x_8, x_9, x_{10}$: Ροές από $A_2$ προς $K_1, K_2, K_3$
    \item $x_{11}$: $A_1 \to A_2$, $x_{12}$: $A_2 \to A_1$ (Μεταφορά μεταξύ αποθηκών με όριο 35)
\end{itemize}

\textbf{Μαθηματική Διατύπωση (LP):}
\[
\min Z = 36x_1 + 27x_2 + 33x_3 + 29x_4 + 21x_5 + 15x_6 + 13x_7 + 16x_8 + 16x_9 + 17x_{10} + 2x_{11} + 2x_{12}
\]

\textbf{Περιορισμοί:}

1. \underline{Περιορισμοί Προσφοράς:}
\[ x_1 + x_2 \leq 350 \quad (E_1) \]
\[ x_3 + x_4 \leq 250 \quad (E_2) \]

2. \underline{Περιορισμοί Ζήτησης:}
\[ x_5 + x_8 = 150 \quad (K_1) \]
\[ x_6 + x_9 = 200 \quad (K_2) \]
\[ x_7 + x_{10} = 220 \quad (K_3) \]

3. \underline{Ισοζύγιο Ροών στις Αποθήκες (Εισροές = Εκροές):}
\[ x_1 + x_3 + x_{12} = x_5 + x_6 + x_7 + x_{11} \quad (A_1) \]
\[ x_2 + x_4 + x_{11} = x_8 + x_9 + x_{10} + x_{12} \quad (A_2) \]

4. \underline{Περιορισμοί Χωρητικότητας Συνδέσμων:}
\[ x_{11} \leq 35 \quad (A_1 \to A_2) \]
\[ x_{12} \leq 35 \quad (A_2 \to A_1) \]

5. \underline{Μη αρνητικότητα:}
\[ x_i \geq 0, \quad \forall i \]

\vspace{0.5em}
\hrule
\vspace{0.5em}

\textbf{Σημείωση (Βέλτιστη Λύση):}
Αν επιλύσουμε το πρόβλημα, προκύπτει ότι συμφέρει να διακινήσουμε το φορτίο κυρίως μέσω της αποθήκης $A_2$ (λόγω χαμηλότερου κόστους εισόδου $27, 29$ έναντι $36, 33$) και να χρησιμοποιήσουμε τον σύνδεσμο $A_2 \to A_1$ στο μέγιστο (35 μονάδες) για να εκμεταλλευτούμε το εξαιρετικά χαμηλό κόστος του $A_1 \to K_3$ (13).
Οι βέλτιστες ροές είναι:
\begin{itemize}
    \item $E_1 \to A_2$: 350 (Όλη η παραγωγή)
    \item $E_2 \to A_2$: 220 (Μέρος της παραγωγής)
    \item $A_2 \to A_1$: 35 (Μέγιστη χωρητικότητα συνδέσμου)
    \item $A_1 \to K_3$: 35 (Τροφοδοσία μέρους του $K_3$)
    \item $A_2 \to K_3$: 185, $A_2 \to K_1$: 150, $A_2 \to K_2$: 200
\end{itemize}
Σύνολο κόστους: $Z_{min} = 24735$.
\end{answer}

\begin{question}[Θέμα 3 (4 μονάδες)]
Η υλοποίηση της μεθόδου της μέγιστης καθόδου στηρίζεται στον αριθμητικό υπολογισμό του διανύσματος κλίσης της αντικειμενικής συνάρτησης, διαδικασία που εμπεριέχει σφάλματα στρογγυλοποίησης, διακριτοποίησης κλπ. Έστω η μέθοδος της μέγιστης καθόδου με σφάλμα, που χρησιμοποιεί βήμα με εξασθένηση
\[
\left(\lim_{k \to \infty} \gamma_k = 0\right).
\]
\[
x_{k+1} = x_k - \gamma_k \left(\nabla f(x_k) + \omega_k\right), \gamma_k > 0.
\]

Το $\omega_k$ παριστάνει σε κάθε επανάληψη της μεθόδου ένα μη-ελέγξιμο διάνυσμα σφάλματος. Το $\omega_k$ είναι μεν άγνωστο αλλά, γνωρίζουμε ότι το μέτρο του είναι φραγμένο άνω με φράγμα ανάλογο του βήματος $\gamma_k$. Στην αναδρομική εξίσωση της μεθόδου το $\nabla f(x_k)$ συμβολίζει την πραγματική αλλά άγνωστη τιμή του διανύσματος κλίσης και το $\nabla f(x_k) + \omega_k$ την εκτίμηση της άγνωστης αυτής τιμής. Συνεπώς, το $\nabla f(x_k) + \omega_k$ είναι γνωστό.

Υποθέτοντας ότι $\nabla f(x_k) \neq 0$, να δείξετε ότι υπάρχει πεπερασμένος αριθμός επαναλήψεων $\bar{k} > 0$ τέτοιος ώστε για κάθε $k \geq \bar{k}$ η κατεύθυνση αναζήτησης $-\left(\nabla f(x_k) + \omega_k\right)$ εξακολουθεί να ικανοποιεί την ιδιότητα της επαναληπτικής καθόδου, για αρκούντως μικρό $\gamma_k$.
\end{question}

\begin{answer}
\textbf{Λύση:}

Δίνεται η αναδρομική σχέση:
\[
x_{k+1} = x_k + d_k = x_k - \gamma_k (\nabla f(x_k) + \omega_k)
\]
με $\lim_{k \to \infty} \gamma_k = 0$ και $\|\omega_k\| \leq c \gamma_k$ (το σφάλμα είναι φραγμένο και φθίνει μαζί με το βήμα).
Υποθέτουμε $\nabla f(x_k) \neq 0$.

Για να έχουμε κατεύθυνση καθόδου $p_k = -(\nabla f(x_k) + \omega_k)$, πρέπει να ισχύει:
\[
\nabla f(x_k)^T p_k < 0 \iff \nabla f(x_k)^T [-(\nabla f(x_k) + \omega_k)] < 0
\]
\[
\iff \nabla f(x_k)^T (\nabla f(x_k) + \omega_k) > 0
\]
\[
\iff \|\nabla f(x_k)\|^2 + \nabla f(x_k)^T \omega_k > 0
\]
Ο πρώτος όρος $\|\nabla f(x_k)\|^2$ είναι θετικός (αφού $\nabla f(x_k) \neq 0$).
Πρέπει λοιπόν να ελέγξουμε αν το άθροισμα παραμένει θετικό.
Από την ανισότητα Cauchy-Schwarz:
\[
|\nabla f(x_k)^T \omega_k| \leq \|\nabla f(x_k)\| \|\omega_k\|
\]
Άρα:
\[
\nabla f(x_k)^T \omega_k \geq -\|\nabla f(x_k)\| \|\omega_k\| \geq -\|\nabla f(x_k)\| c \gamma_k
\]
Αντικαθιστώντας στην ανισότητα καθόδου:
\[
\|\nabla f(x_k)\|^2 + \nabla f(x_k)^T \omega_k \geq \|\nabla f(x_k)\|^2 - c \gamma_k \|\nabla f(x_k)\| = \|\nabla f(x_k)\| (\|\nabla f(x_k)\| - c \gamma_k)
\]
Για να ισχύει η ιδιότητα της καθόδου, αρκεί:
\[
\|\nabla f(x_k)\| - c \gamma_k > 0 \iff \gamma_k < \frac{\|\nabla f(x_k)\|}{c}
\]
Επειδή $\lim_{k \to \infty} \gamma_k = 0$ και $\|\nabla f(x_k)\|$ είναι μη μηδενικό (και υποθέτουμε φραγμένο μακριά από το 0 σε μια περιοχή μη στάσιμου σημείου), υπάρχει ένα $\bar{k}$ τέτοιο ώστε για κάθε $k \geq \bar{k}$, το $\gamma_k$ να είναι αρκετά μικρό ώστε να ικανοποιείται η συνθήκη.
Επομένως, μετά από πεπερασμένο αριθμό επαναλήψεων, η κατεύθυνση θα είναι πάντα κατεύθυνση καθόδου.
\end{answer}




% ============== Ιούνιος 2024 ==============
\newpage
\phantomsection
\hypertarget{iounios2024}{}
\pdfbookmark[1]{Ιούνιος 2024}{iounios2024}
\examyear{Ιούνιος 2024}

\begin{center}
\textbf{\Large Ιούνιος 2024 -- Θέματα \& Λύσεις}
\end{center}

\begin{question}[Θέμα 1 (3 μονάδες)]
Να χαρακτηριστούν ως Σωστές ή Λανθασμένες οι παρακάτω προτάσεις και να αιτιολογήσετε την απάντησή σας.

\textbf{Α) (1 μονάδα)} Η $f(x) = |x|$, $x \in \mathbb{R}$ είναι γνήσια κυρτή διότι:
\[
f(x_1) + \nabla f^T(x_1)(x_2 - x_1) < f(x_2), \quad \forall x_1, x_2 \in \mathbb{R}.
\]

\textbf{Β) (1 μονάδα)} Το $z = \begin{bmatrix} 3 & 1 \\ 2 & 2 \end{bmatrix}^T$ είναι κυρτός συνδυασμός των $x = [1\ 1]^T$, $y = [2\ 0]^T$.

\textbf{Γ) (1 μονάδα)} Για την ελαχιστοποίηση χωρίς περιορισμούς της συνάρτησης $f(x)$, $x \in \mathbb{R}$ η οποία είναι τουλάχιστον δύο φορές παραγωγίσιμη, ο αλγόριθμος:
\[
x_{k+1} = x_k - \gamma d_k, \quad \gamma > 0
\]
επιβάλει την ιδιότητα της επαναληπτικής καθόδου αρκεί:
\[
\nabla f^T(x_k) d_k < 0
\]
σε κάθε επανάληψη $k$ του αλγορίθμου.
\end{question}

\begin{answer}
\textbf{Λύση:}

\textbf{Α)} Η $f(x) = |x|$ δεν είναι γνήσια κυρτή γιατί δεν ικανοποιείται η ανισότητα
\[
f(x_1) + \nabla f(x_1)(x_2-x_1) < f(x_2)
\]
για όλες τις περιπτώσεις $x_1, x_2 \in \mathbb{R}$.
Συγκεκριμένα για τις περιπτώσεις $x_1, x_2 > 0$ και $x_1, x_2 < 0$, η ανισότητα γίνεται ισότητα, άρα η συνάρτηση δεν είναι αυστηρά κυρτή (είναι απλώς κυρτή).
\textbf{Απάντηση: ΛΑΘΟΣ}

\textbf{Β)} Για να είναι το $z$ κυρτός γραμμικός συνδυασμός των $x, y$ πρέπει να υπάρχουν $a, b \in [0,1]$ τέτοια ώστε:
\[ z = ax + by \]
\[ a + b = 1 \]
Αντικαθιστώντας τα διανύσματα:
\[
\begin{bmatrix} 3 \\ 1 \end{bmatrix} = a \begin{bmatrix} 1 \\ 1 \end{bmatrix} + b \begin{bmatrix} 2 \\ 0 \end{bmatrix} = \begin{bmatrix} a + 2b \\ a \end{bmatrix}
\]
Επιλύουμε το σύστημα:
\[ \begin{cases} a + 2b = 3 \\ a = 1 \end{cases} \implies \begin{cases} a = 1 \\ 1 + 2b = 3 \implies 2b = 2 \implies b = 1 \end{cases} \]
Το άθροισμα των συντελεστών είναι $a + b = 1 + 1 = 2 \neq 1$.
Επομένως, το $z$ \textbf{δεν} είναι κυρτός γραμμικός συνδυασμός (είναι απλώς γραμμικός συνδυασμός).
\textbf{Απάντηση: ΛΑΘΟΣ} (Οι συντελεστές που ικανοποιούν τη γραμμική σχέση αθροίζουν στο 2, όχι στο 1).

\textbf{Γ)} Αυτή η συνθήκη ($\nabla f^T(x_k) d_k < 0$) είναι απαραίτητη για να διασφαλιστεί ότι η κατεύθυνση αναζήτησης $d_k$ είναι κατεύθυνση καθόδου (descent direction). Αν ισχύει, τότε για μικρό βήμα $\gamma$, η τιμή της συνάρτησης μειώνεται.
\textbf{Απάντηση: ΣΩΣΤΟ}
\end{answer}

\begin{question}[Θέμα 2 (3 μονάδες)]
Να ελαχιστοποιηθεί αναλυτικά η συνάρτηση
\[
f(x, y) = \frac{4x^2 y^2 + x^2 + 4y^2}{xy^2}, \quad x > 0, y > 0.
\]
\end{question}

\begin{answer}
\textbf{Λύση:}

Η συνάρτηση γράφεται ως:
\[
f(x, y) = \frac{4x^2 y^2}{xy^2} + \frac{x^2}{xy^2} + \frac{4y^2}{xy^2}
= 4x + \frac{x}{y^2} + \frac{4}{x}, \quad x>0,\ y>0.
\]

Επειδή $\frac{x}{y^2} > 0$, έχουμε για όλα τα $x>0, y>0$:
\[
f(x,y) > 4x + \frac{4}{x}.
\]
Με AM--GM:
\[
4x + \frac{4}{x} \ge 2\sqrt{4x\cdot\frac{4}{x}} = 8,
\]
και ισότητα μόνο όταν $4x = \frac{4}{x} \iff x=1$.
Άρα για κάθε $x>0, y>0$:
\[
f(x,y) > 8.
\]

Επιπλέον, αν πάρουμε $x_n=1$ και $y_n\to +\infty$, τότε
\[
f(1,y_n)=4+\frac{1}{y_n^2}+4 = 8+\frac{1}{y_n^2} \to 8.
\]
Συνεπώς, $\inf\{f(x,y):x>0,y>0\}=8$, αλλά \textbf{δεν υπάρχει ελάχιστο} (η τιμή 8 δεν επιτυγχάνεται για κανένα πεπερασμένο $y$).
\end{answer}

\begin{question}[Θέμα 3 (4 μονάδες)]
Θέλουμε να ελαχιστοποιήσουμε τη συνάρτηση
\[
f(x, y) = x^2 + 2y^2 + 2xy
\]
ως προς $x$, $y$ με περιορισμούς:
\begin{align*}
-1 \leq x \leq 1 \\
-2 \leq y \leq 1/2
\end{align*}

Να δείξετε ότι μπορούν να χρησιμοποιηθούν μέθοδοι προβολής για να λυθεί το πρόβλημα.
\end{question}

\begin{answer}
\textbf{Λύση:}

Για να μπορούν να χρησιμοποιηθούν μέθοδοι προβολής (Gradient Projection Methods) για την επίλυση του προβλήματος, πρέπει να ικανοποιούνται δύο βασικές συνθήκες:
\begin{enumerate}
    \item Η αντικειμενική συνάρτηση $f(x,y)$ να είναι κυρτή.
    \item Το σύνολο των περιορισμών $\Omega$ να είναι κυρτό σύνολο.
\end{enumerate}

\textbf{1. Έλεγχος Κυρτότητας Συνάρτησης}
Η συνάρτηση είναι $f(x,y) = x^2 + 2y^2 + 2xy$.
Υπολογίζουμε την Εσσιανή μήτρα (Hessian Matrix):
\[ \nabla f = \begin{bmatrix} 2x + 2y \\ 4y + 2x \end{bmatrix} \]
\[ H_f = \begin{bmatrix} \frac{\partial^2 f}{\partial x^2} & \frac{\partial^2 f}{\partial x \partial y} \\ \frac{\partial^2 f}{\partial y \partial x} & \frac{\partial^2 f}{\partial y^2} \end{bmatrix} = \begin{bmatrix} 2 & 2 \\ 2 & 4 \end{bmatrix} \]
Οι κύριες ελάσσοες ορίζουσες είναι:
\begin{itemize}
    \item $\Delta_1 = 2 > 0$
    \item $\Delta_2 = \det(H) = 2 \cdot 4 - 2 \cdot 2 = 8 - 4 = 4 > 0$
\end{itemize}
Επειδή όλες οι κύριες ελάσσοες ορίζουσες είναι θετικές, ο πίνακας $H_f$ είναι θετικά ορισμένος για κάθε $(x,y)$.
Άρα η συνάρτηση $f$ είναι \textbf{αυστηρά κυρτή} (strictly convex).

\textbf{2. Έλεγχος Κυρτότητας Περιορισμών}
Οι περιορισμοί είναι:
\begin{align*}
-1 \leq x \leq 1 \\
-2 \leq y \leq 1/2
\end{align*}
Αυτοί ορίζουν ένα ορθογώνιο στο επίπεδο, δηλαδή το καρτεσιανό γινόμενο δύο κλειστών διαστημάτων: $\Omega = [-1, 1] \times [-2, 0.5]$.
Κάθε ορθογώνιο (και γενικότερα κάθε πολύεδρο) είναι \textbf{κυρτό σύνολο}.

\textbf{Συμπέρασμα:}
Εφόσον ελαχιστοποιούμε μια κυρτή συνάρτηση σε ένα κυρτό σύνολο, το πρόβλημα είναι Κυρτού Προγραμματισμού (Convex Programming). Επομένως, μπορούν να χρησιμοποιηθούν μέθοδοι προβολής (όπως η Projected Gradient Descent), οι οποίες εγγυώνται σύγκλιση στο ολικό ελάχιστο.
\end{answer}




% ============== Σεπτέμβριος 2024 ==============
\newpage
\phantomsection
\hypertarget{septembrios2024}{}
\pdfbookmark[1]{Σεπτέμβριος 2024}{septembrios2024}
\examyear{Σεπτέμβριος 2024}

\begin{center}
\textbf{\Large Σεπτέμβριος 2024 -- Θέματα \& Λύσεις}
\end{center}

\begin{question}[Θέμα 1 (3 μονάδες)]
Να αποδείξετε ότι αν διαφορίσιμη και $x^* \in \mathbb{R}^n$ είναι ένα τοπικό ελάχιστο της συνάρτησης $f$, τότε $\nabla f(x^*) = \vec{0}$.
\end{question}

\begin{answer}
\textbf{Λύση:}

\textbf{Απόδειξη:}
Έστω $f: \mathbb{R}^n \to \mathbb{R}$ διαφορίσιμη και $x^*$ τοπικό ελάχιστο.
Εφόσον το $x^*$ είναι τοπικό ελάχιστο, υπάρχει $\delta > 0$ τέτοιο ώστε $f(x^*) \leq f(x)$ για κάθε $x \in B(x^*, \delta)$.

Έστω $\epsilon > 0$ τέτοιο ώστε $\delta > \epsilon$.
Από το λήμμα 2.3.1 του βιβλίου (σελ. 15), ισχύει ότι για ένα διάστημα $(x^*, x^*+\epsilon p)$ υπάρχει $z \in (x^*, x^*+\epsilon p)$ τέτοιο ώστε:
\[
f(x^*+\epsilon p) = f(x^*) + \epsilon \nabla f^T(z) p
\]
Επειδή το $x^*$ είναι τοπικό ελάχιστο, ισχύει $f(x^*+\epsilon p) - f(x^*) \geq 0$.
Συνεπώς:
\[
\epsilon \nabla f^T(z) p \geq 0
\]
Για $\epsilon \to 0$, έχουμε $z \to x^*$, άρα:
\[
\nabla f^T(x^*) p \geq 0
\]

Αν ακολουθήσουμε την ίδια στρατηγική για το διάστημα $(x^*, x^*-\epsilon p)$, θα δείξουμε αντίστοιχα ότι:
\[ -\epsilon \nabla f^T(z') p \geq 0 \implies \nabla f^T(z') p \leq 0 \]
Για $\epsilon \to 0$, προκύπτει $\nabla f^T(x^*) p \leq 0$.

Από τις δύο ανισότητες συμπεραίνουμε ότι $\nabla f^T(x^*) p = 0$.
Καθώς το διάνυσμα $p$ είναι τυχαίο, έπεται ότι:
\[
\boxed{\nabla f(x^*) = 0}
\]
\end{answer}

\begin{question}[Θέμα 2 (4 μονάδες)]
Δίνεται η συνεχώς διαφορίσιμη συνάρτηση $h: \mathbb{R}^n \to \mathbb{R}$ και δύο σταθερά σημεία $x_1, x_2 \in \mathbb{R}^n$. Ζητείται να βρεθεί σημείο $x$ πάνω στην επιφάνεια $S = \{x \in \mathbb{R}^n \mid h(x) = 0\}$ που ελαχιστοποιεί το άθροισμα των ευκλείδειων αποστάσεων από τα $x_1$ και $x_2$.

\textbf{α) (2 μονάδες)} Να διατυπώσετε μαθηματικά το πρόβλημα βελτιστοποίησης.
\textbf{β) (2 μονάδες)} Να γράψετε τις αναγκαίες συνθήκες βελτιστότητας (Karush-Kuhn-Tucker) και να δείξετε ότι στο βέλτιστο σημείο $x^*$, τα διανύσματα διεύθυνσης από το $x^*$ προς τα $x_1, x_2$ σχηματίζουν ίσες γωνίες με την κάθετο της επιφάνειας (Νόμος της Ανάκλασης).
\end{question}

\begin{answer}
\textbf{Λύση:}

Το πρόβλημα είναι:
\[
\min_{h(x)=0} (|x_1-x| + |x_2-x|)
\]
όπου $|y| = (y^T y)^{1/2}$ είναι η Ευκλείδεια νόρμα.

\textbf{α) Συνθήκες KKT}
Για προβλήματα με περιορισμούς ισότητας, οι συνθήκες KKT είναι:
\[
\nabla_x (|x_1-x|) + \nabla_x (|x_2-x|) + \lambda \nabla h(x^*) = 0
\]
Γνωρίζουμε ότι η κλίση της απόστασης είναι:
\[
\nabla (|x_i-x|) = \frac{-(x_i-x)}{|x_i-x|}
\]
Άρα η εξίσωση γίνεται:
\[
\frac{-(x_1-x^*)}{|x_1-x^*|} + \frac{-(x_2-x^*)}{|x_2-x^*|} + \lambda \nabla h(x^*) = 0 \quad (1)
\]

\textbf{β) Σχέση γωνιών}
Πολλαπλασιάζουμε την (1) εσωτερικά με το μοναδιαίο διάνυσμα $\frac{(x_1-x^*)^T}{|x_1-x^*|}$:
\[
\frac{-(x_1-x^*)^T(x_1-x^*)}{|x_1-x^*|^2} + \frac{-(x_1-x^*)^T(x_2-x^*)}{|x_1-x^*||x_2-x^*|} + \lambda \frac{(x_1-x^*)^T \nabla h(x^*)}{|x_1-x^*|} = 0
\]
\[
-1 - \cos\theta + \lambda \frac{(x_1-x^*)^T \nabla h(x^*)}{|x_1-x^*|} = 0
\]
\[
\implies \lambda \frac{(x_1-x^*)\nabla h(x^*)}{|x_1-x^*|} = 1 + \frac{(x_1-x^*)^T(x_2-x^*)}{|x_1-x^*||x_2-x^*|} \quad (2)
\]

Πολλαπλασιάζουμε την (1) εσωτερικά με το μοναδιαίο διάνυσμα $\frac{(x_2-x^*)^T}{|x_2-x^*|}$:
\[
\frac{-(x_2-x^*)^T(x_1-x^*)}{|x_2-x^*||x_1-x^*|} + \frac{-(x_2-x^*)^T(x_2-x^*)}{|x_2-x^*|^2} + \lambda \frac{(x_2-x^*)^T \nabla h(x^*)}{|x_2-x^*|} = 0
\]
\[
\implies \lambda \frac{(x_2-x^*)\nabla h(x^*)}{|x_2-x^*|} = 1 + \frac{(x_2-x^*)^T(x_1-x^*)}{|x_2-x^*||x_1-x^*|} \quad (3)
\]

Παρατηρούμε ότι το δεξί μέλος των (2) και (3) είναι ίσο, διότι $(x_2-x^*)^T(x_1-x^*) = (x_1-x^*)^T(x_2-x^*)$ (εσωτερικό γινόμενο στον $\mathbb{R}^n$).
Επομένως:
\[
\frac{(x_2-x^*)^T \nabla h(x^*)}{|x_2-x^*|} = \frac{(x_1-x^*)^T \nabla h(x^*)}{|x_1-x^*|}
\]

Διαιρώντας και τα δύο μέλη με $|\nabla h(x^*)|$ (υποθέτοντας ότι $\nabla h(x^*) \neq 0$):
\[
\boxed{\frac{(x_1-x^*)^T \nabla h(x^*)}{|x_1-x^*||\nabla h(x^*)|} = \frac{(x_2-x^*)^T \nabla h(x^*)}{|x_2-x^*||\nabla h(x^*)|}}
\]
Αυτό σημαίνει ότι οι γωνίες που σχηματίζουν τα διανύσματα από το $x^*$ προς τα $x_1, x_2$ με το εφαπτόμενο επίπεδο (ή το κάθετο διάνυσμα $\nabla h$) είναι ίσες (Νόμος της Ανάκλασης).
\end{answer}

\begin{question}[Θέμα 3 (3 μονάδες)]
Επιθυμούμε να ελαχιστοποιήσουμε τη συνάρτηση:
\[
f(x_1, x_2) = \frac{1}{2}(x_1^2 - x_2^2) - 3x_2
\]
υπό τον περιορισμό:
\[
-x_1 - 2x_2 = 6
\]

\textbf{α) (1.5 μονάδες)} Να επιλύσετε αναλυτικά το πρόβλημα.

\textbf{β) (1.5 μονάδες)} Να επιλύσετε το πρόβλημα χρησιμοποιώντας τη μέθοδο ποινής.
\end{question}

\begin{answer}
\textbf{Λύση:}

\textbf{α) Αναλυτική Λύση με KKT}
Η συνάρτηση είναι $f(x_1, x_2) = \frac{1}{2}(x_1^2 - x_2^2) - 3x_2$ και ο περιορισμός ισότητας $g(x_1, x_2) = -x_1 - 2x_2 - 6 = 0$.
Εφόσον έχουμε μόνο ισότιμους περιορισμούς, η προϋπόθεση κυρτότητας της $f$ δεν είναι απαραίτητη για την εφαρμογή των συνθηκών KKT (Βιβλίο σελ. 42-47).

Οι συνθήκες KKT είναι:
\[ \nabla f(x^*) + \lambda \nabla g(x^*) = 0 \]
Υπολογίζουμε τις κλίσεις:
\[ \nabla f(x_1, x_2) = \begin{bmatrix} x_1 \\ -x_2 - 3 \end{bmatrix}, \quad \nabla g(x_1, x_2) = \begin{bmatrix} -1 \\ -2 \end{bmatrix} \]
Άρα το σύστημα εξισώσεων είναι:
\begin{align}
x_1 - \lambda &= 0 \implies x_1 = \lambda \quad (1) \\
-x_2 - 3 - 2\lambda &= 0 \implies x_2 = -2\lambda - 3 \quad (2) \\
-x_1 - 2x_2 &= 6 \quad (3)
\end{align}
Αντικαθιστούμε τις (1) και (2) στην (3):
\[
-\lambda - 2(-2\lambda - 3) = 6 \implies -\lambda + 4\lambda + 6 = 6 \implies 3\lambda = 0 \implies \lambda = 0
\]
Για $\lambda = 0$:
\[
x_1 = 0, \quad x_2 = -3
\]
Άρα η λύση είναι $\boxed{x^* = (0, -3)}$.

\textbf{β) Μέθοδος Ποινής (Penalty Method)}
Χρησιμοποιούμε την Τετραγωνική Συνάρτηση Ποινής:
\[ P(x, \mu) = f(x) + \frac{\mu}{2} (g(x))^2 = \frac{1}{2}(x_1^2 - x_2^2) - 3x_2 + \frac{\mu}{2}(-x_1 - 2x_2 - 6)^2 \]
όπου $\mu > 0$ η παράμετρος ποινής.
Για να βρούμε το ελάχιστο της $P(x, \mu)$ για δεδομένο $\mu$, μηδενίζουμε τις μερικές παραγώγους:

\begin{align*}
\frac{\partial P}{\partial x_1} &= x_1 + \mu(-x_1 - 2x_2 - 6)(-1) = x_1 + \mu(x_1 + 2x_2 + 6) = 0 \\
&\implies x_1(1+\mu) + 2\mu x_2 + 6\mu = 0 \quad (1) \\
\frac{\partial P}{\partial x_2} &= -x_2 - 3 + \mu(-x_1 - 2x_2 - 6)(-2) = -x_2 - 3 + 2\mu(x_1 + 2x_2 + 6) = 0 \\
&\implies 2\mu x_1 + (4\mu - 1)x_2 + 12\mu - 3 = 0 \quad (2)
\end{align*}

Επιλύουμε το σύστημα των (1) και (2) για τα $x_1, x_2$.
Από την (1): $x_1 = -\frac{2\mu x_2 + 6\mu}{1+\mu}$.
Για $\mu \to \infty$, το κλάσμα $\frac{\mu}{1+\mu} \to 1$. Άρα για μεγάλα $\mu$:
\[ x_1 \approx -(2x_2 + 6) = -2x_2 - 6 \quad (3) \]
Αντικαθιστούμε την (3) στην (2):
\[ 2\mu(-2x_2 - 6) + 4\mu x_2 - x_2 + 12\mu - 3 = 0 \]
\[ -4\mu x_2 - 12\mu + 4\mu x_2 - x_2 + 12\mu - 3 = 0 \]
\[ -x_2 - 3 = 0 \implies \boxed{x_2 = -3} \]
Και από την (3):
\[ x_1 = -2(-3) - 6 = 6 - 6 \implies \boxed{x_1 = 0} \]

Άρα, το όριο της ακολουθίας των σημείων ελαχίστου της $P(x, \mu)$ καθώς $\mu \to \infty$ είναι το σημείο $x^* = (0, -3)$, που ταυτίζεται με την αναλυτική λύση.

\vspace{1em}
\hrule
\vspace{0.5em}

\textbf{Σημείωση Θεωρίας (Μέθοδος Ποινής vs Επαυξημένη Λαγκρανζιανή):}
Γενικά, για την επίλυση τέτοιων προβλημάτων μπορούμε να χρησιμοποιήσουμε και τις δύο μεθόδους. Ωστόσο:
\begin{itemize}
    \item Η \textbf{Μέθοδος της Επαυξημένης Λαγκρανζιανής} σχεδιάστηκε αρχικά μόνο για \textbf{ισοτικούς περιορισμούς}. Δεν περιλαμβάνει απευθείας όρο για ανισοτικούς περιορισμούς. Αν υπάρχουν τέτοιοι, πρέπει θεωρητικά να μετατραπούν σε ισοτικούς (π.χ. προσθέτοντας μεταβλητές "slack").
    \item Η \textbf{Μέθοδος της Ποινής} διαθέτει όρους τόσο για ανισοτικούς όσο και για ισοτικούς περιορισμούς. Όταν έχουμε \textbf{μόνο ισοτικούς περιορισμούς} (όπως στη συγκεκριμένη άσκηση), απλά χρησιμοποιούμε τον όρο των ισοτικών και παραλείπουμε τον όρο που αφορά τους ανισοτικούς.
\end{itemize}
\end{answer}




% ============== Φεβρουάριος 2025 ==============
\newpage
\phantomsection
\hypertarget{fevrouarios2025}{}
\pdfbookmark[1]{Φεβρουάριος 2025}{fevrouarios2025}
\examyear{Φεβρουάριος 2025}

\begin{center}
\textbf{\Large Φεβρουάριος 2025 -- Θέματα \& Λύσεις}
\end{center}

\begin{question}[Θέμα 1 (3 μονάδες)]
Έστω $f: \mathbb{R}^n \to \mathbb{R}$ μια συνεχώς διαφορίσιμη συνάρτηση. Στους αλγορίθμους τοπικής αναζήτησης του ελαχίστου της $f$ η διαδικασία τερματίζει στο σημείο $x_*$ για το οποίο:
\[
\nabla f(x_*) = 0.
\]

Με δεδομένο το παραπάνω, θεωρείστε τη συνάρτηση:
\[
f(x_1, x_2) = (x_1 - x_2)^3 + (x_1 + x_2)^3
\]
και το πρόβλημα:
\[
\min_{x_1, x_2} f(x_1, x_2).
\]
Έστω το σημείο $x^* = [x_1^*\ x_2^*]^T$ για το οποίο
\[
\nabla f(x^*) = 0.
\]

Να δείξετε ότι για τη δοσμένη συνάρτηση $f$, υπάρχουν άπειρα σημεία $z$, αρκούντως κοντά στο $x^*$ για τα οποία:
\[
f(z) < f(x^*)
\]
και επομένως το $x^*$ δεν είναι τοπικό ελάχιστο της $f$.
\end{question}

\begin{answer}
\textbf{Λύση:}

\textbf{Βήμα 1: Εύρεση στάσιμων σημείων}
\[
\nabla f(x) = \begin{bmatrix} 3(x_1-x_2)^2 + 3(x_1+x_2)^2 \\ -3(x_1-x_2)^2 + 3(x_1+x_2)^2 \end{bmatrix}
\]
Εξισώνοντας με το μηδέν $\nabla f(x^*) = 0$:
\begin{align*}
3(x_1-x_2)^2 + 3(x_1+x_2)^2 &= 0 \quad (1)\\
-3(x_1-x_2)^2 + 3(x_1+x_2)^2 &= 0 \quad (2)
\end{align*}
Αθροίζοντας τις (1) και (2) έχουμε $6(x_1+x_2)^2 = 0 \implies x_1 = -x_2$.
Αφαιρώντας τες έχουμε $6(x_1-x_2)^2 = 0 \implies x_1 = x_2$.
Συνεπώς, το μοναδικό κρίσιμο σημείο είναι το $x^* = (0, 0)$.

\textbf{Βήμα 2: Έλεγχος Βελτιστότητας (2ης Τάξης)}
Υπολογίζουμε τον πίνακα Hessian:
\[
\nabla^2 f(x) = \begin{bmatrix} 6(x_1-x_2)+6(x_1+x_2) & -6(x_1-x_2)+6(x_1+x_2) \\ -6(x_1-x_2)+6(x_1+x_2) & 6(x_1-x_2)+6(x_1+x_2) \end{bmatrix}
\]
Στο σημείο $x^*=(0,0)$, έχουμε $\nabla^2 f(0,0) = \mathbb{O}$ (μηδενικός πίνακας).
Επειδή ο Hessian είναι μηδενικός (και άρα έχει ιδιοτιμές 0), το κριτήριο 2ης τάξης είναι \textbf{ατελέσφορο}. Δεν μπορούμε να αποφανθούμε αν είναι ελάχιστο, μέγιστο ή σαγματικό σημείο μόνο από τις παραγώγους.
\textbf{Υποχρεωτικά}, καταφεύγουμε στον ορισμό του τοπικού ελαχίστου.

\textbf{Βήμα 3: Απόδειξη μέσω Ορισμού}
Για να είναι το $x^*$ τοπικό ελάχιστο, πρέπει να ισχύει $f(z) \ge f(x^*)$ για όλα τα $z$ σε μια περιοχή του $x^*$.
Εδώ $f(x^*) = f(0,0) = 0$.

Εξετάζουμε τη συμπεριφορά της συνάρτησης κατά μήκος του άξονα $x_1$ ($x_2=0$).
Έστω σημείο δοκιμής $z = (-\epsilon, 0)$ με $\epsilon > 0$ αρκούντως μικρό.
\[
f(-\epsilon, 0) = (-\epsilon - 0)^3 + (-\epsilon + 0)^3 = -\epsilon^3 - \epsilon^3 = -2\epsilon^3
\]
Για κάθε $\epsilon > 0$, ισχύει $-2\epsilon^3 < 0 \implies f(z) < f(x^*)$.

Άρα, σε οποιαδήποτε γειτονιά του $(0,0)$, υπάρχουν σημεία με τιμή μικρότερη του $f(x^*)$.
Επομένως, το $x^* = (0,0)$ \textbf{δεν είναι τοπικό ελάχιστο}.
\end{answer}

\begin{question}[Θέμα 2 (7 μονάδες)]
Θέλουμε να ελαχιστοποιήσουμε τη συνάρτηση
\[
f(x_1, x_2) = x_1^4 + x_2^2 + x_1^2 x_2 - 2x_2
\]
ως προς $x_1$, $x_2$ με περιορισμό:
\[
x_1 + x_2 = 2.
\]

\textbf{α) (3.5 μονάδες)} Να επιλύσετε αναλυτικά το πρόβλημα.

\textbf{β) (3.5 μονάδες)} Να επιλύσετε το πρόβλημα χρησιμοποιώντας τη μέθοδο ποινής.
\end{question}

\begin{answer}
\textbf{Λύση:}

\textbf{α) Αναλυτική λύση με Πολλαπλασιαστές Lagrange}
\begin{tcolorbox}[colback=yellow!10!white, colframe=yellow!50!black, title=Lagrange vs KKT]
\textbf{Γιατί Lagrange και όχι KKT;} 
Οι συνθήκες KKT είναι μια γενίκευση της μεθόδου Lagrange που καλύπτει και ανισοτικούς περιορισμούς. Όταν το πρόβλημα έχει \textbf{μόνο περιορισμούς ισότητας} (όπως εδώ το $x_1+x_2=2$), οι συνθήκες KKT ταυτίζονται απόλυτα με τη μέθοδο των Πολλαπλασιαστών Lagrange. Γι' αυτό χρησιμοποιούμε τον όρο "Lagrange" για συντομία, αν και τεχνικά λύνουμε τις συνθήκες KKT για ισοτικούς περιορισμούς.
\end{tcolorbox}
Το πρόβλημα είναι: $\min f(x)$ υπό τον περιορισμό $h(x) = x_1 + x_2 - 2 = 0$.
Σχηματίζουμε τη συνάρτηση Lagrangian:
\[ L(x_1, x_2, \lambda) = x_1^4 + x_2^2 + x_1^2 x_2 - 2x_2 + \lambda(x_1 + x_2 - 2) \]

Οι αναγκαίες συνθήκες πρώτης τάξης (KKT) είναι $\nabla L = 0$:
\begin{align}
\frac{\partial L}{\partial x_1} &= 4x_1^3 + 2x_1 x_2 + \lambda = 0 \implies \lambda = -(4x_1^3 + 2x_1 x_2) \quad (1) \\
\frac{\partial L}{\partial x_2} &= 2x_2 + x_1^2 - 2 + \lambda = 0 \implies \lambda = -(2x_2 + x_1^2 - 2) \quad (2) \\
\frac{\partial L}{\partial \lambda} &= x_1 + x_2 - 2 = 0 \implies x_2 = 2 - x_1 \quad (3)
\end{align}

Εξισώνοντας τις εκφράσεις για το $\lambda$ από τις (1) και (2):
\[ 4x_1^3 + 2x_1 x_2 = 2x_2 + x_1^2 - 2 \]
Αντικαθιστούμε το $x_2$ από την (3):
\[ 4x_1^3 + 2x_1(2-x_1) = 2(2-x_1) + x_1^2 - 2 \]
\[ 4x_1^3 + 4x_1 - 2x_1^2 = 4 - 2x_1 + x_1^2 - 2 \]
\[ 4x_1^3 - 3x_1^2 + 6x_1 - 2 = 0 \]

Επιλύοντας την κυβική εξίσωση (προσεγγιστικά):
\[ \boxed{x_1 \approx 0.387} \]
Και από την (3):
\[ x_2 = 2 - 0.387 \implies \boxed{x_2 \approx 1.613} \]

\textbf{β) Μέθοδος Ποινής (Penalty Method)}
Χρησιμοποιούμε την Τετραγωνική Συνάρτηση Ποινής (Quadratic Penalty Function):
\[ P(x, \mu) = f(x) + \frac{\mu}{2} (h(x))^2 = x_1^4 + x_2^2 + x_1^2 x_2 - 2x_2 + \frac{\mu}{2}(x_1 + x_2 - 2)^2 \]
όπου $\mu > 0$ η παράμετρος ποινής. Καθώς $\mu \to \infty$, η ελαχιστοποίηση της $P(x, \mu)$ αναγκάζει τον όρο $(x_1+x_2-2)^2$ να μηδενιστεί, οδηγώντας στη λύση του δεσμευμένου προβλήματος.

Για κάθε δεδομένο $\mu$, βρίσκουμε το $x(\mu)$ που μηδενίζει την κλίση $\nabla_x P(x, \mu) = 0$:
\begin{align*}
\frac{\partial P}{\partial x_1} &= 4x_1^3 + 2x_1 x_2 + \mu(x_1 + x_2 - 2) = 0 \\
\frac{\partial P}{\partial x_2} &= 2x_2 + x_1^2 - 2 + \mu(x_1 + x_2 - 2) = 0
\end{align*}

Από τις δύο εξισώσεις, απομονώνουμε τον όρο ποινής $\mu(x_1 + x_2 - 2)$:
\[ -(4x_1^3 + 2x_1 x_2) = \mu(x_1 + x_2 - 2) = -(2x_2 + x_1^2 - 2) \]
Αυτό οδηγεί στην ίδια σχέση που βρήκαμε με τη μέθοδο Lagrange:
\[ 4x_1^3 + 2x_1 x_2 = 2x_2 + x_1^2 - 2 \]

Στο όριο $\mu \to \infty$, η λύση πρέπει να ικανοποιεί τον περιορισμό $x_1 + x_2 - 2 = 0$.
Συνδυάζοντας τη σχέση αυτή με τον περιορισμό, καταλήγουμε πάλι στην κυβική εξίσωση:
\[ 4x_1^3 - 3x_1^2 + 6x_1 - 2 = 0 \]
Η οποία δίνει την ίδια φορά λύση:
\[ \boxed{x_1 \approx 0.387, \quad x_2 \approx 1.613} \]
\end{answer}




% ============== Ιούλιος 2025 ==============
\newpage
\phantomsection
\hypertarget{ioulios2025}{}
\pdfbookmark[1]{Ιούλιος 2025}{ioulios2025}
\examyear{Ιούλιος 2025}

\begin{center}
\textbf{\Large Ιούλιος 2025 -- Θέματα \& Λύσεις}
\end{center}

\begin{question}[Θέμα 1 (5 μονάδες)]
\textbf{α) (2.5 μονάδες)} Για την ελαχιστοποίηση της συνάρτησης
\[
f(x, y) = x^2 + y^2 - 2x - 4y + 5
\]
ως προς $x$, $y$ εκκινώντας από οποιοδήποτε σημείο, έχετε στη διάθεσή σας τους αλγορίθμους: α) μέγιστης κλίσης, β) Newton, γ) Levenberg-Marquardt. Ποιον από τους τρεις θα επιλέξετε, αν σας ενδιαφέρει η εύρεση του ελαχίστου να πραγματοποιηθεί μ' όσο το δυνατόν λιγότερες επαναλήψεις, εκτελώντας σε κάθε επανάληψη τις λιγότερες δυνατές πράξεις; Να αιτιολογήσετε την απάντησή σας.

\textbf{β) (2.5 μονάδες)} Δίνεται το πρόβλημα:
\[
\min_{x,y} \left( x^2 + y^2 - 2x - 4y + 5 \right)
\]
\begin{align*}
-1 \leq x \leq 1 \\
-1 \leq y \leq 1
\end{align*}

Μπορεί να χρησιμοποιηθεί μέθοδος προβολής για να λυθεί; Να αιτιολογήσετε την απάντησή σας.
\end{question}

\begin{answer}
\textbf{Λύση:}

$f(x, y) = x^2 + y^2 - 2x - 4y + 5 = (x-1)^2 + (y-2)^2$

\textbf{α) Επιλογή αλγορίθμου}
Η συνάρτηση είναι τετραγωνική με Hessian $H = 2I$.
Αυτό προκύπτει διότι:
\[
\nabla f(x,y) = \begin{bmatrix} \frac{\partial f}{\partial x} \\ \frac{\partial f}{\partial y} \end{bmatrix} = \begin{bmatrix} 2x-2 \\ 2y-4 \end{bmatrix} \implies H_f = \begin{bmatrix} \frac{\partial^2 f}{\partial x^2} & \frac{\partial^2 f}{\partial x \partial y} \\ \frac{\partial^2 f}{\partial y \partial x} & \frac{\partial^2 f}{\partial y^2} \end{bmatrix} = \begin{bmatrix} 2 & 0 \\ 0 & 2 \end{bmatrix} = 2I
\]
\begin{itemize}
    \item \textbf{Newton}: Συγκλίνει σε \textbf{1 βήμα} για τετραγωνικές συναρτήσεις.
    \item Μέγιστης κλίσης: Πολλά βήματα.
    \item Levenberg-Marquardt: Για μη-γραμμικά ελάχιστα τετραγώνων.
\end{itemize}
\textbf{Απάντηση}: Μέθοδος \textbf{Newton}, γιατί για τετραγωνική $f$ συγκλίνει σε ακριβώς 1 επανάληψη (Θεώρημα 5.2.4).

\textbf{β) Μέθοδος προβολής και Αιτιολόγηση}
Για να χρησιμοποιηθεί μια μέθοδος προβολής της κλίσης (Gradient Projection) με εγγυημένη σύγκλιση στο ολικό ελάχιστο, πρέπει να ικανοποιούνται δύο βασικές συνθήκες:
\begin{enumerate}
    \item \textbf{Κυρτότητα Συνάρτησης:} Η συνάρτηση $f(x,y)$ πρέπει να είναι κυρτή. Όπως δείξαμε στο (α), ο πίνακας Hessian είναι θετικά ορισμένος ($H=2I$), άρα η $f$ είναι αυστηρά κυρτή.
    \item \textbf{Κυρτότητα Περιορισμών:} Το σύνολο των περιορισμών $\Omega$ πρέπει να είναι κυρτό. Εδώ, το σύνολο $\Omega = [-1, 1] \times [-1, 1]$ είναι ένα τετράγωνο (ορθογώνιο παραλληλόγραμμο), το οποίο είναι εξ ορισμού κυρτό σύνολο.
\end{enumerate}

Επιπλέον, η πρακτική εφαρμοσιμότητα της μεθόδου εξαρτάται από την ευκολία υπολογισμού της προβολής. Στην περίπτωσή μας, οι περιορισμοί ορίζουν ένα ορθογώνιο και η προβολή $P_\Omega(x,y)$ ενός σημείου στο σύνολο αυτό αντιστοιχεί στο πλησιέστερο σημείο του ορθογωνίου (γεωμετρικά προφανής υπολογισμός).
Το ίδιο ισχύει και για το $y$.

\textbf{Συμπέρασμα:} Ναι, η μέθοδος προβολής είναι ιδανική για αυτό το πρόβλημα, καθώς πληρούνται οι συνθήκες κυρτότητας και η προβολή υπολογίζεται υπολογιστικά πολύ εύκολα χωρίς να απαιτείται η επίλυση κάποιου υπο-προβλήματος βελτιστοποίησης.
\end{answer}

\begin{question}[Θέμα 2 (5 μονάδες)]
Θέλουμε να ελαχιστοποιήσουμε τη συνάρτηση
\[
f(x, y) = x^2 + 2y^2 + 2xy
\]
ως προς $x$, $y$ με περιορισμούς:
\begin{align*}
-1 \leq x \leq 1 \\
-2 \leq y \leq 1/2
\end{align*}

\textbf{α) (2.5 μονάδες)} Να δείξετε ότι μπορούν να χρησιμοποιηθούν μέθοδοι προβολής για να λυθεί το πρόβλημα.

\textbf{β) (2.5 μονάδες)} Να σχεδιαστεί η μέθοδος Newton με προβολή που να λύνει το πρόβλημα.
\end{question}

\begin{answer}
\textbf{Λύση:}

$f(x, y) = x^2 + 2y^2 + 2xy$ με $-1 \leq x \leq 1$, $-2 \leq y \leq 1/2$.

\textbf{α) Απόδειξη για μέθοδο προβολής}
Η συνάρτηση $f$ είναι τετραγωνική. Η γενική μορφή είναι $f(\mathbf{x}) = \frac{1}{2} \mathbf{x}^T Q \mathbf{x} + \mathbf{c}^T \mathbf{x}$.
Για να βρούμε τον πίνακα $Q$, υπολογίζουμε τον πίνακα Hessian:
\[
\nabla f(x,y) = \begin{bmatrix} 2x + 2y \\ 4y + 2x \end{bmatrix} \implies H_f = \begin{bmatrix} 2 & 2 \\ 2 & 4 \end{bmatrix}
\]
Άρα $Q = H_f = \begin{bmatrix} 2 & 2 \\ 2 & 4 \end{bmatrix}$.

Ιδιοτιμές: $\det(Q - \lambda I) = (2-\lambda)(4-\lambda) - 4 = \lambda^2 - 6\lambda + 4 = 0$
$\lambda = 3 \pm \sqrt{5} > 0$. Άρα $Q$ θετικά ορισμένος $\implies$ $f$ αυστηρά κυρτή.
Το σύνολο περιορισμών είναι ορθογώνιο (κυρτό).
\textbf{Άρα} μπορούν να χρησιμοποιηθούν μέθοδοι προβολής.

\textbf{β) Μέθοδος Newton με προβολή}
\[
\nabla f = \begin{bmatrix} 2x + 2y \\ 4y + 2x \end{bmatrix}, \quad H = \begin{bmatrix} 2 & 2 \\ 2 & 4 \end{bmatrix}
\]
Αλγόριθμος:
\begin{enumerate}
    \item Υπολογισμός κατεύθυνσης Newton: $d_k = -H^{-1} \nabla f(x_k)$
    \item Βήμα Newton: $\tilde{x}_{k+1} = x_k + d_k$
    \item Προβολή: $x_{k+1} = \text{proj}_\Omega(\tilde{x}_{k+1})$
\end{enumerate}
Όπου $\text{proj}_\Omega(x,y)$ είναι η ευκλείδεια προβολή του σημείου στο ορθογώνιο $\Omega$.

$H^{-1} = \frac{1}{4} \begin{bmatrix} 4 & -2 \\ -2 & 2 \end{bmatrix} = \begin{bmatrix} 1 & -1/2 \\ -1/2 & 1/2 \end{bmatrix}$

Εκτέλεση Αλγορίθμου:
\textbf{Βήμα 1 (Ελαχιστοποίηση χωρίς περιορισμούς):}
Λύνουμε το σύστημα $\nabla f(x,y) = 0$ για να βρούμε το σημείο που θα κατέληγε η μέθοδος Newton σε ένα βήμα, αγνοώντας τους περιορισμούς:
\[ \begin{cases} 2x+2y=0 \\ 4y+2x=0 \end{cases} \implies (x,y) = (0,0) \]

\textbf{Βήμα 2 (Έλεγχος και Προβολή):}
Ελέγχουμε αν το σημείο $(0,0)$ ανήκει στο σύνολο περιορισμών $\Omega = [-1,1] \times [-2, 1/2]$.
Ισχύει ότι $-1 \le 0 \le 1$ και $-2 \le 0 \le 1/2$.
Άρα το σημείο είναι εφικτό: $(0,0) \in \Omega$.

Επομένως, $x_{new} = \text{proj}_\Omega(0,0) = (0,0)$.
Το σημείο $(0,0)$ είναι το ολικό ελάχιστο του προβλήματος με τιμή $f(0,0)=0$.

\begin{center}
\begin{tikzpicture}[scale=1.5]
    % Grid
    \draw[help lines, color=gray!20, dashed] (-1.9,-2.9) grid (1.9,1.4);

    % Constraints Box (drawn first, with opacity)
    \fill[cyan, opacity=0.15] (-1,-2) rectangle (1,0.5);
    \draw[thick, blue] (-1,-2) rectangle (1,0.5);
    \node[blue, anchor=south east] at (1,-2) {$\Omega$};

    % Axes (drawn after box to be visible)
    \draw[->, thick] (-1.6,0) -- (1.6,0) node[right] {$x$};
    \draw[->, thick] (0,-2.6) -- (0,1.2) node[above] {$y$};

    % Optimal Point
    \filldraw[red] (0,0) circle (2pt) node[anchor=south west] {$x^*=(0,0)$};

    % Labels
    \node[below, font=\scriptsize] at (1,0) {1};
    \node[below, font=\scriptsize] at (-1,0) {-1};
    \node[left, font=\scriptsize] at (0,0.5) {0.5};
    \node[left, font=\scriptsize] at (0,-2) {-2};
    \node[below left, font=\tiny] at (0,0) {0};
\end{tikzpicture}
\end{center}
\end{answer}









\newpage
% ============== Σεπτέμβριος 2025 ==============
\phantomsection
\hypertarget{septembrios2025}{}
\pdfbookmark[1]{Σεπτέμβριος 2025}{septembrios2025}
\examyear{Σεπτέμβριος 2025}

\begin{center}
\textbf{\Large Σεπτέμβριος 2025 -- Θέματα \& Λύσεις}
\end{center}

\begin{question}[Θέμα 1 (2 μονάδες)]
Δίνεται η συνάρτηση $f: \mathbb{R} \to \mathbb{R}$ με τύπο
\[ f(x) = \begin{cases} x^2, & 0 \leq x \leq 1 \\ x+1, & x > 1 \end{cases} \]
Να διερευνηθεί αν είναι κυρτή.
\end{question}

\begin{answer}
\textbf{Λύση:}

Για να είναι η $f$ κυρτή στο πεδίο ορισμού της, πρέπει για κάθε $x_α, x_β \in \text{Dom}(f)$ και για κάθε $\lambda \in [0, 1]$ να ισχύει:
\[ f(\lambda x_α + (1-\lambda)x_β) \leq \lambda f(x_α) + (1-\lambda)f(x_β) \]

Παρατηρούμε τη γραφική παράσταση της συνάρτησης.
Στο διάστημα $[0, 1]$, $f(x) = x^2$ (παραβολή που καταλήγει στο $(1, 1)$).
Στο διάστημα $(1, +\infty)$, $f(x) = x+1$ (ευθεία που ξεκινάει από το $(1, 2)$).
\textit{Παρατήρηση:} Υπάρχει ασυνέχεια (άλμα) στο $x=1$, καθώς $\lim_{x \to 1^-} f(x) = 1$ ενώ $\lim_{x \to 1^+} f(x) = 2$.

Θα δοκιμάσουμε να βρούμε ένα αντιπαράδειγμα για τον ορισμό της κυρτότητας.
Έστω τα σημεία $x_a = 1$ και $x_b = 2$.
\begin{itemize}
    \item $f(x_α) = f(1) = 1^2 = 1$.
    \item $f(x_β) = f(2) = 2 + 1 = 3$.
\end{itemize}
Επιλέγουμε $\lambda = 0.5$ (το μέσο το ευθύγραμμου τμήματος που ενώνει τα $(1,1)$ και $(2,3)$).
Το σημείο είναι $x_{mid} = 0.5(1) + 0.5(2) = 1.5$.
Η τιμή της συνάρτησης στο $1.5$ είναι:
\[ f(1.5) = 1.5 + 1 = 2.5 \]
Η τιμή της χορδής (ευθύγραμμου τμήματος) στο μέσο είναι:
\[ 0.5 f(1) + 0.5 f(2) = 0.5(1) + 0.5(3) = 0.5 + 1.5 = 2 \]

Παρατηρούμε ότι:
\[ f(1.5) = 2.5 > 2 = 0.5 f(1) + 0.5 f(2) \]
Δηλαδή $f(\lambda x_α + (1-\lambda)x_β) > \lambda f(x_α) + (1-\lambda)f(x_β)$.
Η ανισότητα της κυρτότητας \textbf{δεν} ικανοποιείται.

\textbf{Συμπέρασμα:} Η συνάρτηση $f$ \textbf{δεν είναι κυρτή}.

\begin{tcolorbox}[colback=white, colframe=black, title=Εναλλακτικός Τρόπος (Άμεσος)]
Η συνάρτηση $f$ παρουσιάζει ασυνέχεια στο σημείο $x=1$ (το οποίο είναι εσωτερικό σημείο του πεδίου ορισμού), καθώς:
\[ \lim_{x \to 1^-} f(x) = 1 \neq \lim_{x \to 1^+} f(x) = 2 \]
Γνωρίζουμε από τη θεωρία ότι κάθε κυρτή συνάρτηση ορισμένη σε ανοικτό διάστημα (ή γενικότερα σε κυρτό σύνολο) οφείλει να είναι \textbf{συνεχής} στο εσωτερικό του. Συνεπώς, η ύπαρξη ασυνέχειας αρκεί για να συμπεράνουμε άμεσα ότι η συνάρτηση \textbf{δεν} είναι κυρτή.
\end{tcolorbox}
\end{answer}

\begin{question}[Θέμα 2 (4 μονάδες)]
Εργοστάσιο παράγει ημερησίως τα προϊόντα Α και Β σε ποσότητες $x_1, x_2$ αντίστοιχα. Η ημερήσια παραγωγή του Α δεν μπορεί να ξεπεράσει τους 4 τόνους, ενώ του Β τους $\sqrt{25 - x_1^2}$ τόνους. Το κάθε προϊόν Α που παράγεται δίνει κέρδος 9 χρηματικές μονάδες και του Β δίνει κέρδος 5 χρηματικές μονάδες. Επιθυμούμε να προσδιορίσουμε τον καταμερισμό της ημερήσιας παραγωγής του εργοστασίου ώστε να βελτιστοποιείται το κέρδος.

\textbf{α) (2 μονάδες)} Να διατυπωθεί ως πρόβλημα ελαχιστοποίησης.

\textbf{β) (2 μονάδες)} Χρησιμοποιώντας τις συνθήκες Karush-Kuhn-Tucker να λυθεί, αν είναι εφικτό, αναλυτικά το πρόβλημα του ερωτήματος (α).
\end{question}

\begin{answer}
\textbf{Λύση:}

\textbf{α) Μαθηματική Διατύπωση}
Έστω $x_1, x_2$ οι ποσότητες παραγωγής για τα προϊόντα Α και Β αντίστοιχα.
Ο στόχος είναι η μεγιστοποίηση του κέρδους:
\[ \max P(x_1, x_2) = 9x_1 + 5x_2 \]
Περιορισμοί:
\begin{enumerate}
    \item $x_1 \leq 4$
    \item $x_2 \leq \sqrt{25 - x_1^2} \implies x_2^2 \leq 25 - x_1^2 \implies x_1^2 + x_2^2 \leq 25$ (παραδοχή $x_2 \geq 0$)
    \item Φυσικοί περιορισμοί: $x_1 \geq 0, x_2 \geq 0$.
\end{enumerate}

Για ελαχιστοποίηση, αντιστρέφουμε το πρόσημο της αντικειμενικής συνάρτησης και γράφουμε τους περιορισμούς στη μορφή $g(x) \leq 0$.
\[ \min f(x_1, x_2) = -9x_1 - 5x_2 \]
Υπό τους περιορισμούς:
\begin{align*}
g_1(x) &= x_1 - 4 \leq 0 \\
g_2(x) &= x_1^2 + x_2^2 - 25 \leq 0 \\
-x_1 &\leq 0, -x_2 \leq 0
\end{align*}

\textbf{β) Επίλυση με KKT}
Η συνάρτηση Lagrange είναι:
\[ L(x, \lambda) = -9x_1 - 5x_2 + \lambda_1(x_1 - 4) + \lambda_2(x_1^2 + x_2^2 - 25) \]

\textbf{Συνθήκες KKT:}
\begin{enumerate}
    \item \textbf{Στασιμότητα ($\nabla_x L = 0$):}
    \begin{align}
    \frac{\partial L}{\partial x_1} &= -9 + \lambda_1 + 2\lambda_2 x_1 = 0 \implies \lambda_1 + 2\lambda_2 x_1 = 9 \quad (1) \\
    \frac{\partial L}{\partial x_2} &= -5 + 2\lambda_2 x_2 = 0 \implies 2\lambda_2 x_2 = 5 \quad (2)
    \end{align}
    
    \item \textbf{Συμπληρωματικότητα:}
    \begin{align}
    \lambda_1(x_1 - 4) &= 0 \quad (3) \\
    \lambda_2(x_1^2 + x_2^2 - 25) &= 0 \quad (4)
    \end{align}
    
    \item \textbf{Εφικτότητα:}
    \[ x_1 \leq 4, \quad x_1^2 + x_2^2 \leq 25, \quad x \geq 0 \]
    
    \item \textbf{Δυϊκή Εφικτότητα:}
    \[ \lambda_1 \geq 0, \quad \lambda_2 \geq 0 \]
\end{enumerate}

\textbf{Ανάλυση Περιπτώσεων:}
Από την (2): $2\lambda_2 x_2 = 5 \implies \lambda_2 \neq 0$ και $x_2 \neq 0$.
Εφόσον $\lambda_2 > 0$, από την (4) ο περιορισμός είναι ενεργός: $\boxed{x_1^2 + x_2^2 = 25}$.

\textbf{Περίπτωση 1: $\lambda_1 = 0$}
(1) $\implies 2\lambda_2 x_1 = 9 \implies \lambda_2 = \frac{4.5}{x_1}$
(2) $\implies 2\lambda_2 x_2 = 5 \implies \lambda_2 = \frac{2.5}{x_2}$
Εξισώνοντας: $\frac{4.5}{x_1} = \frac{2.5}{x_2} \implies 9x_2 = 5x_1 \implies x_1 = 1.8 x_2$.
Στον κύκλο: $(1.8x_2)^2 + x_2^2 = 25 \implies 3.24x_2^2 + x_2^2 = 25 \implies 4.24x_2^2 = 25 \implies x_2^2 \approx 5.896$.
$x_2 \approx 2.428$ και $x_1 \approx 4.37$.
Όμως $x_1 \leq 4$. Το $4.37 > 4$, άρα \textbf{Απορρίπτεται}.

\textbf{Περίπτωση 2: $\lambda_1 > 0$}
Από (3) $\implies \boxed{x_1 = 4}$.
Στον κύκλο: $16 + x_2^2 = 25 \implies x_2^2 = 9 \implies \boxed{x_2 = 3}$ (αφού $x_2 > 0$).
Έλεγχος πολλαπλασιαστών:
(2) $\implies 2\lambda_2(3) = 5 \implies \lambda_2 = 5/6 > 0$ (ΟΚ).
(1) $\implies \lambda_1 + 2(5/6)(4) = 9 \implies \lambda_1 + 40/6 = 9 \implies \lambda_1 + 20/3 = 27/3 \implies \lambda_1 = 7/3 > 0$ (ΟΚ).

Άρα η βέλτιστη λύση είναι: $\boxed{x_1^* = 4, \quad x_2^* = 3}$.
Μέγιστο Κέρδος: 51.
\end{answer}

\begin{question}[Θέμα 3 (4 μονάδες)]
Να διερευνήσετε αν είναι εφικτή η ελαχιστοποίηση της συνάρτησης $f_1(x_1, x_2)$ ως προς $x = [x_1, x_2]^T \in S_1 \subset \mathbb{R}^2$ και της συνάρτησης $f_2(x_1, x_2)$ ως προς $x = [x_1, x_2]^T \in S_2 \subset \mathbb{R}^2$ όταν:

\textbf{α) (2 μονάδες)} $f_1(x_1, x_2) = x_1^2 + x_2^2 - 3x_1 - 3x_2 + \frac{9}{2}$ και
\[ S_1 = \{x \in \mathbb{R}^2 : x_1 - x_2 \leq 0 \text{ και } x_1 + x_2 - 2 \geq 0 \text{ και } x_1 - x_2 + 2 \leq 0 \text{ και } x_1 - 2 \leq 0 \text{ και } x_2 \geq 0\} \]

\textbf{β) (2 μονάδες)} $f_2(x_1, x_2) = x_1^2 + x_2^2 - 4x_1 + 4$ και
\[ S_2 = \{x \in \mathbb{R}^2 : x_1^2 + x_2^2 - 2x_1 - 4x_2 + 1 \leq 0\} \]
\end{question}

\begin{answer}
\textbf{Λύση:}

Εξετάζουμε την ύπαρξη ελαχίστου με βάση το \textbf{Θεώρημα Weierstrass} (Compacntess) ή μέσω της ιδιότητας \textit{coercivity} (για μη φραγμένα σύνολα).

\textbf{α) Μελέτη για $f_1, S_1$}
\[ f_1(x_1, x_2) = (x_1 - 1.5)^2 + (x_2 - 1.5)^2 + \text{σταθ.} \]
Η $f_1$ είναι συνεχής και \textbf{coercive} ($\lim_{||x|| \to \infty} f_1(x) = \infty$).
Το σύνολο $S_1$ ορίζεται από: $x_2 \geq x_1 + 2$, $x_1 \leq 2$, $x_2 \geq 0$.
Περιέχει π.χ. την ημιευθεία $x_1 = 0, x_2 \geq 2$, άρα είναι \textbf{μη φραγμένο}.
Είναι όμως \textbf{κλειστό} σύνολο (τομή κλειστών ημιεπιπέδων).
Επειδή η $f_1$ είναι coercive και το $S_1$ κλειστό (και μη κενό), το ολικό ελάχιστο \textbf{υπάρχει}.
\textbf{Απάντηση:} Η ελαχιστοποίηση είναι εφικτή.

\textbf{β) Μελέτη για $f_2, S_2$}
\[ f_2(x_1, x_2) = (x_1-2)^2 + x_2^2 \]
Το σύνολο $S_2$ είναι:
\[ (x_1 - 1)^2 + (x_2 - 2)^2 \leq 4 \]
Αυτό περιγράφει έναν δίσκο (με το σύνορό του).
Ως δίσκος, το $S_2$ είναι σύνολο \textbf{κλειστό} και \textbf{φραγμένο}, άρα είναι \textbf{συμπαγές}.
Η $f_2$ είναι συνεχής συνάρτηση.
Σύμφωνα με το \textbf{Θεώρημα Weierstrass}, κάθε συνεχής συνάρτηση σε συμπαγές σύνολο λαμβάνει ελάχιστη και μέγιστη τιμή.
\textbf{Απάντηση:} Η ελαχιστοποίηση είναι εφικτή.
\end{answer}

\end{document}
